\documentclass{IEEEtran}
\usepackage[utf8]{inputenc} 
\usepackage{times}
\usepackage[T1]{fontenc} 
\usepackage[magyar]{babel}
\author{ Bodor Ádám}
\title{Sinistra körzet -- Egy regény fejezetei}
%\date{}
\begin{document}
\maketitle

\vspace{1cm}

Digitális Irodalmi Akadémia © Petőfi Irodalmi Múzeum • Budapest • 2011

\section{(Borcan ezredes esernyője)}

Két héttel azelőtt, hogy meghalt, Borcan ezredes magával vitt terepszemlére a
dobrini erdőkerület egyik kopár magaslatára. Arra kért, tartsam nyitva a
szemem, főként az út menti berkenyés bozótot figyeljem, megérkezett-e már a
csonttollú madár. Ősz közepe volt, a cserjés idegen hangoktól zsibongott.

Az erdőbiztos szemléje máskülönben abból állt, reggelente ellátogatott a
medvészetbe, szemrevételezte az állományt, majd hazatérőben végigbaktatott
valamelyik hegyháton, s miközben magába szívta a rezervátum bódító csendjét, a
völgyek mélyéről áradó patakzúgást, jelentést fogalmazott a látottakról. Most
azonban járatlan ösvényeken, a hegyi vadászok útjelzéseit követve egyenesen
titkos kilátóhelye felé tartott. Állítólag megjelentek a csonttollúak,
nyomukban az erdővidéket telente meglátogató láz, amelyet Sinistra körzetben
ki tudja, miért, tunguz náthának neveztek.

Borcan ezredest a tetőn kőből rakott, mohával bélelt pihenőhely várta,
közelébe érve jégesőkre való, bőrből készült hegyivadász-ernyőjét a fűbe
ejtette, megoldotta köpenyét, és mindjárt kényelembe helyezte magát. Sapkáját
is levetette, nehezéknek néhány zuzmótól tarka követ dobott rá, aztán
hajadonfőtt, szélben lobogó hajjal, rezgő fülcimpákkal órákon át meredt
távcsövével a keleti látóhatárra.

A fenyvesből éppen csak kibukkanó bérc már a Pop Ivan gerincéhez tartozott,
messzire el lehetett róla látni a határon 5túlra, a ruszin erdővidék egymást
követő kék vonulataira. A legutolsó halmok mögül, talán már a róna
messziségéből sötét füst emelkedett, az égbolt nagy részét keleten, mintha
máris az éjszaka közelednék, lila függöny takarta. Ahogy a nap emelkedett,
tompultak a távoli színek, s amikor a völgyek kiteltek a délután opálos
fényeivel, az erdőbiztos eltette messzelátóját, vette a sapkáját, jelezve,
hogy a szemle véget ért.

Soha nem derült ki, megpillantotta-e a túloldali lankákon azt, amit keresett,
a csonttollút vagy a bokorról bokorra közeledő tunguz nátha valami más jelét,
és az sem, miért pont engem, az egyszerű erdei gyümölcs-gyűjtögető idegent
vitt magával aznap az ukrán határra.

Hazafelé menet már az aljban megkérdezett, láttam-e csonttollút. Amikor azt
feleltem, úgy rémlik, igen, kettőt vagy hármat, azt mondta, akkor meg fogja
rendelni az oltásokat.

Már a laktanya közelében jártunk, amikor esernyőjét megint a fűbe ejtette –
egyébként ő volt az egyetlen hegyivadász, aki a nyirkos erdőket télen-nyáron
ernyővel a hóna alatt járta –, és újra elővette tokjából a távcsövet. Túl a
patakon, a kifakult őszi réten éppen arra haladt el a vörös kakasnak nevezett
idegen. Lábával a földet alig érintve, kevélyen lépdelt a mezsgyén, amely az
erdőt a kaszálótól elválasztotta, piros haja, szakálla föl-föllobbant a fekete
fenyők előtt. Borcan ezredes addig követte messzelátójával, mígnem eltűnt egy
sereg sárgán villogó nyírfalevél között. Akkor halkan, szinte bizalmasan
megszólított:

– Mondja csak, Andrej. Nem hagytak mostanság véletlenségből magánál egy kis
csomagot? – Aztán amikor látta, mintha nem is érteném a kérdést, bután meredek
rá, hozzátette: – Mármint számomra valami apróságot. Mondjuk egy frissen
fogott halat.

Bár a kérdés furcsa volt – még furcsább az, hogy vála 6szom, miszerint nem
tartogatok számára semmiféle csomagot, mennyire elkedvetlenítette –, bizonyára
elfelejtettem volna az esetet. Csakhogy nem sokkal ezután fölkeresett a
gyümölcsbegyűjtőben a vörös kakasnak nevezett idegen. Kezében párával befutott
tasakot lóbált, alján kevés vízben egy hal villózott, mindez az erdőbiztost
illette volna meg. De Borcan ezredes akkor már nem élt.

Sinistra lakói jobbára sötétbarnák vagy feketék, a szőkeség ritka, vörös
közöttük pedig egyáltalán nem akad. Kivételnek a rezervátumbeli kantinos
kislánya, Bebe Tescovina számított, akit messzire világító berkenyeszínű
hajáról mindenki ismert a környéken. Ezért, ha valahol vörös ember tűnt fel,
minthogy arrafele a hajfestés sem volt divat, mindjárt tudni lehetett, csak
átutazó idegen.

A vörös kakas egyszerű vándornak mutatkozott, könnyű léptekkel szelte át a
lejtőket, haja, szakálla, mint az égő csipkebokor, hol itt, hol ott lobbant
föl a fekete fenyves alatt. Ősz közepén, csipkebogyóérés idején, a korai
fagyokkal egyszerre érkezett; egy reggel csak ott sötétlettek idegen
gyártmányú gumicsizmája nyomai a deres ösvényeken.

Cingár, keszeg ember volt, ukránul, románul, magyarul vagy cipszerül egyaránt
rosszul beszélt, és valószínűleg tisztességesen egyetlen errefele használatos
nyelven sem tudott. Járása is olyan kevély és magabiztos volt, mint aki nem az
idevaló népekhez tartozik. Amellett egész idejét a szabadban töltötte, mintegy
jelezve, csak azért csatangol naphosszat a Sinistra mentén, hogy a körös-körül
pipáló hegyormokat csodálja.

Dobrin környékén, ahol az idegen mondhatni naponta tiszteletét tette, a
Sinistra vize patakokra ágazott, meredek völgyek mélyültek a Pop Ivan
oldalába. A vízmosások mentén szögesdróttal borított acélrudak, betonoszlopok,
7őrtornyok, csapdákkal teli árkok kanyarogtak föl a sziklás hegygerincre: a
magasban, a vízválasztón húzódott a határ. A kerítések, árkok, akadályok
szövedéke csak egy huzatos hegyszorosban nyílott meg résnyire, ahol a régi
földút átbukott a túloldali lejtőkre, amelyek már észak idegen fényeiben
áztak.

Az utat ott kék-sárga sorompó zárta el, közelében kicsi őrszoba állt és egy
kopott tábori sátor, tele didergő katonával. Ez az egy határátkelőhely
működött Sinistra körzetben, a sorompót itt is csak hetente egy alkalommal
emelték fel néhány órára, méghozzá mindig csütörtök délelőtt. Ilyenkor helyet
cseréltek a katonai járőrök, hogy a fegyverbarátság jegyében a határsáv két
oldalán egymás területét is végigfürkésszék, ugyanakkor áthaladt az a két vagy
három civil jármű, amely az útvonalat kormányzósági engedéllyel igénybe
vehette.

A vörös vándor, bár haja, öltözéke, gondtalan délceg tartása messziről
elárulta, hogy külföldi, mégsem csütörtökön bukkant fel először, hanem egy
szombati napon. Az erdei kápolna bontásán dolgozó kőművesek találtak egy
reggel lábnyomaira, délután Géza Kökény éjjeliőr – aki nappalait álmatlanul a
falu szélén egy magaslesen töltötte – elevenen is megpillantotta, ereszkedőben
a Pop Ivan lejtőin. Úgy látszik, mint a szél, csak úgy szabadon járkált a
sövények, drótakadályok között. Az aljban többször is igazoltatták, ám iratait
– vélhetően hamisak voltak – a hegyivadászok mindig rendben találták.

Barna gumicsizmát viselt, amolyan zöld kordbársony kivágásokkal televarrt
szürke posztózekét, amilyet a Pop Ivan túlsó lejtőin hordanak, hajadonfőtt
járt, keskeny karimájú, kék vércsetollal díszített kalapja hosszú pánton lógva
a hátán fityegett. Feje búbján tarajos vörös haj lobogott, állán két oldalra
fésült hetyke szakáll.

Mivel névről senki sem ismerhette, kezdettől fogva, 8hogy a nappal is vigyázó
éjjeliőr megpillantotta – mindjárt el is keresztelte – az egyszerű és
sokatmondó vörös kakas név ragadt rá. Vállán átvetve veretektől, rézcsatoktól
ékes tarka borjúbőr táskát hordott, jobb kezében, homályosan áttetsző műanyag
tasakot lóbált, amelyben, mint valami ezüsthasú hal, fényes tálca
fickándozott. Néha megállított erdőn vagy mezőn dolgozó embereket, eladásra
kínálta nekik, holott tudhatta, tálcát arrafele nemigen használnak az emberek.
Egy darabig találgatták, igazából mit is akarhat: az emberek vásárlókedve
felől puhatolózik, avagy csak barátkozó oldalukról kívánja megismerni a
helybelieket. A hegyivadászok vagy másfél napon át nyakra-főre igazoltatták,
aztán úgy látszik, leintették őket, többet ügyet sem vetettek rá. Ilyen
pompázatos külsővel, még ha nagyon erőltetné is, emberfia ügynöknek, kémnek el
nem szegődhet.

Az erdők alját reggelente már dér szürkítette, vagy egy-egy futó éjszakai
havazás, a hamvas halmok messzire mutatták a nyomokat, amelyek rendszerint a
Pop Ivan felől vezettek Dobrin irányába. Vándorútján az idegent néha
véletlenségből egy sereg csonttollú madár kísérte. Ez a madár olyankor
költözik a Sinistra völgyeibe, amikor nyomában észak felől már a tél dermesztő
szelei közelednek. A fakó réteken lépdelve, feje fölött a kavargó madarakkal,
az idegen mintha nem is Ukrajnából, hanem valami régi képeskönyv lapjáról
keveredett volna a hazai tájra.

A csonttollút egyébként nem kedvelték errefelé; kővel hajkurászták, a
szemfülesebbje egyszerűen leköpte őket, azt tartották, ahol a madarak
seregestől megjelennek, követi őket a tunguz nátha. A láz, ami a végén Borcan
ezredest is elemésztette.

Szegény még a halála előtti valamelyik utolsó napon is fölkeresett – ilyesmit
igazán nem sokszor tett –, és szinte könyörögve faggatott a bizonyos csomag
felől.
9

– Mondja, Andrej, de becsületszavára. Nem hagytak magánál egy tasakot? Benne
egy hal, semmi egyéb. Nem baj, ha megette, csak mondja meg.

Bár megesküdtem neki, tekintetében a gyanakvás és neheztelés homályával
távozott, aztán többet nem is találkoztunk. Nemsokára Nikifor Tescovina, a
természetvédelmi terület kantinosa újságolta, az erdőbiztos eltűnt. Medvészek
és ezredesek ittak nála a kantinban, ő mindenről tudott. Hamarosan azt is ő
hozta hírül, Borcan ezredesre rátaláltak földobott talpakkal egy csupasz
hegytetőn. Sajnos nem idejében, tátott szájában már egy madár fészkelt. Később
valaki a halottat odaszögezte – csakis hegyivadásznak öltözött pecér lehetett
–, kezén át szuronyokat döfött a földbe, lábát pántokkal fogta a kövek közé,
nehogy a griffmadarak elragadják.

Ezekután keresett fel a vörös kakas, akkori munkahelyemen, az
erdeigyümölcs-begyűjtő központban. Áfonyával, szederrel, gombával
foglalkoztam, s magam is a tároló egyik kamrájában laktam kádak, dézsák,
illatos hordók között. Jól emlékszem az esetre, mert ugyanaznap járt nálam a
begyűjtőben először portékájával az új gyűjtögető asszony, Elvira Spiridon. A
hegyi lakó Severin Spiridon felesége – elárulhatom, későbbi kedvesem –
bemutatkozásul egy puttony szedret és egy tarisznya őzlábgombát hozott.

A dobrini természetvédelmi területen pár száz medvét tartottak, ezek imádják
az őzlábgombát, a szedret, az általam igazgatott gyümölcsbetakarító
szállítmányozta nekik az eleséget.

Észrevettem, Elvira Spiridon, ez a máskor nyughatatlan, remegő inda, tüzes
kígyó, parázs cinege most fél lábra sántikált, biceg. Magamban azt kívántam
hirtelen, bárcsak 10tüske ment volna a talpába, és én lennék az, aki kiveszi
neki. Bármennyire is botor kívánság volt ez, a fennvaló meghallgatott. Amíg a
szedres puttonyt hordóba ürítettem, és szitán szétteregettem az
őzlábkalapokat, Elvira Spiridon leült a küszöbre, és miközben hatalmas
fülbevaló rézkarikái körbe villámlottak, örömömre tekergetni kezdte bokájáról
a bocskorszíjat. Nem haboztam, elébe térdeltem, ölbe vettem a lábát, és saját
kezűleg bontottam ki a fehér posztókapcából. Zömök kicsi lábfeje a nyári
szénatakarások óta még mindig barna volt, illatosan borult rá az erek lila
hálója. Talpa, mint aki mindig lábujjhegyen jár, majdnem rózsaszín volt,
kicsit nyirkos és puha, ráadásul nem is tüske lapult benne, hanem egy szirom
aranyos-ezüstös bábakalács. Természetesen a fogammal szedtem ki, aztán a
körmöm hegyén megcsillantva megköpdöstem, és az ingem alá rejtettem. Elvira
Spiridon lábát a kezemben szorongattam, és ha valaki akkor megpillant, azt
hiszi, éppen bemutatkoztam neki.

És az a valaki tényleg ott ácsorgott a közelben: a küszöb elé hirtelen minden
nesz nélkül színes szegélyű árnyék vetült, a vörös kakasé. Persze hogy ő volt
az, mellén, a széles táskaszíjról, a derekán hordott bőrtüszőről veretek,
csatok vakítottak. Tasakjában, amelyet jobb kézben tartott, kevés zavaros
vízben, mint egy tálca, ezüsthasú hal fickándozott.

– Figyel rám, Andrej – szólított mindjárt keresztnevemen. – Eztet vinni Borcan
ezredes nekije. Mígnem lemegy a nap.

– Rendben – hagytam rá, Elvira Spiridon közelségétől zavarodottan –, tegye
csak oda.

Borcan ezredes akkor már nem élt, de hát ez nem az idegenre tartozott. A
tasakot, benne a hallal egy üres hordóba dobtam, és miután az idegen távozott,
Elvira Spiridon után siettem, aki félig mezítláb, bocskorát kézben lóbálva,
csillámló fülbevalóival ijedten tünedezett el a berekben. 11Bókjaim sorra
elröppentek a füle mellett, ahogy észrevettem, a találkozás a vörös kakassal
őt is elkedvetlenítette.

Akkoriban egyébként a vén csataló Aranka Westinnek csaptam a szelet. Apró
jelekből ítélve közömbös ő sem volt irántam, s arról álmodoztam, egyszer egy
éjszaka, mialatt borbély élettársa a laktanya szobáit járva a hegyivadászokat
nyírja, ő lenge pendelyben vagy éppen teljesen ruhátlanul kiszalad a faluból,
a holtág mentén egyenesen az erdeigyümölcs-tárolóba, ahol magányosan éltem.
Varrónőként ő is a hegyivadászoknak dolgozott, és a valóságban én voltam az,
aki foszlott gallérokat, fityegő gombokat használva ürügyül, néha kései
órákban is fölkerestem.

Ez történt a vörös kakas látogatása után is. Éjszaka a vadludak ébresztettek,
őket akkor ősszel a keleti rónaságot elborító füst terelte Sinistra ormai
fölé. A csend süket és néma deres éjszakái megteltek a vonuló madarak
gágogásával, nyikkanásaival, elcsukló hangjuk – mint néha a bakter klarinétjáé
– bekúszott a kéményeken, s virradatig motozott a tűzhely hamujában. Idegeket
borzoló nyivákolásuk mindig magányomra ébresztett, s eszembe juttatta Aranka
Westint.

A kertek mélyén, a lombtalan szilvafák rácsai mögött még világított Aranka
Westin ablaka. Letéptem zubbonyomról egy gombot, s néhány kerítésen
átszökellve nemsokára megzörgettem az ablakát. Kinyúlt a zubbonyért, s
miközben öltögetett, ezt kérdezte:

– Mi az ördögöt keresett magánál a vörös idegen?

– A kakasra gondol? Nem is emlékszem, azt hiszem, semmit. Csak egy jó
árnyékszék iránt érdeklődött.

– Andrej, Andrej, bele ne keveredjék valamibe. Mindenki tudja, magánál hagyott
egy tasakot. Benne szép ezüsttálca.

Bosszantott a dolog, hazatérve a gyümölcstárolóba, a halat, amely még mindig a
hordóban vívta tusáját, kivittem 12az udvar végébe, s a latrina gödrébe
pottyantottam. Hallgatni szándékoztam róla s a vörös kakas látogatásáról. Nem
szerettem volna kétes históriákba keveredni, hogy a végén még kitiltsanak a
területről. Évekkel korábban titokban tudomást szereztem arról, az egyik
közeli természetvédelmi területen él kényszerlakhelyen Béla Bundasian, a
fogadott fiam, őt kerestem. Azért szegődtem el erdei gyűjtögetőnek, hogy
nyomára bukkanjak. Kár lett volna most óvatlanul mindent, amit eddig elértem –
már a diszpécserségig vittem –, könnyelműen elrontani.

És volt a dologban valami: a következő nap hajnalán újra a vörös kakas
keresett. Ápolatlan volt, csapzott, combig csatakos, lucskos, ahogy a gazzal,
fűvel benőtt réten sebtében keresztülcsörtetett. Még a haja sem tüzelt most,
inkább a bőre, orra és füle hegye szikrázott, fénylett, egyszerre a rémülettől
és haragtól.

– Embernek fia, Andrej – vett elő sziszegve. – Miért nem mondta nekem, ezredes
Borcan nem él!

Miért is. Megvontam a vállam: csak.

A halat kereste, s amint megtudta, nem ettem meg, és pillanatnyilag hol
található, rohant érte a sárból elővájni. A holtág vizében tisztára sikálta,
lapiba csomagolta, eltette tarka szőrös borjúbőrtáskájába, és távozott. A
vörös kakas örökre eltűnt Dobrinból.

Borcan ezredes helyébe a dobrini hegyivadászokhoz új erdőbiztos, Izolda
Mavrodin érkezett. Valamelyest megváltozott az életem. Eltűntem onnan egy
fergeteges tavaszi napon magam is.

Sok évvel később, zsebemben görög útlevéllel, vadonatúj négykerék-meghajtású
metálzöld Suzuki terepjárómmal 13végiggurultam Sinistra körzet útjain, és
Dobrinban töltöttem szinte egy napot. A Baba Rotunda-hágó felől érkeztem,
gondoltam, megnézem, hogyan irul-virul a kakukkfüves réteken egykori kedvesem,
Elvira Spiridon, aki férjével, Severin Spiridonnal egy boronaházban lakott a
tetőn, egy út menti tisztáson. De telkük helyén már nem állt semmi, csak egy
halom eső- és jégverte kövesedett sötétkék üszök. Körülötte zsenge sárga
fűszál, friss hajtás csalán és sáfrány növögetett. Alighanem a hantjuk
lehetett.

Késő délutánra járt, a keleti égbolton a bánat felhője, hatalmas narancsvörös
gomolyag világított. Újabban egy-egy ilyen távolodó, tornyoktól ékes, pompás
habos felhő, elmerülőben a közeledő este lila leplei között, a múló időre
emlékeztetett, és kicsit elszomorított. A terepjárót az út mentén hagytam, s
lógó orral jártam végig az erdőszél néhány jól ismert zegét-zugát.

Előttem, a tisztáson kanyarogva jégből vagy ki tudja talán üvegből való kettős
csík ragyogott a felhő visszfényében. Az izzó tavaszi füvön saját sítalpaim
egykori nyomai kanyarogtak el, tova az erdő homályába, odatapadtak a földhöz,
megmaradtak a legutolsó télről, amit évekkel korábban pont ott, a hágón
töltöttem. Aki valaha is sítalpakon járta az erdőt, jól tudja, ha az ember
néhányszor saját nyomain végighalad, alatta megkeményszik a hó, néha kicsit
megolvad, majd újra meg újra megfagy. Egy ilyen páros nyomvonal, ezüstös
fényektől selymes pántlika, ha elolvad is, csak nyár elején enyészik el
végképp. De van közöttük olyan, amelyik soha.

Azon az utolsó télen nap mint nap erre jártam el sítalpakon a Kolinda-erdő
búvópatakjai felé. A föld alatti nyirkos odúkban, barlangokban néhány
engedetlen vadóc húzta meg magát a hegyi vadászok elől; sem felszólításra, sem
kö 14nyörgésre nem voltak hajlandók előbújni. Előbb úgy volt, csapdákat,
csapóvasakat állítok nekik, aztán a vége az lett: a járatokat egyszerűen
beöntöttük cementtel. Hátamon a cementeszsákokkal heteken át csúszkáltam
errefelé, mindig ugyanazokon a nyomokon. A cement nehéz, súlyom alatt a hó
gyémánttá kövesedett.

Már-már elmerengtem régi dolgaim fölött, amikor a közelben két darab vörös
parókát találtam; fenyőgallyra akasztva, himbálózva szikkadoztak a szélben,
föl-föllobbantak a bánat-felhő fényében. Vesszővégre tűzve közelebbről is
megvizsgáltam őket, egyik fejre való volt, a másik formáját tekintve
szakállnak. A tisztás egyik homályos szegletében, elnyúlva a nyálkás tavalyi
avaron, álmában nagyokat horkantva, színes legyektől körüldongott fiatalember
aludt. Oldalán tarka szőrös borjúbőr táska, mellette fölborult üres üveg.
Valakire nagyon hasonlított, elsiettem onnan.

Külföldi lévén, bejelentkezés után a dobrini fogadóban béreltem szobát, de a
sötétség beálltával – közben persze kortyoltam is egy keveset – kiosontam, és
az estét régi barátnőmnél, Aranka Westinnél töltöttem. Tőle tudtam meg, Borcan
ezredesnek – akit halála után még külön halálbüntetéssel sújtottak – cimborája
volt a lengyel határőr ezredes, valamiben törhették a fejüket: a lengyel
mindig egy hal hasába rejtve juttatta el hozzá üzeneteit, ha nem éppen valódi
dollárokat.

Hallani sem akartam többet az ügyről.

Az még a történethez tartozik, hogy Aranka Westinnel, bár jócskán eljárt
fölöttünk az idő, az éj leple alatt újra összedörgöltük a csülkeinket.
Ütőeremet tapogatva, ernyedten heverésztem mellette, és már szinte
fontolgattam, maradhatnék közelében legalább még egy napot, amikor a
magasságos égből nyivákolás, távoli klarinétvijjogás hal 15latszott: a Dobrin
felhői között megszólaltak a vadludak. Úgy látszik, végképp odaszoktak. Az
éjszakai csendben tisztán hallatszott, délről, a Kolinda-erdő felől
közelednek, és a Dobrin fölé érkezve hirtelen északnak kanyarodnak, a Pop Ivan
felé. Kisujjam begyében is őket éreztem, esküszöm, nincs nyugtalanítóbb hang
az övékénél.

Úgyhogy amikor hajnaltájt értem jöttek a hegyivadászok, és tudatták, mivelhogy
kijelölt szálláshelyemet, a fogadót titokban elhagytam, tartózkodási
engedélyemet bevonják, és örökre kitiltanak Sinistra körzetből, én már
réges-rég éberen vártam a reggelt, hogy végre mehessek innen.
16
\section{(Andrej dögcédulája)}

Egy tavaszi napon érkeztem kerékpáron a Baba Rotunda-hágóra, onnan
pillantottam meg először azokat a kevély ormokat, amelyek tövében később
szinte elfelejtettem addigi életemet. A délután narancsos fényeiben hosszú,
éles árnyékokkal a Sinistra medencéje terült elém. A völgy alján, a folyó
kanyarulatai mentén füzes sötétlett, túlsó partján ritkás házsor kígyózott, a
messzi napsütötte lejtőkön zsindelyfedelek csillogtak, leghátul, a fenyves
fekete gallérja fölött villóztak a Pop Ivan és a Dobrin jeges tornyai.
Mögöttük üveges zöld idegenséggel az északi égbolt.

Több út onnan nem vezetett tovább, a szemközti meredek falak alatt lehetett az
a természetvédelmi terület, amelynek közelében készültem meghúzni magam.
Valahol a rengeteg mélyén élt Béla Bundasian, a fogadott fiam. Évek óta őt
kerestem.

Az országút, miután lekanyarodott a hágóról, egy ideig a vasúti töltés mentén
haladt, aztán a sínpár hirtelen eltűnt egy alagútban, amelynek bejáratánál a
bakter klarinétozott. Később, útban a falu felé a töltés megint az országút
mellé szegődött, nemsokára egy vicinális is utolért, és a szerelvénnyel
majdnem egyszerre érkeztem a sinistrai szárnyvonal végállomására.

A sínek végében földszintes, kopott épület állt, ereszéről festett deszkadarab
lógott, rajta Dobrin, a falu neve. Valaki még sárral alája mázolta a falra:
City. Tavasszal, estefele érkeztem meg Dobrin Citybe.

A kerékpárt egy palánknak támasztva várakoztam, vo 17nuljon el a sok
nesztelen, gumicsizmás, bocskoros utas, megfordult a fejemben, ha valamelyikük
megtetszik, beszédbe elegyedem vele. Először jártam Dobrinban.

Az állomás fölött fafüst lengedezett, errefele fával fűtötték a mozdonyokat,
néhány gomolyag, mintha a távozó utasok húznák maguk után, fölfelé kúszott, a
főutcán. Az út túloldalán, a rakodónál olajbarna férfi támasztotta a falat, s
mintha a füst csípné a szemét, hunyorogva méregetett az elvonulók közt támadt
réseken. Ujjatlan, piszkos fehér trikót, foltos katonanadrágot, mezítlábra
húzott szandált viselt. Nem készültem megszólítani, de amint az utasok
szétszéledtek, ő ugrott le a rámpáról, és a kiürült térségen át egyenesen
hozzám sietett.

– Úgy látom – szólt halk, olajos hangon –, szállást keresel.

– Valami olyasmit.

– Mert én tudok egyet.

Így ismertem meg Nikifor Tescovinát. Neve nyomban kiderült, láncon függő
bádoglapocskán, jól látható helyen a nyakában viselte. De ő az enyémre nem
volt kíváncsi, a kézfogást is elhárította; a kilétemet most ne feszegessük,
mondta, amíg személyesen föl nem keres Borcan ezredes. Az erdőbiztos majd dönt
a nevem felől is. Ő a dobrini hegyivadászok parancsnoka.

– És ha nem vetted volna észre, itt senki nem jár kerékpáron. Neked sem lesz
rá szükséged többé. Hagyd itt, majd elviszi valaki.

Egy lépéssel mindig előttem haladt, amint a völgy alján elnyújtózó falun
végigbaktattunk. Csupasz lába a szandálban megtelt porral, hogy lemossa, néha
belesétált egy-egy pocsolyába. Számára már megérkezett a nyár, holott alighogy
a nap eltűnt a nyugati magaslatok mögött, a tócsák pereme jegesen bebőrözött.
A falu fölött, egy közeli meredek hegyoldalon menyét alakú, keskeny hófolt
világított. 18Arról ereszkedett befele fenyőrügy illatával rakottan a hűvös
esti szél. Dobrin City főutcáján a villanypóznák körül elvágott vezetékeket
lebegtetett.

– Itt minden a hegyivadászoké – magyarázta halk, olajos hangján Nikifor
Tescovina. – Az a hely is, ahol lakni fogsz. Ők gondoskodnak itt a népről.

– Eddig csak képeken láttam belőlük – szóltam én is, lehetőleg halkan –, de
hallomásból tudom, a hegyivadászok jóravaló, rendes emberek.

– Ők nagyon. És majd mondd azt nekik, hogy elvesztek az irataid. Puiu Borcan
ezredes majd úgy fog tenni, mint aki elhiszi.

– Tényleg, az irataim – torpantam meg hirtelen. – Az ülés alatt, a kerékpár
csővázában tartottam őket. Jó volna értük visszafordulni.

– Ó, hadd. A kerékpárodat már elvitték onnan. Többet ne gondolj rá.

A falu vége felé, egy fedeles fahíd alatt fehér zuhatagokkal átvonult a patak,
partján egy törpe ült, és a lábát áztatta. Nikifor Tescovina nemsokára letért
az útról egy sikátorba, amely hamarosan ösvénnyé keskenyedett, és egy gazzal
benőtt nyirkos holtág mentén kivezetett a kertek közül a mezőre. Végében,
néhány fenyő, fűz és kutyabenge szomszédságában színes kövekből rakott,
horpadt fedelű épület sötétlett. Valamikor vízimalom lehetett, de a patak vagy
folyó elköltözött mellőle, s egymaga maradt a réten. Kitört ablakai zugában
madarak fészkeltek, a zsindelytető hasadékain az égről, mint színes pengék, az
alkony fényei villóztak. Az egykori berendezést, a tengelyeket, őrlőköveket
kiszerelték, a falon tátongó hatalmas réseken át most a rét esti illata
fújdogált.

Nikifor Tescovina a kongó falak között egyenesen az emeletre kaptatott,
megállt egy kitárt, kissé megroggyant ajtó előtt. Amolyan kis raktárféleség
lehetett, egykori szer 19számoskamra, egyik sarkában most frissen tépett
fenyőgallyból fekhely sötétlett.

– Ez az – mondta Nikifor Tescovina –, ez az a hely, ahol meghúzhatod magad.
Senki nem fog kérdezni tőled semmit.

– Honnan tudtad, hogy jövök?

– Azóta, hogy betetted a lábad Sinistra körzetbe, Borcan ezredes minden
lépésedről tud. Ez a táj vonzza a hozzád hasonló embereket. Aki egyszer
elindult fölfelé a Sinistra mentén, meg sem áll Dobrinig.

– Megnyugtattál. Akkor azt is tudja az ezredes, hogy csak egy egyszerű vándor
vagyok.

– Persze hogy tudja. És mondd, egyszerű vándor, mivel óhajtasz foglalkozni?
Sokoldalú embernek látszol.

– Az erdőt szeretem nagyon, a fát és a bokrot. Mondjuk értek a gombához,
gyümölcshöz, dolgoztam már piacokon. Elmehetek, ha kell, rönktelepre, a
fahántolókhoz is. Vagy szükség esetén csapdákat állítok.

– Nem hangzik rosszul. Majd beszélek az ezredessel. De amíg ő maga föl nem
keres személyesen, kérlek, ne hagyd el ezt a helyet. Úgy értem, ki se lépj a
házból.

– És a nagydolgomat, ha lenne, engedelmeddel hova végezhetem?

– Legjobb, ha kitolod a feneked az ablakon.

Nikifor Tescovina tenyerét a homlokához érintve intett búcsút. Mire a rét
végét elérte, ahol a falu első kerítései kezdődtek, elnyelte a szürkület. A
romos párkányon könyökölve bámultam utána, mígnem nagy szárnysuhogás közepette
kirepült mögülem egy bagoly.

Nikifor Tescovina napokon át nem mutatkozott. Reggelente a bejárat reteszén
kicsi tarisznya lógott, benne egy üveg víz, néhány darab dermedt főtt krumpli,
hagyma, maréknyi aszalt szilva, pár szem mogyoró. Ezek a napok a főtt
krumplival, az asztalt szilvával, mint a völgy fölött 20elsiető ködök,
hamarosan egybeolvadtak, attól fogva sokáig nem tudtam, hétfő van-e, szerda
avagy szombat. Az idő múlását a Dobrin oldalában a hófoltok alakváltozásai
jelezték.

Egy reggel aztán a himbálózó tarisznya mellett újra maga Nikifor Tescovina ült
a küszöbön.

– Örülök, hogy ilyen jókat alszol – mondta. – Bár gyakran megfordulok
errefelé, nem zavartalak, hadd pihend ki magad. Azért közben elbeszélgettünk
rólad Puiu Borcan ezredessel.

– Csakugyan ráér törődni velem?

– Hajjaj. Hát ő az erdőbiztos Dobrinban, vagy nem? Hamarosan eljön, mert látni
akar. A dolog úgy fest, itt maradhatsz.

– Ha tényleg elintézted, egyszer meghálálom neked. Szeretném vinni valamire.
És nekem valami azt súgja, itt fog kikerekedni az életem.

– Az könnyen meglehet. Borcan ezredesnek tetszik, amit a fejedben forgatsz.
Arra gondolt, ha az erdei gyümölcsöt illetően szándékod komoly, lenne értelme
a dolognak. A betakarított gyümölcsöt hordókban, dézsákban itt lehetne tárolni
a malomban.

– Nos, szerintem is.

– Te pedig jó nagyokat alhatnál mellette. Az erjedő gyümölcs illata altat.

– Akkor máris érdekelne, hogy áll a vidék például szeder dolgában. Én ugyanis
főleg áfonyára, szederre gondoltam.

– Hm, ezt pontosan nem tudom. És az igazat megvallva, a dolog a medvéktől is
függ, mit kívánnak majd. Ők fogják megenni, amit te leszüretelsz. Tudod,
tartanak belőlük itt a területen vagy százat, százötvenet. Ezért is tetszett
az ötleted Borcan ezredesnek.

Naphosszat az ablakban könyököltem, az olykor konok, máskor szeszélyes
ábrázatú hegyormokat bámultam, és 21Borcan ezredesre vártam. De a réten, amely
Dobrin City és a Sinistra vize között nyújtózott el hosszan, heteken át csak
az elhúzó varjúseregek és a felhők árnyéka vonult. Jött a tavaszi eső
nyugatról, Sinistra felől, s ha a felhő a Dobrin falainak ütközött, napokon át
kóborolt a jeges tornyok között. A bércekre néha minden oldalról könnyű felhő
ereszkedett, ráidomult a hegységre, mint valami szobrot takaró lepel; amikor
napok múltán föllebbent, a Dobrin megint ott állt szikrázó fehéren, miközben
körülötte kitavaszodott. Ha Nikifor Tescovina tarisznyájával véletlenségből
estefele érkezett, a langyos küszöbön ültünk, a holtágból feltörő
boroszlánillatban.

– Láthatod, részünkről teljes a bizalom – mondogatta Nikifor Tescovina. –
Meglásd, aligha fogja valaki megkérdezni, honnan, merről jössz. Ne is mondd el
magadtól senkinek. Ha valaki véletlenségből mégis kérdezget, netán faggat, hát
hazudj.

– Ühüm. Az lesz. Remélem, beleszokok. Majd mindenkinek mást mondok.

– Érted a dürgést, azt látom. És máris felejtsd el a nevedet. Annyira, hogy ha
véletlenségből meghallod a közeledben sziszegni, meg se rezzenj. Mindenhez
fapofád legyen.

Napnyugta után Dobrinra sűrű vak sötétség ereszkedett, a házak fekete
körvonala fölött csak a laktanya távoli ablakai világítottak, néha fényjelek
villantak a hegyivadászok őrtornyain. Az éjszaka felhői között a Dobrin
villámai derengtek, a távoli morajlásokat át- meg átszőtték a bagoly
rikoltozásai a berekben. A ködös, sárga hajnalok már mindig az ablakban
könyökölve találtak.

Egy napon Nikifor Tescovina a kislányával érkezett. A gyerek rövid vörös haja,
mint az érett őszi berkenye, már messziről áttűzött a ködön. Már a malom
közelében jártak, amikor észrevettem, apa a lányát pórázon vezeti. Kőhajítás
22ra a bejárattól kikötötte egy határjelző cövekhez, és egymaga tért be az
épületbe.

Aznap Nikifor Tescovina egy üveg denaturált szeszt is hozott, melléje egy
csuprot és egy lyukakkal telefúrt lábasban faszenet. Kioktatott, ahhoz, hogy
az ember megihassa, a szeszt szenen kell átszűrnie valamilyen más edénybe. Ha
nincs kéznél faszén, megteszi a közönséges tapló vagy az áfonya.

– Kezdetben hányni fogsz tőle, de aztán megszokod.

– Biztos.

Ő máris töltögetni kezdte az italt a szénre, alája tartotta a bádogcsuprot, és
leste az első cseppeket.

– Hamarosan dologhoz láthatsz, az ezredes megrendelte a kádakat, a csebreket.
Már felfogadta a gyűjtögető asszonyokat is. Nyüzsögni fognak körülötted, de te
nagyon vigyázz magadra. Ahogy már mondtam, mindenhez fapofád legyen.

– Újabban nagyon bírom fegyelmezni magam.

– Akkor majd arra is ügyelj, tudjál viselkedni, ha egy bizonyos Géza Kökény
nevűvel találkozol. Azt fogja mondani, nem akárki, mellszobra áll a Sinistra
mentén. De te ne higgy neki.

– Nem fogom végighallgatni.

– Ez a helyes beszéd. Az ott pedig a kislányom, Bebe. – Ezzel nyitott
tenyérrel kimutatott a rétre, ahol a cövekhez kikötve a vörös hajú gyerek ült
a fűben. – Majd megismered, még csak nyolcéves, de úgy veszem észre, el akar
menni tőlem.

– Ne engedd.

– Szerelmes lett Géza Hutirába.

– Nem ismerem az illetőt. Biztos álneve neki.

– Hm, ki tudja. Ő a rezervátum meteorológusa. Úgy veled egykorú, jó ötvenes.
De neki földig ér a haja. Övé a kislányom, Bebe szíve.
23

Négy, öt vagy hat hete laktam már az elhagyott vízimalomban pockok, denevérek,
gyöngybaglyok között, amikor végre személyesen is fölkeresett Puiu Borcan
ezredes. Eljött, hozta az új nevemet. Sinistra erdőségeire aznap pár órára
visszatért a tél. A virágzó rétre is jeges felhő ereszkedett, a holtág halmai
között üveges kása csillogott, a magasból havas tisztások világítottak a
falura. Elhúzó ködfoszlányok között pillantottam meg a két közeledő alakot,
egyikük pártfogóm, Nikifor Tescovina volt. A másik, a tiszti köpenyes, táskás
arcú, nagy fülű ember, sapkáját homlokán igazgatva közeledett, kezében nagy,
fekete esernyőt lóbált. Bár a levegő még tele volt a távolodó fergeteg jeges
cseppjeivel, ő az ernyőt nem nyitotta fel, ázott fekete anyaga, mint az alvó
denevér szárnya, lankadtan csüngött. Az erdőbiztos nyakában hatalmas távcső
himbálózott.

Később, miután bizalmát némiképp kiérdemeltem, magam is belenézhettem
messzelátójába. Egy alkalommal a berekbe kísértem, s ő, amíg egy helyen betért
a cserjésbe, hogy elvégezze a dolgát, rám bízta ernyőjét és a látcsövét. A
forradalom ünnepe volt aznap, tudtam, a patak partján a hegyivadászok
tollaslabdáznak a dobrini vasutasokkal. Emlékszem, a két- vagy háromméteres
lengedező szűz fű fölött látszott is az ide meg oda cikázó apró hófehér madár.

Szóval, nyakában a távcsővel, kezében lankadt ernyőjével Puiu Borcan ezredes
megállt a küszöbön. Tekintete bús, kicsit nedves volt, füle cimpáján
átderengett a távoli havas tisztások fénye, sapkája alól kikunkorodó haja,
álla borostája még tele volt az elvonult jeges eső cseppjeivel.

– Szóval maga az.

– Én.

– És mi a neve?

– Nem tudom. Elvesztek az irataim.

– Akkor jó. Minden rendben.

Elővette zsebéből azt az óraláncon függő, csillogó bá 24doglapocskát, amelyre
frissen belevésve sötétlett: Andrej Bodor. Az álnevem. Saját kezűleg
akasztotta a nyakamba, tarkóm alatt csípőfogóval összesajtolta a szabad
végeket, s a fém máris melegedni kezdett a bőrömön. Tetszett is az új nevemből
az Andrej nagyon.
25
\section{(Aranka Westin ablaka)}

Hetek, hónapok vagy talán már évek óta éltem Andrej Bodor álnéven Sinistra
körzetben, amikor a keskenyvágányú erdei vasútnál megüresedett egy pályaőri
állás. A kisvasúton, bádoggal bélelt tehervagonokban, kiselejtezett
bányászcsilléken gyümölcsöt, lódögöt s egyéb eleséget szállítottak a
természetvédelmi területre a medvéknek. Ugyanott, a rezervátum kerítései
mögött távol a világtól, élt fogadott fiam, Béla Bundasian, aki miatt az
északi hegyvidékre költöztem. Így aztán, amint tudomást szereztem róla, hogy
Augustin Konnert pályaőrt egy reggel a sínek mentén találták több darabban,
azonnal kértem felvételemet a helyére.

Az ügyben, bár gondolom, nemcsak én pályáztam, hamarosan kihallgatásra
rendeltek. A laktanya folyosóján, várakozás közben találkoztam a dobrini
borbéllyal, akit pont akkor tanácsoltak el a körzetből. Azzal a nappal
kezdődött életre szóló barátságom Aranka Westinnel.

Dobrinban akkoriban szűnt meg az erdeigyümölcs-betakarító központ, aminek
addig a diszpécsere voltam, és bár nyomban szélnek eresztettek, továbbra is a
tárolóban laktam, egy kamrában húztam meg magam dézsák, hordók között. A
begyűjtő a Sinistra egyik holtága mentén, egy elhagyott vízimalomban működött;
a patak még régen, egy tavaszi áradáskor elszökött mellőle, a kőépület egymaga
maradt a réten, néhány fenyő, berkenye, fűz és kutyabenge szomszédságában. A
helyet magas, sárgára mázolt pózna jelezte, borús időben messzire világított,
hogy a környező 26hegyoldalakról érkező gyümölcsszedő asszonyok a bolyongó
ködök között is könnyen odataláljanak.

Az emlékezetes nap reggelén az immár feleslegesen meredező sárga póznán egy
papírzsákból tépett keskeny cédula lobogott, néhány szénnel írott,
hevenyészett, ákombákom betűkkel odarótt szó sötétlett rajta: „Siessen,
Andrej, a hivatalba.” Az üzenet nekem szólt, Coca Mavrodin ezredes, a dobrini
hegyivadászok új parancsnoka írta saját kezével, ráismertem a szárukkal
fordított irányba mutató N és S betűkről. A papírszeletet még hajnalban
tűzhette oda az ismeretlen küldönc, a pózna körül a deres talajon gumicsizmás
lábnyomok motoztak. Ősz vége felé járt.

A mezei ösvényen, a patakot szegélyező füzesek mentén rövidítettem, nem is
találkoztam senkivel, csak Géza Kökény mellszobra tetszett át a kerteken, a
kopaszodó ágak szövedékén. Túl a vízen terült el Dobrin, a falun is túl,
félig-meddig már a hegyoldalra épülve álltak a hegyivadászok laktanyái. Mint
valami hatalmas, leszakadt sziklatömbök szürkélltek a meredély aljában.
Mögöttük, a határig nyúló völgyek valamelyikében élt Béla Bundasian, a
fogadott fiam.

Egyszerű, hétköznapi kis história a miénk. Fogadott fiam, Béla Bundasian
egyszer nem tért haza Moldvából, ahova vásári feketézőkhöz rendszeresen eljárt
kottapapírért, attól fogva többet nem láttam. Eltűnése után pár napig, mondjuk
egy vagy két hétig még azt lehetett hinni, megint a vérmes Cornelia
Illafeldnél múlatja az időt – kedvese a Kárpátok kellős közepén lakott
valahol, egy alagút közelében –, de miután hetek múltán sem került elő, és
életjelt sem adott magáról, biztosra lehetett venni, belekeveredett valamibe.

Csakugyan az történt, belekeveredett. Az is csak úgy jó másfél évvel később
derült ki, Béla Bundasiant valahová az 27ukrán határ közelébe telepítették,
egy természetvédelmi területen él Sinistra körzetben. Mindezt egy ismeretlen
jóakaró tudatta velem, tűvel karcolt írással, egy húszas pénzérmén, amit az
illető – nem kizárt, valami jobb indulatú hatósági ember – a postaládámba
hullatott.

Tudom, az ilyenszerű hír távolról sem valami vidító, engem mégis lázba hozott.
Fölmondtam a piacfelügyelőségen, ahol akkoriban ellenőrként, alkalmi
gombaszakértőként működtem, és északra utaztam, hogy munkát vállaljak
valamelyik Sinistra menti hegyi településen. Mindvégig a jobbik szimatom
vezetett, a végén – persze évek teltek el közben – pont itt kötöttem ki, a
bizonyos rezervátum közelében, a nyirkos, huzatos Dobrinban.

Az erdei gyűjtögetés a szűkös időkben biztos kenyérnek számított, az ember a
kincstári puttony mellett a maga tarisznyáját is teleszedte. Az áfonya, szeder
vagy a rókagomba különben is sok örömöt tud szerezni az embernek. Félreértés
ne essék, nem valami neves konzervgyárnak szállítottunk, csak a közeli
rezervátumnak, ahol egy romos kápolnában és elhagyott, beomlott bányákban
tartották a medvéket. Elejtett szavakból, körmönfont puhatolózások során
kiderítettem, Béla Bundasian a Géza Hutira meteorológus házában él, túl az
erdőhatáron, a havason. Semmi dolga, csak szívességből jár ki néha
megszemlélni a szélkakasok állását a szirteken, néha leolvassa a fennsíkon
szerteszéjjel telepített műszereket. A faluba nem járt be, ezért a véletlen
alkalomra vártam, hogy valamilyen úton-módon összetalálkozzunk.

Így is, Puiu Borcan ezredes, a terület előző erdőbiztosa, mintha csak átlátott
volna szándékaimon, vonakodott aláírni a passzust, amivel eljárhattam volna
szüretelni a rezervátumba. De Puiu Borcannak váratlanul vége lett, egyik
szemléjéről nem tért vissza. Egy ideig vártak rá, hátha egy hosszúra nyúlt
kaland után mégiscsak megjelenik, de ami 28kor egy napon Dobrin City fölött,
mint egy hatalmas denevér, átrepült egy széltől hajtott magányos fekete ernyő
– egyedül ő, a hegyivadászok parancsnoka használt portyáin ilyet –, mindenki
megtudta, az ezredes nincs többé.

Puiu Borcan ezredes helyébe a dobrini hegyivadászokhoz egy nő, Izolda Mavrodin
került, becenevén Coca. Vékonyka volt, halk, átlátszó, mint egy szitakötő. Ha
látni akart, egy-két szóval ilyen papírzsákból tépett fecniken üzent, könnyen
ráismertem, mert az N és az S betűket mindig fordítva írta. A laktanya felé
vezető ösvény mentén, kórókra, csupasz vesszőkre tűzve most is ilyen szénnel
telefirkált barna papírcsíkok lobogtak. „Várják magát, Andrej, nagyon fontos
ügyben.”

Coca Mavrodin azon a napon másokat is hivatalába rendelt, előszobája tele volt
gyantaszagú favágóval, erdőkerülővel, így esett meg, hogy amíg soromra vártam,
Vili Dunkával, a dobrini borbéllyal találkoztam. Mintha nem ismerne többé
senkit, a haragtól nagy kevélyen épp sietős léptekkel távozott a hivatalból,
de én utána eredtem. Amolyan alkalmi ivócimborám volt.

Nem örült nekem sem, elmondta, siet nagyon, az első vonattal el kell hagynia a
községet, egész Sinistra körzetet. Már reggel útibatyuval, váltás
fehérneművel, kedvenc tárgyaival hívatták, úgyhogy innen egyenest az állomásra
megy. A fodrászat megszűnt Dobrin Cityben, a kocsma is bezárt; minden olyan
helyet becsuktak, ahol az emberek várakozás közben elbeszélgettek. Bizonyságul
Vili Dunka megmutatta vasúti szabadjegyét, amellyel kijelölt új lakóhelyére
ingyen utazhatott.

– És Aranka Westin mit szól az ügyhöz? – érdeklődtem.

– Semmit, őrá nem vonatkozik a dolog; továbbra is majd tiszti köpenyeket
foltoz. Természetesen itt marad.

A szóban forgó nőszemély a laktanyának varrt, és addig a napig élettársa volt
Vili Dunkának.
29

– Mert gondolom, tisztában vagy vele – folytattam –, sok-sok évig távol
leszel. Talán vissza sem térsz már ide többé soha.

– Ühüm, úgy fest a dolog. Felkészültem mindenre.

– És azt nem tudom, sejted-e, nekem mindig is nagyon fájt a fogam Aranka
Westinre. Most, hogy elmégy, majd mindent elkövetek, hogy az öröködbe lépjek.

– Aha, ez nekem is megjárta az eszemet. Hát egyszerűen nem fogok rátok
gondolni.

– Azért mondom most ezeket a dolgokat, mert egyenes embernek tartom magam.
Nehogy úgy nézzen ki, a hátad mögött cselekszem. Nem szeretném, ha a végén még
rosszra gondolnál.

– Máris el vagytok felejtve. A holmim nagy része nála maradt, úgyhogy használj
bármit nyugodtan, ami megnyeri a tetszésedet. Tornaingem, papucsom, alsóm
maradt ott, nagyjából egyezik a termetünk. Én csak ollókat, borotvakéseket,
pár pamacsot és kenőcsöt, szóval a borbélyszerszámaimat viszem magammal.
Minden egyéb a tied.

– Rendes vagy.

– Hát most mi a fenét tegyek?

– Azért még nem tudhatom, velem is mi lesz. Amint látod, engem is hívattak.

– De nálad nincs útitarisznya. Te még maradhatsz. Legalábbis egy darabig.

– Nagyon remélem. Ezért is bátorkodnék érdeklődni, el tudnál-e látni néhány
hasznos tanáccsal. Mégis, miként viselkedjem vele? Milyenek a szokásai, melyek
az asszonyi szeszélyei?

– Az ördögbe is. Te a nagy fehér cubákjával törődj, ne pedig a szeszélyeivel.
De mondjuk, amikor varr, ne próbálkozz vele. Nála mindig is a kötelesség az
első. És most már mennék, ha meg nem sértelek. Isten veletek.

– Kösz. És vigyázz magadra.
30

Ezzel Vili Dunka, a volt dobrini borbély eltávozott. A folyosó ablakából utána
bámultam, követtem a tekintetemmel, amíg átvágott a tócsáktól csillogó
udvaron, amint a porta előtt várakozott, hogy az ügyeletes tiszt kiengedje,
aztán már csak a felröppenő verebek jelezték az útját a kerítés mögött. Eltűnt
az állomás felé vezető úton, és soha többé nem hallott róla senki.

Késő délután került sor a kihallgatásra. Az erdőbiztosi székben a halottkém,
Titus Tomoioaga ezredes ült, azt mondta, ki kell mentenie Coca Mavrodint, aki
pillanatnyilag nem ér rá, közben azért tanulmányozza a pályaőri állás ügyében
benyújtott kérelmemet. Csakhogy volna egy kis bökkenő, irataim, útban a
nyilvántartó felé, elvesztek. Amíg előkerülnek, néhány megbízható ember
magánvéleményét fogják kikérni. És ha pont pályaőrnek nem is, de ki tudja,
valami futárfélének esetleg alkalmaz; szüksége lenne egy emberre, aki
üzenetekkel bejár a rezervátumba.

Úgy nézett ki, pont oda akar majd küldeni, ahonnan eddig kitiltottak. Annyi
évi várakozás után talán hamarosan találkozni fogok Béla Bundasiannal. Közönyt
színlelve, unott képet vágtam, mint akinek nemigen fűlik a foga az egész
dologhoz. És ennyi idő elteltével már nem is bírtam örülni a hírnek olyan
nagyon. Amellett, őszintén szólva, egyre csak Vili Dunka járt az eszemben, aki
azóta, zsebében az érvényes vasúti szabadjeggyel, már az állomáson várakozik.
Ha majd rövid vonatfütty hallatszik, az azt jelenti, elment. Jó volna biza,
gondoltam magamban, még aznap este fölpróbálni a papucsát.

Ősz vége volt, már szürkülődött, mire a laktanyából jövet elhaladtam a
néptelen, kutyaugatástól, kóbor ködöktől átszőtt falun. Jó pár éve már, hogy a
villanydrótokat elvagdosták, a házak estelente jobbára néma sötétségben
gunnyasztottak, most is csak elvétve villant valamerre viharlámpás vagy
faggyúmécses lángja. Egy halvány, sá 31padt ablak a varrónő, Aranka Westin
kertje mélyén derengett.

Jó ideig csak a függöny lyukain, résein leselkedtem, mint tesz-vesz odabenn,
félig-meddig megözvegyülten, amint az imbolygó mécsvilág mellett a nehéz
posztó egyenruhákat foldozgatja. Hátára kétrét hajtott vastag gyapjúkendő
terült, csücske fenekéig ért, két szárnya combjára simult, amelyet Vili Dunka
cubáknak nevezett. Fázhatott kicsit, úgy látszik, aznap még nem ért rá tüzet
rakni.

Megkerültem a házat, a fásszínben fölnyaláboltam néhány tuskót, kevés gyújtóst
is markoltam hozzá, aztán kopogtatás nélkül, csak úgy a térdemmel lenyomtam a
kilincset. Aranka Westin nyomban fölkapta fejét, de mindjárt vissza is
horgasztotta, egyszer-kétszer még rám villantott, miközben ugyancsak a
lábammal ügyetlenkedve visszacsuktam az ajtót. Ha éles volt a szeme, márpedig
a varrónőknek az, észrevehette, lábam körül remeg a nadrág, talán a
légvonattól, gondolhatta. De én akkor legalább öt éve nem voltam nővel.

Az első biztató jelre vártam, hogy például állán megenyhüljenek a ráncok, hogy
lábujjai a papucsban hívogatóan elernyedjenek, s legfőképpen arra, hogy a
tiszti köpenyt, amelynek szürke posztóból éppen új zsebeket rakott, kezéből
végre maga elé ejtse. Tudtam, nyert ügyem van, és most már azt is: varrás
közben még véletlenségből se próbálkozzam vele.
32
\section{(Coca Mavrodin neve)}

Amikor hírül hozták, hogy a Dobrin egyik szélfújta magaslatán megtalálták Puiu
Borcan ezredest, kiporoltam a vattakabátomat, sáros gumicsizmám leáztattam a
patakban, majd fölkerestem a törpe Gábriel Dunkát, hogy stuccolja meg egy
kicsit a hajamat. Puiu Borcan ezredes Sinistra körzet erdőbiztosa volt,
illett, hogy én, mint az erdeigyümölcs-betakarító központ diszpécsere, ápolt
külsővel jelenjek meg a temetésen.

Hamarosan kiderült, kár volt a sietségért, az ünnepségből nem lesz semmi:
Izolda Mavrodin ezredes, a hegyivadászok frissen kinevezett új parancsnoka jó
előre minden gyülekezést betiltott. Még csak útban volt Dobrudzsából új
állomáshelye, az északi hegyvidéki terület felé, máris megüzente, Puiu Borcan
ezredes a hegytetőn maradt, pont azon a helyen, ahol a láz leterítette, hozzá
ne merjen nyúlni senki. Még akkor sem, ha netán – és ezt már én teszem hozzá –
kezdi körüludvarolni néhány arra tévedt borz vagy róka.

Borcan ezredest a dobrini hegyivadászoknál egy asszony követte. Úgy hírlett, a
Mavrodin csak álneve, az igazi: Mahmudia, és nem bánja, ha Cocának becézik.
Érkezése előtt Dobrin Cityben kevesen aludtak, az éjszakában a várakozás, az
izgalom hangjai kígyóztak. Egy ideig azt lehetett hinni, Tomoioaga bakter
klarinétja vijjog az alagútban, máskor meg mintha megkésett vadludak vonultak
volna a völgy fölött. Éjszaka közepén, amikor az árnyékszék felé menet
átballagtam az udvaron – az esténként benyakalt denaturált szesz egyre-másra
meghajtott –, túl a 33falu sok fekete háztetője fölött sárgán derengett a köd,
a laktanya valamennyi lámpája égett, az őrtornyok fényei mint hatalmas
vattacukrok lebegtek a nyirkos sötétség mélyén. A vijjogó hang is onnan
érkezett: a hegyivadászok talpukra kötözött párnákkal fényesítették a
folyosókat, nedves újságpapírral sikálták a laktanya ablakait.

Izolda Mavrodin kora reggel, vöröskeresztes katonai terepjárón érkezett.
Sapkája ellenzőjét, a szélvédőt, a sárhányókat fehér lepedék borította, abba
valaki ujjal belerajzolta becenevét: Coca. A kocsi nyomában Dobrin City utcáin
kesernyés orvosságszag kavargott, vagy még inkább olyanszerű illat, mint az
eltaposott bogaraké; föl-fölcsapott, végighullámzott a falun, aztán mint az
esővíz, meggyűlt az út menti árkokban, az udvarokon.

Coca Mavrodin-Mahmudia még aznap, csak úgy pofára kiválogatott maga mellé vagy
tizenöt-húsz, szinte egyforma falusit – véletlenségből mind hosszú nyakú,
golyófejű, gombszemű, színtelen fiatalt –, darócaikat eldobatta, valamennyien
szürke öltönyt, hegyes orrú fekete félcipőt, ezüstösen csillámló nyakkendőt
kaptak. A hasonlatosságra a falubeliek is fölfigyeltek, az átvedlett
szomszédokat mindjárt elnevezték szürke gúnároknak. Bár kiképzésükre idő sem
lett volna, maguktól kitalálták, mi lesz a dolguk, már az első percektől fogva
szigorúan hordozták tekintetüket körös-körül, mindenfelé. Ha valamerre
elindultak, csak úgy kopogott a sok bőrtalpú cipő a vízmosta köveken.

Bár én mindjárt, a kezdet kezdetén kiporolt pufajkával, fényesre áztatott
gumicsizmával, amolyan bemutatkozó látogatáson az új parancsnoknál
tiszteletemet tettem, ő, amint végigmért, nyomban kitessékelt az erdőbiztosi
irodáról. Később mégis üzeneteket hagyott számomra itt-ott, egy-egy
jelentéktelen kis irkafirkát, aztán amikor lihegve jelentkeztem, megintcsak
elküldött. Tévedés, mondta, nem is tudja, ki vagyok, máskor így: most hagyjuk,
inkább majd 34egy más alkalommal. Bizonyosra vettem, csak próbálni,
bosszantani akar, és egy napon majd igazi szándékát is elárulja – ha nem is
nyíltan –, és akkor majd égen-földön kerestetni fog a hegyivadászaival,
kutyáival, sólymaival.

Azon az őszön korosodó ember létemre éppen Aranka Westinnek csaptam vadul, nem
is reménytelenül a szelet. Ő a laktanya számára varrt, s ha szállítmányozás
után néha őrizetlenül magára maradt, bevettem magam hozzá. Nála találtak rám
egy délelőtt nagy nyalakodás közben a dobrini szürke gúnárok. Már vittek is
magukkal.

Coca Mavrodin tudatta, mostanig vívódott, mi is történjék velem; a
gyümölcsbegyűjtő ugye megszűnt, azzal együtt az én diszpécserségem is. Akkor
pedig, mivel nem is ezen a vidéken születtem, hanem ki tudja, hol, legjobb, ha
a körzetből is hamarosan eltávozom.

– Az efféle eprészés, gombászkodás, természetjárás ideje lejárt – mondta halk,
fakó hangján –, eddig sem volt rá semmi szükség. És ami a legnagyobb baj –
tette hozzá –, nincsenek meg az iratai. Itt nem maradhat.

És, hogy mutassa, nem a levegőbe beszél, fiókjából irattartót vett elő,
összefogdosott szürke dossziét, nagy ákombákom betűkkel volt ráírva az Andrej,
a Bodor: az álnevem. Kinyitotta, mutatta, mintha nem is léteznék, a belseje
teljesen üres. Nem kizárt, valaki, mint feleslegeseket, egyszerűen eltüzelte,
eldobta őket, vagy talán maguktól megsemmisültek.

Megcsörgettem a kicsi bádoglapocskát a nyakamon, mutattam, engem Puiu Borcan
ezredes még annak rendje-módja szerint törzskönyvezett, ha kell, mégiscsak van
mivel igazolnom magam. Dobrinban, aki az erdőn dolgozott, nyakán ilyen
bádoglapocskát viselt, belevésve minden személyi adata és persze a neve. Igazi
igazolványnak errefele ez számított.

– Ha itt maradna – mondta Coca Mavrodin –, egy 35szer majd az is kellene. De
akkor sem addig, amíg él és mozog.

Alacsony, görbe, fakó nő volt Coca Mavrodin-Mahmudia, köpenye szárnyai közé
süppedve gubbasztott, mint egy matt éjjeli lepke. Szeme is mintha bőrből lett
volna, nem pislogott, csak ült rám szegzett fekete orrlyukakkal, fénytelen
nemezes hajából, a fülében hordott sárga vattákból bogárszag párállott.

– Azért, ha lehetne, én mégis maradnék – próbálkoztam. – Bármit elvállalok.
Már pályaőrnek is jelentkeztem a kisvasúthoz. Talán még megbeszélhetnénk a
dolgot.

– Értesültem a terveiről – legyintett. – De ősz végén, ha lehull a hó, a
kisvasút leáll. Nem biztos, hogy tavasszal újraindítom. Maga itt előbb-utóbb
bajba keverednék, hiszen még neve sincsen. Menjen el idejében, tisztességben,
távozzék, amíg elengedem.

Világos beszéd volt, megmarkoltam a sapkám, egyet-kettőt gyűlölettel
rávillantottam köszönés helyett, útközben az ajtó felé, kiköptem az ablakon.
Coca Mavrodin hangja a küszöbön utolért:

– No, álljon csak meg. Felőlem köphet. De én úriembernek hittem.

– Az vagyok és nem is köptem.

– Az más. Akkor mégiscsak megkérhetem egy szívességre. Van itt egy hágó, Baba
Rotunda a neve, szeretném, ha odakísérne. Nincs sok kedvem ezekhez a bölcs
hegyivadászokhoz. – Székestől megfordult, és a mögötte függő domborított
falitérképen kikereste azt a helyet, ahol az országút átbukott a túloldali
lejtőre. – Őszinte leszek, ilyen terepen ez idáig nemigen mozogtam, délvidéki
vagyok. Örülnék, ha egy ilyen civil igazítana el, mint maga. Akit aztán úgysem
látok többet.

– Legyen, nem utasítom vissza.

A bejárat előtt egy padon egymás mellett ültek a szürke 36gúnárok. Fekete
félcipőjükön fehér szegéllyel ütött ki az izzadságfolt. Gombszemük csillogott
az őszi verőfényben, pihéik fölött olcsó illatszer szaga lebegett.

– Ő az illető jómadár – mutatott rám Coca Mavrodin –, megígérte, elmegy.
Holnap reggel majd a körzet határáig kísérik, megvárják, míg felröppen és
elszáll.

A laktanya portája előtt a vöröskeresztes terepjáró várakozott. Ponyvájában
sárga nyírfalevéltől pettyes kék esővíz ringott, benne égnek görbített
lábakkal egy varjú hevert. Csak úgy hulldogáltak akkoriban az égből a madarak.

A kerítés tövében Géza Kökény, az egykori hős medvész sütkérezett, pipázott,
füstje ha utolért, megcsiklandozott a senyvedő kakukkfű illata. A tisztelgés
jeléül jobb keze ujjait homloka közelében tartotta.

A Baba Rotunda-hágóra nyolc-kilenc szerpentinnel vezetett föl a kátyútól
csillogó, vízmosta árkoktól szabdalt régi földút. Az egyetlen buszjáraton
kívül, amely Bukovina felé közlekedett, csak szénégetők, erdőkerülők és a
dobrini hegyivadászok használták. A tetőn Zoltán Marmorstein útkaparó háza
állt, a környező tisztásokon elszórva néhány tanya; széltől, esőtől szürkére
fakult boronaház. Ott álltunk szemben a Dobrin meredek falaival; kelet felé a
Kolinda-erdő sötétlett, északon a Pop Ivan menyétvörös sziklái égtek.

Amolyan ismerkedésféle volt ez, első terepszemle, Coca Mahmudia előtt
haladtam, félrehajtottam előle az ágakat, elrugdostam útjából a
fenyőtobozokat, s tapsoltam nagyokat a madaraknak, hogy idejében
felröppenjenek. Egy-két hete errefele még lángolt a berkenye, de mostanra az
ágak mind lekopasztva szürkélltek: észak dermesztő szelei elől megérkezett a
berkenyeevő csonttollú madár.

Hogy a csendet megtörjem, meg is említettem ezt az apróságot. Coca Mavrodin,
úgy tűnt, elereszti a füle mellett, csak jóval később felelt rá.

– Tudós ember, akkor sincs mit kezdenem magával – 37mondta –, egy férfi ne
adja gyümölcsre, madárra a fejét. Mi a fene terem itt meg egyáltalán?

– Áfonyára, főleg szederre hajtottam – feleltem. – Tudja, a rezervátumnak
szállítottam. A medve imádja a szedret.

– Áfonyát életemben nem láttam, de ugye, mindjárt mutat egyet. Ami a szedret
illeti, a gyalogszeder például nálunk Dobrudzsában is megterem. És ha mifelénk
ritkán havazik is, a sótól a dombok, halmok nálunk télen-nyáron fehérek.
Buckáik között százlábú szőrös indák mászkálnak, telis-tele azzal a kis
vigyorgó bogyóval.

– Érdekes lehet.

Most, hogy elbocsátottak, nem sok kedvem volt az ilyen udvariassági
beszélgetésekhez. Igen-igen bosszantott, hogy el kell költöznöm a területről.
Úgy nézett ki, pár évem csak úgy a semmibe röppen. Azt terveztem volt, ha
rátalálok fogadott fiamra, megszöktetem innen. Vagy ha neki nem volna kedve,
hát megszöktetem Aranka Westint. És most jött ez a nő, Izolda Mahmudia, és
eltanácsolt a körzetből. Csak fújtattam, köptem nagyokat mérgemben.

– És mondja, Andrej, fogalma sincs, hova lettek a papírjai?

– De igen – feleltem rosszkedvűen –, azt hiszem őurasága zsebében maradtak.

– Miféle őuraságéban?

– Hát az övében, az ezredesében. – Hanyagul előremutattam, ahol a lecsüngő
felhőfoszlányok között a Dobrin hójárta meredélyei csillogtak. A csupasz
hegytető felé, ahol lapos zöld kövek között Puiu Borcan ezredes örökre
elpihent.

– Az pech. Most már eszébe ne jusson a zsebében kotorászni. Azt mondják,
ragály terítette le. Majd tüzet gyújtatok alatta. Ne is nevezze már őt
ezredesnek.

Valamivel később a távolban, alacsonyan repülve, a rét buckáira le-lecsapva
éppen Puiu Borcan ezredes ernyője húzott el a Dobrin falai előtt.
38

– Ekkora denevért életemben nem láttam – suttogta Coca Mavrodin.

Túl Zoltán Marmorstein útkaparó házán, egy fenyőkkel benőtt hajlatban havasi
tanyaház állt, közelében pajta, szénatároló, fészer. Kerítése mentén, a deres
őszi réten feketén csillogó trágyakupacok gőzölögtek. Közöttük zsebre dugott
kézzel, fejét olykor riadtan fölkapva, egy látásból ismerős hegyi lakó,
Severin Spiridon sétálgatott. Előtte hatalmas dolmányos varjak lépdeltek,
mögötte gyapjas, tarka kutya.

A kutya vett észre előbb, fölmeredő farka a szélben lobogott, sietve vizelt
egyet-egyet. Severin Spiridon is megállt, szemét fél kézzel beárnyékolta, s a
trágyagőz áttetsző csóvái között kémlelt a távolba. Megnyitotta nyakán a
zubbonyt, később a nadrágját is megoldotta, s egyik kezét még mindig tekintete
elé tartva, idegesen ő is vizelt egyet.

– Az kicsoda?

– Severin Spiridon a neve. Majd kegyed is megismeri. Úgy tudom, a
hegyivadászok régi embere.

– Megsúgom, nem kedvelem ezeket a hegyivadászokat. Egytől egyig hiú szarvasok.

Severin Spiridon ezalatt körbejárta házát, közben valahonnan egy távcsövet is
leakasztott, most már azzal kémlelt körbe-körbe. Bizonyára a távolban
megpillantotta Coca Mavrodint meg engem is, akit látásból ismerhetett, amint a
vizenyős csapásokon bejárjuk a terepet.

– És mondja, hogy a fenébe kerültek az iratai a Borcan ezredes zsebébe?

– Szégyellem az esetet: ő azt hitte, tartozom neki, és magánál tartotta őket
zálogul. Állítólag adósa voltam egy hallal, amit a fene tudja, ki küldött
neki.

– Ezredesnek tartozni nem jó.

Déltájt a süppedős réteken jó nagyot kerülve elértük a láp 39szegélyét,
amelynek zsombékjai egészen Severin Spiridon kerítéséig hullámoztak. Bár a
boronaépületek körül akkor éppen senki nem mutatkozott, sem a gazda, sem a
tarka kutya, Coca Mavrodin a cuppogó fűcsomók között egyenesen a tanya felé
vette útját.

– Jöjjön, fiú, kerüljünk egyet. Beszélni szeretnék a távcsöves emberrel.

Mondom, a tanya környékén akkor már nem látszott sehol Severin Spiridon.
Odaérkezve később csak a levetett csizmáját találtuk meg szépen egymás mellé
helyezve a kapuban. A sárban mezítlábas nyomok a pajta felé vezettek. A kutya
szűkölése a csukott konyhaajtó mögül hallatszott.

– Elbújt – jegyezte meg Coca Mavrodin, és végigment az udvaron. – Akkor most
megkeressük.

Mivel a pajta ajtaján zár nem volt, Coca Mavrodin legelőször oda nyitott be.
Kis ideig, fél perc sem lehetett, elmerült a benti homályban, de mire én is
odaértem, már újra künn állt, a küszöb előtt. Száraz bőrszeme meg sem rebbent.

– Van egy kése?

Gombavizsgáló késemet mindig magamnál tartottam, tüstént odanyújtottam neki.
De ő elhárította.

– Én nem érem fel, menjen inkább maga. Vágja le szépen.

A pajta homályában, a mennyezetről, mint fényes pengék, világítottak a tetőn a
hézagok, a jégverte zsindely repedései. Alattuk Severin Spiridon kötélen függő
árnya imbolygott, nyakán még ott lógott a távcsöve. Mezítláb volt, ernyedt
lába körül érződött a gumicsizma szaga.

Izolda Mavrodin megnógatott:

– Siessen. Még azt hiszi valaki, én tettem.

Bementem a pajtába, fölkapaszkodtam a jászol peremére, és – nyissz – levágtam.
Severin Spiridon a szénával borított padlóra puffant, s a fény bevillanó
lemezein lát 40szott, szája még kissé gőzölög. Melléje térdeltem,
meghuzigáltam arcán a bőrt, és az ajkára tapadtam. Fújtam, szívtam, megint
fújtam, és megint szívtam, beleadtam a lelkemet, addig folytattam, amíg
megéreztem, könnyedén ő is belém köhög. Amikor szeme héja is repdesni kezdett,
hoztam egy vödör vizet, arcát, nyakát megöntöztem, s ott hagytam, hogy majd
találja maga mellett.

Coca Mavrodin egész idő alatt a ház előtt sétálgatott.

– Arra kértem, vágja le. Nem pedig arra, hogy csókolózzék vele. Hogy jut
ilyesmi eszébe?

– Csak megpróbáltam.

– De hát föltámasztotta.

A ház ajtaja mögött Severin Spiridon kutyája ugatott. A levetett csizmák
mellett elhaladva kikanyarodtunk az ösvényre, átvágtunk a réten a magára
hagyott terepjáró felé. A ponyvában a varjút ellepték valami színesen csillogó
őszi bogarak. Mint a vadludak hangja, a levegőben ökörnyál nyújtózott, a
Dobrin ormai alatt, mint kurta klarinéthangok, felhőpamacsok kunkorodtak.

– Bármi is történjék, most rágyújtok – mondtam. – Ha kegyed siet, ne várjon
meg. Majd elboldogulok, leérek egymagam, ismerem a rövidítéseket.

Coca Mahmudia beült a kormány mögé, magára csukta az ajtót, s a maga felén
letekerte az ablakot.

– Gyakran tesznek itt ilyesmit az emberek?

– Még nem kaptak rá nagyon.

– Annyit mondok, nehogy nekem még egyszer halotthoz nyúljon.

– Ha maradhatok, hát megígérem. Ha nem, hát nem állok jót magamért.

– Vésse az agyába: a halottnak az a dolga, hogy többet ne mozduljon.

A cigarettavégeket, amelyeket a laktanya körül szedegettem rendszeresen, egy
kis bádogdobozban hordtam 41magammal a zsebemben. Egy kövérebbet
kiválasztottam, és szipkába dugtam. Mivel a kocsi nem indult, a motorházon
könyökölve pöfékeltem. Onnan láttam, hason fekve már Severin Spiridon is
könyökölget pajtája küszöbén. A nyála még az arcomon feszült.

– Azért fura, hogy pont most tette – morogta Coca Mavrodin. – Éppen most, hogy
errefele jártunk. Kissé fura.

– Nem olyan nagyon. Valamikor meg kellett, hogy tegye.

– Ha fölépül, kihallgatom. Majd megkérdezem, mi a fenét bámult olyan nagyon a
távcsövével. Számoljon csak be nekem szépen.

– Azt fogja mondani, semmit. Vagy azt, csak minket. Ismerem én ezeket.

Miután megfordult a terepjáróval, Coca Mavrodin kikapcsolta a motort, hagyta,
guruljon a kocsi nesztelenül lefele a szerpentineken.

– Annyit megjegyezhetek, én is ismerem ezeket. Az egészet csak azért tette,
hogy bosszantson.

Gurult a kocsi a sok kátyú között lefele a Baba Rotunda-hágóról, Coca Mavrodin
fél kézzel tekergette a kormányt, másik kezével a fülében turkált. Biztos
javában pukkant, pattogott neki; ezek szerint nem hazudott, először járt
magaslati utakon. Az aljban átnyúlt előttem, saját kezűleg nyitotta ki az
ajtót.

– A szürke gúnárok, ahogy maga nevezi őket, reggel a körzet határáig kísérik.
Menjen, felejtse el azt az egészet.

– Kár – mondtam. – Reméltem, változtatni fog az elhatározásán. Csak átkozni
tudom magam a miatt a bizonyos halügy miatt. Minden bajom abból származott.

– Miféle halügyről beszél?

– Mint említettem, Borcan ezredes egy halat keresett rajtam, azt hitte,
rejtegetem előle.

– Hm. A halott ezredes nem ezredes.
42

Az estét – úgy nézett ki, a legutolsó lesz Dobrin Cityben, Sinistra körzetben
– Aranka Westinnél töltöttem. Volt neki egy fából faragott régi kádja, amiben
akár fürödni is lehetett. Megtöltöttem langyos vízzel, egy-egy adag itókát is
kevertem – denaturált szeszt, gyantát és vizet –, fölkészültem, megmondom az
igazat: órákon belül örökre eltávozom. Addig is iszogattunk; egymás vállán
pihentetve a lábunkat, ketten is elfértünk a kádban.

Az éjszaka kellős közepén, Aranka Westinnél találtak rám a szürke gúnárok.
Amúgy kissé szeszesen, pecsétektől jeges alsóban támolyogtam ki a
terepjáróhoz. A szederszínű sötétségben, mint valami távoli, sóval lepett
halom, Coca Mavrodin-Mahmudia arca világított. Mivel a közelben a Sinistra
vize harsogott, kiabálva tudatta: meggondolta magát, egy darabig, ameddig majd
ő jónak látja, mégiscsak Dobrinban maradhatok.

– A szürke gúnárok majd kitalálnak magának valami szép új nevet. Vagy egye
fene, maradjon a régi, az sem az igazi.

Talányos asszony, szeszélyes katona volt Coca Mavrodin-Mahmudia; úgy tűnt,
csak játszadozik velem, közben meg akart tartani. Évekkel később, Dobrinban
jártamkor hallottam, maga is milyen talányos módon végezte. Ültében elaludt az
erdőn, ott lepte meg az ónos eső, s ő mozdulatlanul, mint egy alvó lepke, a
rárakódó jégcseppek üvegébe fagyott. A jégtuskót később a szél fölborította,
széttört darabokra, és egyszerűen elolvadt. Helyén csak egy ázott, bogárszagú,
ezredesi csillagokkal teletűzdelt rongykupac maradt.
43
\section{(Mustafa Mukkerman kamionja)}

Abban az időben, amikor fogadott fiam nyomait keresve Dobrinban éltem, az
egész erdőkerületben egyetlen fényképész működött. Ez az egy is kizárólag a
hegyivadászoknak dolgozott. És még csak nem is terepszínű hegymászó katonákat
fényképezett, vagy a pirosított szájú irodistákat, hanem a természetvédelmi
terület burkus medvéit – tartottak belőlük az erdőn vagy
százharminc-száznegyvenet – a kormányzati nyilvántartások számára.

Valentin Tomoioaga fényképész – maga is ezredes – hétszámra az erdőt járta,
protekciósan valószínűleg különféle oltásokat is kapott, mégis, a tunguz nátha
végül őt is leterítette. Dobrin City közelében, az erdőszélen lett rosszul,
néhány csupasz, hántott törzsű fenyő alatt, el lehetett látni odáig a faluból
is. Bár korán fölfedezték, hiszen a szél egyfolytában lobogtatta rajta köpenye
szegélyét, nem került a gyengélkedőbe, hanem azon a helyen, ahol a láztól
olvadozva kinyúlt, szétterült, körbecövekelték, lécekkel beszögezték, s még a
csupasz fenyőtörzseket is gerendákkal ácsolták körbe, nehogy másokat is
megfertőzni valamerre elinduljon. Tartottak itt a tunguz náthától nagyon, s
úgy nézett ki, az lesz a legjobb, ha fertőző beteget, legyen az akár ezredes,
nem engednek többet vissza sem a laktanyába, sem pedig a faluba, sehova. A
sebtiben köréje vont palánk résein enni kukoricacsöveket dugdostak be neki,
inni ott volt neki a harmat.

Mivel a maródi Valentin Tomoioaga fényképész helyébe új ember nem került,
hamarosan elérkezett a nap, amikor 44helyettesíteni kellett. De nem a
medvéknél, vagy valamilyen rejtett objektum körül a rezervátumban, hanem az
ukrán határon, ahova azon a napon egy külföldi kamionost vártak. Az illetőnek
nem állhatott valami jól a szénája, ha fényképésszel készültek elébe. A
húsfuvarozó Mustafa Mukkerman lett volna az, akit ezüstfényű, színes alakokkal
telepingált kamionjával magam is gyakran láttam elrobogni a Dobrint megkerülő
észak-déli országúton.

Az, hogy pont nekem kellett beugranom a dögrováson lévő fényképész helyébe,
bár amúgy mindenben jártas, tapasztalt ember hírében álltam, csakis a
kiszámíthatatlan asszonyi szeszély műve lehetett. Örök rejtély marad, annyi
titoktartó, ravasz hegyivadász mellett hogyan esett rám Coca Mavrodin ezredes
választása.

Igaz, alighogy átvette Puiu Borcan halála után a dobrini erdőkerületet, egy s
más máris megváltozott. A változás szelei lobogtatták szerte a faluban azokat
a kicsi papírszalagokat, amolyan személyre szóló idézőket, majdhogynem
magánleveleket, amelyeken engem név szerint újra meg újra az erdőbiztosi
hivatalba rendelt. Ez történt most is: egy reggel ott zizegtek a zsákpapírból
tépett, szénnel telefirkált fecnik a gyümölcsbegyűjtő közelében,
villanypóznákon, kerítéseken, út fölé hajló faágakon, rajtuk a felszólítás:
„Siessen, Andrej, várja magát Coca kisasszony.”

Izolda Mavrodin-Mahmudia – a Coca beceneve volt – a megboldogult Puiu Borcan
ezredes karosszékében ült, előtte az íróasztalon két hatalmas fényképezőgép:
egy Konica és egy mordály Canon. Nem baj, ha nem értek a dologhoz nagyon,
mondta, ezek szinte maguktól mindent elvégeznek, csak egy megbízható, érzékeny
ember kell, aki kezében tartja őket, kicseréli időnként a filmet, és
nyomogatja a gombokat.

Az ukrán határ, ahova Mustafa Mukkermannak érkeznie kellett, a közelben, a Pop
Ivan vízválasztóján vonult. Éjje 45lente néha Dobrinból is látszottak a
fellőtt világítórakéták, gyakran felhőkön villant az őrtornyok pásztázó fénye,
de nappal ugyanolyan közöny áradt lejtőin a völgy fele, mint bármelyik
környező hegyről; évtizedek óta nem történt arrafele semmi.

Coca Mavrodin ezredes már a huzatos hágóra öltözötten, csuklyás szürke
hegyivadászköpenyben gunnyasztott a parancsnoki székben. Fülét a légvonat
ellen sárga vattával tömte tele, körülötte az a savanykás-kesernyés bogárszag
terjengett. Úgy hírlett, a miazmás deltavidékről, az óriás harcsák, gödények
vészterhes világából került ide a rideg északra.

– Amikor megengedhetem magamnak – mondta –, szívesen dolgozom civilekkel.
Ezért magára gondoltam, Andrej. Egyébként lesz velünk még két illető, a
fiatalabb korosztályból.

A vöröskereszt jelével ellátott katonai kétéltű járművön utaztunk
patakmedreken, süppedős lápokon, vizenyős réteken át, amíg el nem értük a Pop
Ivan tövét, ahonnan szerpentinek kanyarogtak a szoros felé. A vezetőülésen
maga Coca Mavrodin ült, mellette, fényképezőgépekkel a nyakamban jómagam, a
hátsó üléseken két télikabátos, sálas, kalapos, öltönyt, félcipőt viselő,
majdnem egyforma fiatalember, két teljesen egyforma dobermankutyával. A két
komor fiatal Coca Mavrodin szürke gúnáraihoz tartozott.

Útközben, bizalmas beszélgetésükből kiderült, a nemzetközi fuvarozó Mustafa
Mukkerman a Beszkidek felől érkezik, kamionja tele fagyasztott birkával. A
rakománnyal a Balkán legdélibb csücskei felé igyekszik, hetente egyszer
fordul: mindig, de mindig csütörtöki napokon, a déli órákban lép be az
országba. Az illető egyébként a legkevésbé sem hétköznapi lény, szabályos
óriás, úgy hírlik, súlya meghaladja a hatszáz kilót. A két szürke gúnár most
azon 46tanakodott, ha megérkezik, és majd sor kerül a testi motozásra, hogyan
osztoznak meg oldalain, hatalmas tagjain. Azért keltek útra, velem együtt,
hogy találjanak nála valamit.

Amint a kétéltűvel a szerpentineken a határ felé kanyarogtunk, a fordulókban
párszor elővillantak a Pop Ivan menyétszínű szirtjei, a főgerincről az erdő
mélyére nyúló fakóvörös sziklabordák, de a szoros felé közeledvén
elhomályosultak, kezdett elromlani az idő. A Pop Ivan magaslataira aznap
érkezett meg nagy dérrel-dúrral a tél.

A határállomás ott fönn a vízválasztón mindössze egy kicsi őrszobából és egy
tábori sátorból állt, előttük kék-sárga vassorompó zárta el az utat. Az
átkelőhely a hegységen átvezető régi földút legmagasabb pontjára települt,
ahol a kétoldali lejtők majdnem szorossá szűkültek. Komisz, huzatos hely volt,
fuvolázó szirtekkel, vén szakállzuzmótól lobogó szürke fenyőkkel. Túl a völgy
vágásán, a távoli áttetsző égbolton már észak nyugtalanító színei villóztak.

De mondom, mire elértük a tetőt, elromlott az idő. Az ormok közé szürke
foszlányok ereszkedtek, előbb tüskés eső kopogott a kétéltű bádogján, a
plexiüvegen, aztán hirtelen minden elcsendesült: tollú-nagy pelyhekben hullani
kezdett a hó. A szorosra a tél homálya ereszkedett, mélyén, a határőrök
sapkáin vörös csillagok izzottak.

Váratlanul az ég is megdördült, és bár a függönyökben érkező sűrű hóesést a
villámok fénye át- meg áthasította, egy határőr biztonságból piros fénnyel égő
viharlámpást akasztott a sorompóra, nehogy Mustafa Mukkerman, ha éppen a
fergeteg kellős közepén érkezik, váratlanul nekihajtson. Tudták, nem késik,
hiszen évek óta mindig csütörtökön, délidőben lépett be kamionjával: pontos
embernek ismerték. Úgy hírlett, apai, azaz Mukkerman ágon félig-meddig német.

A dobermanok a kétéltű alá heveredtek, míg Coca 47Mavrodin lassan előresétált
a sorompóig. Egész hóesés alatt a kék-sárga vasrúdra könyökölve várakozott,
nehogy véletlenségből is elmulassza azt a pillanatot, amikor a túloldali
szerpentineken átderengenek Mustafa Mukkerman fényszórói a havazás kárpitján.
Csak az arca elől időnként ellobbanó pára jelezte, posztóba csavarodva egy
lény gunnyaszt szemben a tüskés, ellenséges széllel. Szép lassan rá is
ugyanannyi hó rakódott, mint közelében a homokos ládára, a sorompó bakjára.
Prémsapkája fölött kicsi örvény kavargott, a végén szinte észrevétlenül a
vállára szállt egy madár.

Mustafa Mukkerman érkezését legelsőként a kutyák neszelték meg. A hóval rakott
szélrohamok még foszlányokat sem ragadtak el a rakomány alatt fújtató motor
távoli morgásából, amikor a dobermanok ásítozni kezdtek; mindenki tudja, ez
náluk a figyelem jele. A kamion, tele fagyasztott birkával, rohanó ködöktől
elfátyolozottan már a közeli kaptatón járt. A két kutya füle kihegyesedett,
farkuk csonkja megrebbent, előbújtak a fáradt olajtól csöpögő kétéltű alól.
Coca Mavrodin ismerte kutyáit, olvasott a nyakukon végighullámzó szőrből, és
nyomban kiegyenesedett. Hátán megrepedezett a rárakódott hó, puha tömbökben
omlott le róla.

Vállán megbillent, fölfordult a madár is. Mereven, jeges szárnyakkal hullott a
hóba. Úgy látszik, az előbb már csak meghalni szállt volt oda. Azt mondják, a
tunguz náthába a madár, aki északról hozza, maga is belehal.

Mustafa Mukkerman érkezésekor csend lett, a szél elült, a hópelyhek megálltak
a levegőben. Csak a sűrű szürkeség maradt, amit most villámok helyett a kamion
fényszórói pásztáztak; a katonák megemelték előtte a sorompót, hogy
áthaladhasson. Az ezüstösre festett kocsi fala tele volt pingálva mindenféle
badarsággal, ami csak egy ilyen országhatárok között hontalanul kószálgató
fuvarosnak juthat eszébe: bíbor égbolt alatt kék pálmák, zöld majmok éktelen
48kedtek rajta, egyik falát egyetlen, mélyen lecsüngő magányos női mell
díszítette.

A kétéltű belsejéből előbújt a két szürke gúnár, a kamion rendszámáról
lerugdosták a havat, hogy megbizonyosodjanak, valóban ő érkezett-e meg, akit
vártak. Körbejárták a meleg, magában még mindig kimerülten szuszogó járművet,
néha ujjal érintették a raktér ezüstös falát, s a színes pingálmányok láttán
szemükben a rosszallás árnyai motoztak.

Közben Mustafa Mukkerman is letekerte maga mellett az ablakot. Elképesztően
hatalmas, zsákszerű, kerek és csupasz karját kinyújtotta, kezét ökölbe
szorította, s az üdvözlet jeléül könyökből és csuklóból is néhányszor
megmozgatta. A szürke gúnárok összenéztek: jól látnak-e. Előjelnek nem volt ez
jó, mindenesetre.

Coca Mavrodin oldalba lökött, rajta, ettől kezdve nyomogathatom a gombokat. Ha
majd lámpa villózik az ujjam alatt, ki kell cserélni a filmet.
Belekukkantottam a keresőbe, s a telepingált kamion, a sofőr, a két szürke
gúnár a két dobermannal mindjárt életre kelt kicsiben, a matt üvegen.

Mustafa Mukkerman közben működésbe hozott egy gépezetet, a vezetőfülke fala
megnyílt, a sofőrt, ezt a hatalmas bőrzsákot az üléssel együtt a szerkezet
emelte ki, és a földre helyezte, ahol talpra állt. Vörös overallt viselt,
alatta hatalmas, gömbölyű húsok, hájas bőrlebernyegek remegtek. Remegett
közelében a levegő is, és a hó szemlátomást olvadásnak indult körülötte.
Amikor észrevette, hogy az őrszobából kilép két soványka vámtiszt – régi
ismerősei lehettek – vidáman integetett nekik, és a kamion falán biztos erre a
célra kiképezett fogantyúkba kapaszkodva kulcsait csörgetve maga is elindult,
kinyitni nekik a raktér hátsó ajtaját, hogy zseblámpáik fényét majd
végigsétáltassák a ködlő, zúzmarás húsok között. Már éppen letépni készült a
49zárról a plombákat, amikor Coca Mavrodin ezredes közbeszólt, és az egészet
leállította: ilyesmivel most fölöslegesen ne töltsék az időt.

A szürke gúnárok erre Mustafa Mukkermanhoz siettek, kétfelől közrefogták, s
fölkérték, ott, azon a helyen, ahol áll, vetkőzzék le. A parancsnoknő kérése
ez, de ő hölgy létére vonakodik kimondani. Netán még félreértené.

– Az első ilyenszerű fogásom – jegyezte meg mellettem halkan Coca Mavrodin
ezredes –, tudja, eddig a langyos délen állomásoztam, Pelikán-farmon
dolgoztam.

– A jóisten meg fogja segíteni.

A havazás valóban elállóban volt, a szürke gúnárok azzal vigasztalták Mustafa
Mukkermant, benn az őrházban, ahová egyébként sem férne be, amúgy sem lenne
melegebb. Akkor meg rajta, vetkőzzék le teljesen mielőbb.

– Az csak természetes – bólintott a sofőr. – A legnagyobb örömmel.

– Hol tanulta meg ilyen jól a nyelvünket? – szólt oda Coca Mavrodin.

– Hogy hol? Ó, hát csak így átutazóban. Bejön az a levegővel az ablakon.

– Tudja, nagyon sajnálatosnak tartom, hogy ilyesmire került sor. És pont
magával szemben, akit annyira tisztelünk.

– Engem kellemesen érint a dolog – mosolygott Mustafa Mukkerman, a sofőr. –
Amúgy is meg szerettem volna mutatni maguknak a pucámat.

Coca Mavrodin előbb félrenézett, majd hirtelen rám pislantott, hogy leolvassa
az arcomról, jól hallott-e. Zsebéből kihegyezett tintaceruzát vett elő, mintha
a tenyerére vagy a levegőre akarná írni, ami elhangzott. A két szürke gúnár is
nyújtogatta nyakát az elszálló szavak után. Mustafa Mukkerman pedig, mintha
csak erre várt volna, melle, hasa fölött végighúzta a cipzárt, hogy
előbújhasson öltözékéből. 50Különleges, az ő testére szabott kezeslábast
viselt: alighogy a megnyitott, itt-ott megoldott ruhát megrázta magán, az
nyomban lehullott róla. Amint az előbb kívánták tőle, máris ott állt teljesen
csupaszon, remegő hájakkal az ezüstös hópelyhek között.

– Ne higgye, hogy szívvel-lélekkel teszem – fordult felém Coca Mavrodin. – A
mesterségem nem ez, a meztelen embert pedig különösen utálom. De kaptam a
lengyel elvtársaktól egy fülest. Hogy ez a személy a hája közé dugva készül
hazánk területén átcsempészni valamit. Hogy mit, azt sajnos nem mondták meg.

Mint valami lankadt szárnyak, Mustafa Mukkerman válláról, lapockájáról,
derekáról remegő húsok, lebernyegek csüngtek. Már ha azokat bárki még vállnak,
deréknak merné nevezni. A két dobermant noszogatni kellett, hogy a sofőrt
végigszaglásszák, a szürke gúnárok az örvüknél fogva egyenként vonszolták őket
a közelébe, mert azok valósággal megkutyálták magukat. A két dobermant Mustafa
Mukkerman egyáltalán nem érdekelte.

– Legjobb lesz, ha önként előadja – szólalt meg kis idő elteltével Coca
Mavrodin ezredes. – Azzal, mondhatni, máris túlestünk a dolog nehezén.

– Én nem sietek.

– Pedig nem hinném, hogy arra vágyik, hogy az embereim végigmatassanak magán.

– Mért ne. Amikor a tökömet vakargatják nekem, én azt nagyon élvezem.

Coca Mavrodin ceruzáját remegtette ujjai között, a két szürke gúnár pedig
nekilátott a testi motozásnak. Csak úgy kutakodtak a hús fodrai, redői között,
ujjukat reménykedőn, lassan és érzéssel húzták végig a hurkák árkain. Mustafa
Mukkerman feneke két pofáját is félrelebbentették, nagy komoran
beletekintettek, megringatták a herezacskóját, benne az álmos galuskákat.
Dolguk végeztével alig mertek 51egymásra pillantani: a török fuvarozó sofőr
legmeghittebb réseiben, titkos bőrerszényeiben sem találtak semmit.

Az meg csak állt szétvetett lábbal, terpeszben, még kissé várakozóan, mint aki
sajnálja, hogy az egész ilyen hamar véget ért. Leeresztett zsíros szemhéja
alól lesett ki körbe, szórakozottan emelgette talpát a friss olvadékban.

– Maga is úgy találja, hogy vigyorog? – villant rám Coca Mavrodin. – De hát
vajon mi az ördögöt?

– Egyrészt – válaszolt helyettem Mustafa Mukkerman, aki a kérdést meghallotta
– szeretek jól kinézni fényképeken. Azonkívül én ezt az egészet megálmodtam.
Azért most sajnos nincs is nálam az, amit maguk keresnek.

Coca Mavrodin a szürke gúnárokra meredt, futólag talán rám is vetett egy röpke
pillantást, aztán a ceruzát, amivel pedig bizonyára készült valamire, egy
roppantással kettébe törte, darabjait a hóba hullatta. Ezzel, mint aki a maga
részéről kiküldetését befejezte, a várakozó kétéltű felé indult, nyomában
szigorú tekintettel a szürke gúnárok. Már én is indulófélben voltam, nyakamban
a súlyos fényképezőgépekkel.

Ekkor találkozott a tekintetem a Mustafa Mukkermanéval. Az övé jósággal,
szeretettel, bársony emberi meleggel volt tele. Karját felém nyújtotta,
hatalmas mutatóujját hívogatóan begörbítette. A kesztyűtartóból egy csomag
Kentet, egy kicsi celofánzacskó Haribo gyümölcszselét vett elő, végül
valahonnan még egy Kinder-tojást is előhalászott, tálcányi tenyerén mindezt
felém nyújtotta. Azon a zimankós délelőttön a hegyszorosban, a tél érkezése
ünnepén egy anyaszült meztelen török, engem, az elcsapott
erdeigyümölcs-szakértőt megajándékozott.

– Figyelj – szólított meg halkan. – Egyszer bizonyára ezt az egészet megunod.
Csak szólj, szívesen magammal viszlek a Balkánra. Szalonikibe, a Dardanellákra
vagy éppen 52Rodostóba. Beduglak hátul a húsok közé, nem lesz meleged, de te
majd jól felöltözöl. Ott senki sem talál meg.

– Kérlek, hallgass.

– Szerezz magadnak idejében egy jó vastag és meleg subát. Én minden
csütörtökön erre haladok el, olajat veszek a kútnál, tudod, lenn, az
észak-déli országúton. De leinthetsz menet közben is. Csak lehetőleg ne essen
az eső azon a csütörtöki napon: ázottan, vizes ruhával nem ülhetsz be a jeges
húsok közé, a fagyba. Na jó, most menj, Allah veled.

– Fogalmam sincs, miről beszélsz. Nem hallottam abszolúte semmit. De azt meg
kell adni, tényleg ismered a nyelvet.

– Ugyan. Csak betanult szövegeket mondok.

A kétéltű már begyújtott motorral, remegő bádoggal várakozott. Amint
elhelyezkedtem az ülésen, Coca Mavrodin indított is, s a friss hóban
meg-megcsusszanva, lassan ereszkedtünk a szerpentineken a Sinistra völgye
felé. Máris a Haribo zselét szopogattam, s a két szürke gúnár feje között a
hátulsó ablakon nézegettem az elmaradozó tájat: Mustafa Mukkerman még mindig
meztelenül állt a hóban, bámult utánunk, amíg az első kanyar el nem takarta.

– Képzelem, meghívta magát a Balkánra – vetette oda nekem Coca Mavrodin. – A
görög tengerpartra, az Olimpiára.

– Megpendítette.

– Azért ilyen tervekkel most ne izgassa magát.

Az aljba érkezve a kétéltű újra letért az útról, és a patakmeder kanyarjait is
átszelve, bukdácsolva haladt át az ázott réteken. A két doberman fölágaskodva
bámult ki az ablakon, a szürke gúnárok szeme éberen csillogott, pedig odalenn
különösebb megfigyelnivaló már aligha akadt. Coca Mavrodin homloka megizzadt,
a falu felé közeledve arra kért, igazítsam meg fején a sapkát.
53

– Legközelebb majd meglepem – mondta. – De nagyon. Kieresztem a kerekekből a
levegőt, például, vagy valami ilyesmi. Esetleg kifordítom neki a gumibelsőket.
Menjen el tőlünk örökre a kedve.

– Még hogy megálmodta… – próbálkozott az egyik szürke gúnár.

– A lengyel elvtársak, azt hiszem, szántszándékkal átvertek minket – jegyezte
meg a másik.

– Azt ajánlom, hallgassanak.

Hamarosan elviharzott az országúton maga Mustafa Mukkerman is, a pálmafákkal,
majmokkal és azzal az egyetlen csüggedt női mellel telepingált kamionjával.
Persze látta a kétéltűt bukdácsolni a havas esőtől üveges réten, hosszasan
tülkölt, integetett. Mögötte szikrázva örvénylett a fölkavart hó, hozta
magával a telet, miközben ő már a napfényes Balkán felé tartott.

Az erdeigyümölcs-begyűjtő központ, ahol akkoriban laktam, a dobrini állomás
felől elérhető volt egy keskeny szekérúton, hisz a falun kívül, egymaga állt a
réten. A letérőnél Coca Mavrodin lefékezett, és miután a még mozgó járműből
kiugrottam, leállította a motort.

– Dehogyis hasítom ki neki a gumibelsőket – kiáltott utánam. – Én tudtam a
legjobban, nincs nála semmi. Nehogy elhiggyen rólam ilyesmiket.

– Gondoltam is, hogy csak viccelni tetszik.

– Ha most maga is belegondol egy kicsit, mi történt, rájön, ez az egész meg
volt beszélve a lengyel kollégákkal. Csak gyakorlatot tartottunk.

– Szó, ami szó, magam is így sejtettem.

– Dehogyis sejtette. Ezt a fontos dolgot tőlem tudta meg most, hogy elárultam
magának.

A tél azokban az órákban ereszkedett lefele a Pop Ivan lejtőin a Sinistra
völgyébe. Csorgott az egykori malom, a gyümölcsbegyűjtő eresze, közben már
jégcsapok is növö 54gettek rajta. Szerettem volna fölbontani az ajándékba
kapott Kinder-tojást, hogy ugyan bizony miféle elmés kis apróság rejtőzik
benne, de ősz vége lévén, korán sötétedett, ezeket a kellemes perceket
másnapra halasztottam. A folyosó homályában már csak találomra merítettem
csuprommal az erjedt gyümölcslevekből, amellyel italomat, a vízzel kevert
denaturált szeszt szokás szerint megízesítettem. Kuckómban, egy kamra sarkában
kuksoltam a sötétben, nálam esténként a szesz belülről, az ereimben
világított.

Egy idő után megéheztem, gombát, rideg főtt krumplit áztattam a vizezett
denaturált szeszben, azt szopogattam, s örömmel hallgattam, odakünn a szél
orgonálni kezd a jégcsapokon. Elalvás előtt még szokásom szerint az ablakba
térdeltem, és kivizeltem az udvarra. Ha esett, ha nem, mindig ezt tettem.

De most nem a legjobbkor. Hamarosan zseblámpa fénykévéje ugrált a nyirkos
falakon, keresgélt a hordók között a korhadt padlón, mígnem megállapodott
rajtam, aki kamrám sarkában a szalmazsákomon pihentem. Az egyik szürke gúnár
volt az lucskosan, csúnyán összevissza ázottan.

– Arra kérem, máskor ilyet ne tegyen – mondta halkan, szigorúan. – Ha pisilnie
kell, bármikor szívesen átkísérjük a sötét udvaron. Mostantól valamelyikünket
mindig a közelben találja.

Igen, gondolhattam volna, ettől a naptól kezdve Coca Mavrodin bizalmasának
számítok, beleláttam egyik titkába, ezentúl rám is vigyáznak majd a szürke
gúnárok. Egyikük máris ott állt előttem, húgytól csatakosan.

Hajnaltájt, amikor az illendőségi elvárásoknak megfelelően az udvar végében
lévő árnyékszék felé baktattam, megszólítottam:

– Álmos lehetsz. Ha feljössz, kerítek egy papírzsákot, amin elheveredhetsz. Új
nap kezdődik, pihend ki magad.
55

– Nem, nem – hárított el –, maga idegen, nem tudhatom, miféle fekhellyel akar
megkínálni.

– Akkor tárgytalan a dolog.

Közeledett a reggel, a hegyoldalakon egyszerre csak, mint ahogy a fény elönti
a lejtőket, végighullámzott a kutyaugatás, aztán túl a patakon, az út mentén,
Géza Kökény mellszobra körül sárgásan izzani kezdett a köd.

Aztán megint a kutyák. Ha ugatásuk rövid időre kimaradt, a dermedt csöndben
meghallottam az erdő széléről Valentin Tomoioaga fényképész vonítását. A
fényképészezredesét, akit egy napon át én helyettesítettem, s akinek Mustafa
Mukkerman barátságát köszönhetem.
56
\section{(Elvira Spiridon férje)}

Törpével találkozni reggel, azt tartják, jó jel. Életem egyik szerencsés
napján, amikor hozzám költözött a bársonyfenekű Elvira Spiridon, kora reggel
Gábriel Dunkával találkoztam, a törpével. Dobrin Cityben, ahol akkoriban
mindketten életünket tengettük, a kevesek közé tartozott, akik házukban ollót
tarthattak; ha tehettem, eljártam hozzá a nyesegetnivalóimmal. Az egyetlen
igazi borbélyt, Aranka Westin élettársát elkergették, ezért ha benőtte a haj a
nyakamat, a törpénél nyiratkoztam.

Az emlékezetes napon, ki tudja, mi okból, azt hittem, csütörtök van, Mustafa
Mukkerman török kamionost lestem, aki mindig a hét egy bizonyos napján haladt
el a falut megkerülő észak-déli országúton; Gábriel Dunkával csak véletlenül
találkoztam. Bár ősz vége felé járt, és a patak öbleit már vékony jéghártya
borította, őt a parton találtam, sebes, szederjes bokáját áztatta a jeges
vízben. Homokkal dolgozott, műhelyében naphosszat egy nedves homokkal teli
ládában taposott mezítláb, kikezdte bokáját az egyhangú munka. Az épülő
sinistrai börtön szerződtette, a homok alatt üveglapokat homályosított az
ablakokra. Lévén a körzet egyetlen törpéje, egyedül ő volt alkalmas a kényes
feladatra: könnyű, kicsi teste alatt nem törött el egyre-másra az üveg. Dagadt
bokáját lohasztani télvíz idején is lejárt a lékekre.

Löcsbölés közben találtam, és nem állt szándékomban hosszasan eldiskurálni
vele, egy darabig mégis nála ragad 57tam. Érdeklődtem, látta-e a töltőállomás
körül várakozni Mustafa Mukkermant a húsos kamionjával, vagy netán már el is
haladt. De egyikünk sem volt biztos benne, csütörtök van-e valójában. És
bennem, úgy látszik, valami előérzet motoszkált, mert úgy döntöttem, egyúttal
kicsit megstuccoltatom a hajam. Holott Puiu Borcan ezredes temetésére
készülődve nemrég annak rendje-módja szerint megnyiratkoztam.

Miközben ollójával a fülem mögött csattogott, mint egy igazi borbély,
történeteivel traktált. Nemsokára gazdag lesz, említette, emberek jártak nála
a körzeti természetrajzi gyűjteménytől, a csontváza iránt érdeklődtek:
megvásárolnák jó pénzért, hogy majdan kiállítsák. Első mérgében elküldte őket
ugyan, de ha újra megjelennek, márpedig az ilyen vigécek aligha adják föl
egyhamar, nem fog nemet mondani. Engem nem nagyon érdekeltek az ügyletei, meg
is kértem, az olló villanásaira figyeljen, mintha csak tudnám, Elvira Spiridon
számára szépítem magam.

Nyiratkozás közben, Gábriel Dunka üvegesműhelyében találtak rám a
hegyivadászok, és mindjárt magukkal vittek a laktanyára. Az erdőbiztosi irodán
Coca Mavrodin várt rám, arra kért, még aznap költözzem ki a faluból a Baba
Rotunda-hágóra. Most is tompán köpenyébe burkolózva gunnyasztott a
természetvédelmi terület domborított térképe alatt, mint hálója zugában a pók,
talán órák óta nem moccant, szemén, ajkán, nyelvén nem villant semmi fény.

– Lakott a hágón egy útkaparó – kezdte –, valami Zoltán Marmorstein nevű. Fene
tudja, mi ütött belé, elment, itt hagyott csapot-papot. Üresen áll a háza,
szeretném, ha maga költöznék belé.

– Nem szívesen túrnám ki a helyéből.

– Nos, az illető egyén, úgy néz ki, nem tér vissza. Ha hinni lehet az
embereknek, nagy jelenetet rendezett az este. Ki méltóztatott ontani a belét.
58

– Így már aligha mondhatok nemet.

– Az utászház szolgálati lakás, ottléte majd bizonyos teendőkkel jár. Zoltán
Marmorstein minálunk úgy mellékesen a halottkém egyik helyettese volt.

– Jólesik, hogy rám gondolt. Bár úgy érzem, e téren vár rám még egy s más
tanulnivaló.

– Hát iparkodjék.

Coca Mavrodin ezredes, bár maga is velem járt ott először, most megmutatta a
falitérképen a Baba Rotunda-hágóra szerpentinekkel fölkanyargó utat, a
tisztásokon szerte a szénagyűjtő tanyákat, legvégül a tetőn Zoltán Marmorstein
útkaparó házát. A hágót hálóként ellepő ösvények mentén még a karámok, pajták
és kutyaólak is be voltak rajzolva. Ismertem jól a helyet.

– És mi lenne a teendő odafönn?

– Semmi. Csak ott fog lakni, méghozzá nem is egyedül.

Az ezredesnő fiókjából egy köteg fényképet vett elő, szétteregette őket az
íróasztalon. A képek környékbeli asszonyokat ábrázoltak, legtöbbüket ismertem
a gyümölcsbegyűjtőből, ahol áfonyával, szederrel, vargányával teli
puttonyokkal a hátukon rendre megfordultak. Ismertem jól a gyűjtögető
asszonyokat.

– Válasszon – mutatott a kiteregetett képekre Coca Mavrodin, hol egyiket, hol
másikat tologatva a tekintetem előtt. – Természetesen egyelőre csak egyet.

Közöttük volt a szépséges berkenyemadár, Elvira Spiridon. A képen még
csillogott az orra hegye, domború homloka, két nagy rézkarika fülbevalója. Ő
volt az, Elvira Spiridon, akinek a talpából egyszer fogammal szedtem ki egy
tövist.

– Válasszon nyugodtan: szívesen magához költözik bármelyik. – És az ezredesnő
egy pillanatra eltakarta Elvira Spiridon fényképét. – Akár még ez is.

– Coca kisasszony túl jó hozzám – ingattam zavartan a 59fejem. – Nem biztos,
hogy megérdemlem. Meg aztán, ugye, egyebek is.

– Nyugodjék meg. Már beszéltem a férjével. Elengedi.

Bár mondom, amúgy ismertem a Baba Rotunda-hágó környékét, meg az erdőbiztosi
iroda falitérképe is eligazított, egy katona rövid szemrevételezésre kivitt
terepjáróval a tetőre.

A szeles vízválasztón át vezetett az út a bukovinai dombvidék felé. A régi
földutat, amelyen naponta egyszer a sinistrai buszjárat is átdöcögött, inkább
csak medvészek és a dobrini hegyivadászok használták. A csendes őszi napok
után most fölkerekedett a szél, felhőket görgetett át a tisztásokon. A tetőn,
vízcseppektől elborítva állt az útkaparó háza, egy kiugró üvegezett tornáccal,
ahova, ha néha föllebbent a felhő, bevillant a szerpentin egy-egy kátyútól
csillogó kanyarja. A hasadékokkal teli falak között, ruhaszárító dróton Zoltán
Marmorstein ottfelejtett kapcái lebegtek.

Akkoriban, azonkívül, ami a zsebemben elfért, egy pléhtányérom volt, két
bádogcsuprom, egy lópokróc, pár kapca, rongy meg zsineg és egy üveg denaturált
szesz. Visszatérve a régi vízimalomba mindezt tarisznyába gyömöszöltem, s
vállamra lendítve búcsút intettem a gyümölcstárolónak, a sok bódító illatú
hordónak, s elindultam új munkahelyemre. A halottaskamra a laktanyaudvar egyik
nyirkos szögletében állt.

Azért izgatott, mi lehet Mustafa Mukkermannal, útközben újra betértem Gábriel
Dunkához. Jó lett volna tisztázni a kamionossal, mennyiért vinne a fagyott,
zúzmarás húsok között két személyt a Balkán legvégére. Gábriel Dunkától
megtudtam, aznap a törököt hiába várom, közben kiderült – Géza Kökény járt
nála –, nem csütörtök van, hanem legföljebb szerda.

Tehát valószínűleg egy szerdára esett a dobrini halottasházban az első
munkanapom. A helyettes halottkém teen 60dője abból áll, ül egy kamrában a
halottal, lesi, kémleli, megmoccan-e a műszak alatt. A nyirkos, szürke
kőasztalon a volt útkaparó, Zoltán Marmorstein feküdt, nadrágja telis-tele a
belével. Nem moccant, szikkadó kapcáit máris a magaménak éreztem.

Este fölváltott Titus Tomoioaga ezredes, s engem künn a levegőn oktalan öröm
töltött el, amint a szeszesüvegbe gyakran belekortyolva a Baba Rotunda-hágó
felé bandukoltam. Arcomon hópelyhek olvadoztak, hullani kezdett, a felhők
mögül áttetszett a száguldó hold.

Mire a tetőre értem, az utászházat teljesen körbefújta a hó. Már készültem
meglengetni a zseblámpa fénykévéjét a tornácra vezető néhány lépcsőfok fölött,
amikor észrevettem, a tornác ablaka belülről vörös párától gyöngyözik, lobogó
tűz fényétől olykor villan, földereng. Coca Mavrodin tehát nem áltatott. Már
nem voltam egyedül.

Az utászházban a kályhaajtó három piros szeme világított, az imbolygó, repdeső
fények között fülbevaló rézkarikák csillogtak. Elvira Spiridon a priccs szélén
ült ölbe ejtett kezekkel. Előtte levetett bocskora.

– Mostantól az úrnál lakom.

– Isten hozta.

– Megmondták, az úr keveset beszél. Akkor majd én is inkább hallgatok.

– Remélem, magának sem lesz oka panaszra.

Zoltán Marmorstein elhagyott priccsén most két, kerekre tömött párna hevert,
két, frissen mosott rongyszőnyeg, amelyekből még áradt a hágóra aznap érkezett
északi szél illata. Az asztalon, régi fekete érclábasban egérszagú
krumplileves, felét még egy másik ember ehette meg. És a hegyivadász ezredesek
kedvenc itala: egy tele üveg szederpálinka. Dugójába tűzve tündöklő csillag,
egy aranyos-ezüstös bábakalács.

– Az uram küldi.
61

– Az ura kedves. Biztos majd őt is megismerem. És most arra kérném, ne bőgjön.

– Az uram Severin Spiridon, hiszen már ismeri félig-meddig.

– Hát így névről nem is tudom.

– Volt egy buta kis esete. Akkor maga segítette ki a bajból. Nem akart élni,
mire lelket lehelt belé az úr.

– Hm, nem tagadom, rémlik valami. Akkor, ha jól emlékszem, van egy szép tarka
kutyájuk is.

– Igen, igen. És a kutyánk sem felejtette el az urat.

Észrevettem, bokája körül megmozdul a kapca, elébe térdeltem, és saját kezűleg
tekertem le róla. Arról az erekkel hálózott, szénaillatú, langyos lábról,
amelyet, úgymond, szegről-végről, a bizonyos tövisügy óta már volt szerencsém
ismerni. Most újra a tenyeremben tartottam.

– Hát így – morogtam szórakozottan –, egy ezredes akárhogyan is, de betartja a
szavát. És én még azt hittem, csak ugrat Coca Mavrodin-Mahmudia. A fennvaló
ezerszer áldja meg.

– Igen, az ezredes kisasszony kívánsága úgy szól, mostantól az úrnál lakom. De
ha elenged, néha hazamegyek.

– Menjen, amikor csak kedve tartja. Elvégre van kihez. És most megint csak
arra kérem, ne bőgjön.

Kidugaszoltam az üveget, töltöttem Severin Spiridon ajándékából a két
bádogcsuporba. A priccs alatt egy lavórt találtam, vízzel tele a kályhára
készítettem, majd belekortyoltam a levesbe. A lavór eresztett, a forró
kályhalapon szertefutó golyóbisokat figyeltem, közben az italt is
megkóstoltam, és fél kézzel integettem Elvira Spiridonnak, rajta, tessék,
vetkőzzék.

Elég rég nem hallottam ezt a fajta ruhasuhogást, amint az egymáshoz simuló
csupasz karok és főként a bársonyos combok neszelnek, amint a víz végigpereg a
bordákon, ahogy a bőr szikkad. Kiválasztottam Elvira Spiridon comb 62ján egy
eret, amely hol szerteágazva, hol összetalálkozva futkosott fölfelé.
Mutatóujjaimmal elindultam rajta, látszólag tétován.

– Tudja meg – szóltam hozzá halkan, magam is csodálkozva hangomon –, én voltam
az, aki egyszer megtaláltam a tüskét a talpában. Ha emlékszik még a dologra:
tulajdon saját fogammal szedtem ki onnan.

– Nem felejtettem el az urat.

– Elmondhatom, magamban azóta is berkenyének, madárnak, berkenyemadárnak
hívom. Ahányszor eszembe jut, annyiféleképpen.

– Nem értem pontosan, de azt hiszem, szépeket mond most az úr nekem.

– És, hogy folytassam: hamarosan minden egyes porcikáját végigpuszikálom. Hogy
aztán ne érjék meglepetések.

– Ott puszikál végig az úr, ahol akar.

Jóval később, már túl az éjszaka derekán, amikor a tüzet élesztgetve
meztelenül guggoltam a kályha előtt, megint a magam dolgain elmélkedtem.
Eszembe jutott Mustafa Mukkerman és persze, Béla Bundasian, a fogadott fiam,
akit már négy vagy öt éve nem láttam, holott itt a közelben élt a
rezervátumban. Abban reménykedtem, a végén csak ráakadok, és talán együtt
távozunk a napfényes Balkánra. Bizony nem a legjobbkor költözött hozzám ez az
asszony. De most ott pihegett a közelemben. Térden állva a priccs felé
fordultam és a takaró alá nyúltam.

– Remélem, jó volt.

– Nem volt rossz, uram.

– Talán a jóisten keze lesz a dologban. Ha mégsem, egy szép napon elmegyek
innen. Megsúgom, van nekem egy másik életem.

– Gondoltam. Ismerte az úr Zoltán Marmorstein urat? Ő is elment.
63

– Nem, nem volt szerencsém hozzá.

– Lehet, visszatér, mert az ő kapcái itt ezek.

– Ha jön, itt lesz. Várjuk szeretettel.

Éjszaka elállt a szél, a havazás, és a holdfényes bércek bevilágítottak a
házba. A nagy csöndben, mintha Zoltán Marmorstein súlyos beleivel a
nadrágjában máris közelednék, ropogott a hó a ház körül. Elvira Spiridon
kikelt a takaró alól, megállt az ablak előtt. Elég sokáig, talán órákon át
állingált ott, s mint a közeli behavazott ormoknak, puha, lila és gömbölyű
volt a válla. Amikor virradatkor visszabújt mellém, combja, feneke, mint a
jég, mint az üveg.

Leheletemmel végigfújdogáltam a rideg tagokon, beleszimatoltam a hajlataiba.

– A szagáról még nem is beszéltem. Ezt például, itt a kis nyakán, nagyon
kedvelem. Nem akárkire vall.

– Mielőtt elindultam, az uram megmosdatott. Bekent itt-ott mogyoróolajjal.

– Szóval mogyoró! Ilyesmiről most hallok először. Mindenképpen szeretném
megismerni a férjét.

– De hiszen ismeri. Egyszer megmentette az életét.

Évek óta nem pihentem meztelenül rongyszőnyegen, meleget árasztó kályha
közelében, és úgy még soha, hogy közben könnyű mogyoróillat csiklandozzon. Azt
mondtam magamban: „Mit akarsz még? hiszen mindent elértél. Heversz elnyúlva,
Elvira Spiridon bársony fenekével az öledben. A tetőre érkeztél, fiam.”

– Még nem tudom az úr nevét – riasztott fel váratlanul Elvira Spiridon.

– Csakugyan. De megnyugtathatom, nemsokára, legkésőbb estére megmondom.

– Mert azért, ha úgy adódnék, lehet, néha-néha én is megszólítom.

– Igaza van, és kérem, legyen egy kis türelemmel, hamarosan sor kerül a
bemutatkozásra. Lehet, nem hisz 64majd nekem; nemrég elvesztek az irataim. A
nevemet illetően sürgősen beszélnem kell Coca Mavrodin ezredessel. Addig
sajnos nem nyilatkozhatom.

– Csak úgy gondoltam, ha majd nevén szólíthatom, könnyebben megszokom majd az
urat.

Ha néha felült mellettem a priccsen, tréfából a hóna alatt néztem ki az
ablakon, máskor közvetlenül a válla fölött. A völgyeket lila pára takarta,
csak a fenyők csúcsa állt ki belőle, néha varjak emelkedtek ki seregestől, és
húztak el a Pop Ivan meredélyei felé. A behavazott tetőket akkor öntötte el a
nap.

Amíg a ház szellőzött, egymás mellett álltunk szótlanul a nyitott ablak előtt.
Kézfejünk egymáséhoz ért, lassan-lassan egymásnak is feszült, végül a két kéz
megbékélve összekulcsolódott. Közöttük, a titok kagylójában bizonyára
ugyanannak az embernek a neve lapult. Azé, akinek a lábnyomai, mint a hűség
kötelékei tekerőztek körkörösen a falak mentén a hóban.

A szemközti tisztáson, a frissen hullott hó alól feketén csillogva bukkantak
elő a meleg trágyarakások, közöttük egy tarka kutya sétált, a fölöttük kavargó
gőzben melegedve csonttollú madarak lebegtek. A közeli ház zsindelyfedelén
hínáros indákkal lebegett a füst, éjszakai portyája után Severin Spiridon már
otthonában tett-vett.

Hétágra sütött a nap, hamarosan indulnom kellett új munkahelyemre. Látszott,
asszony van a házban, mert zubbonyomat, amelyet azon nyomban elneveztem a
helyettes halottkém gúnyájának, friss levegővel telve a kapura akasztva
találtam. Nincs az a megátalkodott szag, amit egy ilyen göncből a hágó szele
egy éjszaka alatt ki ne fújna.
65
\section{(Bebe Tescovina vére)}

A helyettes halottkémet, az egykori szeder- és áfonyaszakértőt, aki Andrej
álnéven élt a dobrini erdőkerületben, és a Baba Rotunda-hágón, az utászházban
lakott, egy kora reggel hegyivadász kereste. A katona megvárta, amíg Elvira
Spiridon távozik, és útban férje tanyája felé eltűnik a fenyők mögött, akkor
lépett elő az erdőből, és a tetőn magányosan álló ház felé indult. Andrej
Bodor, amint meghallotta a közeledő lépteket – keményen ropogtak az út menti
jeges marton, mint amikor valaki nagyon rossz hírrel közeledik –, az ajtó mögé
húzódott, és mint a kutya, a sarokba vizelt pár cseppet.

De a katona csak egy csomagot hozott, ami fölér egy meghitt üzenettel.
Térképtáskájából használt altiszti egyenruhát, gumicsizmát, szűk
csizmanadrágot vett elő azzal, hogy Andrej Bodor mindjárt öltse is magára,
majd induljanak el Dobrin City felé; ez pedig túl rosszat nem jelent. Ilyen
rangjelzés nélküli, levetett egyenruhadarabokat Sinistra körzetben a
hegyivadászok bizalmi emberei viseltek.

– Tudja, azért jöttem gyalog, hogy legyen időnk útközben elbeszélgetni.

– Velem nem lehet. Legfönnebb az eperről, a szederről, esetleg a
fülesbagolyról.

– Akkor közeledünk a tárgyhoz. Ha megengedi, nekifogok.

– Attól félek, nincs nekünk közös témánk.

– Ez az, hogy van, éspedig a Coca kisasszony. Ő küldött engem. Nos, kezdetben
félreismerte magát, de mostanra 66megváltozott a véleménye: megsúgom, most már
sokra tartja. Ráadásul kényes feladattal kezdené – természetesen, csak ha
vállalja –, el szeretné küldeni a rezervátumba.

– Nincs nekem bejárásom oda, Borcan ezredes nem adta ki a passzusomat.

– Most már van passzus. Coca kisasszony arra kéri, töltsön egy éjszakát a
kantinban. Lakik ott egy kislány, Bebe Tescovina a neve. Azt beszélik, mint a
hiúznak, éjszaka világít a szeme. Nem ártana a dolog végére járni, igaz-e.

A Dobrin lejtőin valamikor ércet bányásztak, a rakodórámpákig, a meddőhányóig
keskenyvágányú vasúti pálya vezetett. Később, amikor a kitermelés megszűnt, és
az elhagyott vágásokba medvéket költöztettek, a kisvasút megint csak jól
fogott, csilléken szállították a dúvadaknak az eleséget, gyümölcsöt,
hulladékot, lódögöt; még élő szamarakat is. A sínek végében még állt a régi
bányászkantin, most medvészek, erdőkerülők jártak oda inni, esténként
malmoztak, kockáztak, dominóztak, gombát és madártojást sütöttek a forró
kemencelapokon.

Bebe Tescovina a kantinos Nikifor Tescovinának volt az egyik gyereke,
hirtelenvörös hajáról mindenki ismerte a környéken; amíg az első hó el nem
borította a síneket, kézihajtányon járt le az iskolába, Dobrin Citybe. Mint az
égő berkenyebokor, haja a szürke kerítések alatt messzire világított. Most
kiderült, világít a szeme is.

– Nem értek én az ilyesmihez.

– Dehogynem. És a laktanya portáján várja magát egy kis csomag, azt meg el
kellene vinni Géza Hutirának.

– Géza Hutira, Géza Hutira, ki is az? Nem volt szerencsém az illetőhöz.

– Ő a rezervátum meteorológusa. Könnyű lesz felismerni, földig ér a haja.
Huszonhárom éve nem nyiratkozott.

Andrej Bodor öt év óta várt erre a napra. Azalatt szám 67talanszor elképzelte
azt a pillanatot, amikor ő és fogadott fia összetalálkoznak. De arca a hír
hallatán meg sem rebbent.

– Őszintén szólva, nem is tudom. Kérdés, odatalálok-e.

– Meggyőződésem.

– Tudja, télvíz idején nem szívesen barangolok az erdőn. Kitalálhatja, mire
gondolok: még fölszedek valamit. Az idén még nem kaptuk meg az oltásokat.

– Ez igaz, Coca kisasszony volt az, aki leállította. Azt mondta, nem szereti,
ha összeszurkálják az embereit. Majd kitalál helyette valami egyebet.

A csomag, ami Andrejra várt a portán, mindössze egy alumíniumrúdból állt. Nem
is rúd volt, inkább egymásba süllyeszthető csövek szerkezete; az elemeket
kihúzva bizonyára jócskán meg lehetett hosszabbítani. Azonkívül el volt látva
különböző méretű lyukakkal, furatokkal, amelyekből narancsvörös és sárga
gyapjúfonalak csüngtek. Ki nem lehetett volna találni, mire való. Andrej
vállára kapta és elindult.

Géza Kökény, a hős medvész mellszobra koromfeketén állt az út túloldalán,
varjaktól rakottan, de azok közeledtére fölrebbentek, s akkor ott maradt
tiszta fehéren a rárakódott piszoktól. Jó jelnek számított ez, meg nem is. Ott
állt közelében a vöröskeresztes terepjáró, letekert ablaka mögött Coca
Mavrodin lepkeporos arca. Nem is gondolná róla az ember, hogy nő, ha nyakában
nem visel tüzes fényű medaliont: rézkeretbe foglalt ötágú vörös csillagot.

Ősz végén az erdők alját még csak futó havazások szürkítették, a töltésről, a
keskenyvágányú sínekről az is hamar elpárolgott. Andrej, vállán az
alumíniumrúddal, egyenesen az állomásra baktatott, hogy kézihajtánnyal evezzen
ki a természetvédelmi területre. A sínek ott, Nikifor Tescovina kantinja előtt
értek véget.

Félúton, a rezervátum bejáratánál sorompó zárta el a síneket. Az őrszobán Jean
Tomoioaga ezredes messziről 68látta, ki az, aki közeledik, kilépett a töltés
mellé, hogy megnyissa előtte az utat.

Andrej lefékezte a járgányt, kikötözte a sorompó villájához, nehogy a lejtőn
visszaguruljon. Az ezredes látta, nem siet, priccse alól elővette a sakkot.
Nyitott ajtó mellett, a padlóra terített kockás ingen tologatták az otromba
kicsi faragványokat; az egészet egyetlen mozdulattal össze lehetett szedni, ha
netán közelednék valaki.

Jean Tomoioaga ezredes tudta, cimborája még nem járt túl a kerítéseken, ezért
figyelmeztette, a sorompó után erősen emelkedni kezd a pálya, indulás előtt
nem ártana alaposan bezsírozni a tengelyeket. A faggyú, benne széles
falapáttal, egy bödönben állt az őrszoba eresze alatt. Amíg Andrej kenegetett,
az ezredes megvizsgálta az alumíniumrudat. Kihúzogatta egymásból a csöveket,
mígnem az egyik közbülső elemen ott sötétlett vastagon bevésve a Puiu Borcan
ezredes, a halott erdőbiztos neve.

Úgy látszik, ha nem temetik is el, de a helyet, ahol műanyag zsákokkal
letakarva hevert a földhöz szegezve, mégiscsak megjelölik a fényes, messzire
világító alumíniumrúddal. A sok rákötözött színes szalag, főleg a
narancsvörös, még a sűrű ködön is átdereng, s azokon a bizonyos lyukakon,
furatokon pedig majd a szél fuvolázgat. Így szükség esetén éjszaka is meg
lehet közelíteni, vagy később, amikor majd a hófúvások betemetik.

– A katonák mesélik, akik rátaláltak – tette hozzá Jean Tomoioaga ezredes –,
már meg volt csipegetve egy kicsit. Persze, a denevérek.

– Jó vicc – morogta Andrej –, télen a denevér alszik.

Eloldozta a járgányt, beült a hajtókar mögé, és továbbhajtott. Szemben vakító
habokkal zúdult lefelé a patak, robaja egészen elnyomta a járgány nyikorgását.
De a kerekek hangja messzire végigfutott a sínekben, egészen a végállomásig,
ott búgott, zümmögött a vágány végét jelző 69bakokban. A kantinba is
behallatszott, mert amikor a hajtány föltűnt az utolsó kanyarban, Nikifor
Tescovina már összefont karokkal várakozott a sínek végében.

– Lefogadom, a kislányomat keresed – fogadta. – Sajnos, nincs itthon. Sétálni
ment Géza Hutirával.

Bár a magaslatokról, bolyongó szürke ködökkel a tél ereszkedett befelé, ő
fedetlen fővel, ujjatlan tornaingben, lyukas katonanadrágban, mezítlábra
húzott bőrszandálban ácsorgott a sárban, ami körülötte tele volt gyermekek
mezítlábas nyomaival.

– Előbb megkeresem a meteorológust – mondta Andrej. – Visszatérőben nálad
töltöm az éjszakát.

– Igen, tudom. Lehet, én már aludni fogok. Gyere, igyunk meg most egy-két
pohárral.

A kantin egyetlen hosszú, nyirkos, gombaszagú teremből állt, egyik felén
hevenyészett söntéssel, mögötte a tűzhely körül konyhaféleséget rendeztek be,
alvóhelyül ott állt a sarokban egy széles priccs is. A teremben három medvész
gubbasztott, magas nyakú, szutyoktól fényes zubbonyt viseltek, amelyet – talán
a medvekarmok ellen – vértek, vasalások, szegecselések borítottak. A
főmedvész, Oleinek doki egyedül terpeszkedett egy asztal mellett, a fal
mentén, egy keskeny lócán, egymásba karolva az albínó ikrek iszogattak. A
nyakukban viselt bádoglapocska tanúsága szerint – és ez még ikrek esetében is
a legnagyobb ritkaság – a nevük is ugyanaz volt: mind a kettőt Hamza
Petrikának hívták. Andrejt először látták, feléje most nyelvüket öltögették.

Előkerült Nikifor Tescovina két sötét hajú gyereke, az alumíniumrúd fénye
csalogatta őket az asztalok közé. Megnyalogatták a csillogó csöveket, ujjukkal
kitapogatták a belevésett betűket. Ők látták utoljára élve Puiu Borcan
ezredest, aki innen a kantinból indult el végső nyughelye felé. A halál
közelségétől már majdnem átlátszó volt, csak 70a körvonalai remegtek az asztal
mellett, ahova forralt bort inni még utoljára leült, szép nagy füle, mint az
összegyűrt celofán, a láztól áttetszően, petyhüdten csillogott. Mire
rátaláltak – így szól a mendemonda –, jócskán meg volt már csipegetve.

– Akkor majd este egy kicsit zavarni foglak.

– Gyere csak. Mondom, tudok a dologról.

A meteorológus háza felé szemközt a kantinnal, keskeny völgy alján vezetett az
ösvény. Odáig túrásokkal teli rét nyújtózott, peremén a törpe Gábriel Dunka
mászkált négykézláb a buckák között. Vállig érő vastag kesztyűféleséget
hordott, időnként egész karjával benyúlt a föld alá, valamilyen rejtett kis
üregbe. Régtől fogva üzletelt Nikifor Tescovinával, a kantin körüli réten
mormotát fogott neki. Valahányszor szerelvény vagy csak egyszerű kézihajtány
közeledett a síneken, a mormoták kiözönlöttek a buckák közé.

– Ide figyelj – szólította meg Andrej a törpét –, most nem hall senki. Tudok a
bizniszeidről, tele vagy pénzzel. Adj kölcsön nekem.

– Sarokba szorítottál. És mennyi kellene?

– Négy darab húszdollárosra gondoltam. Egyszer majd biztos megadom. Pontosan
négy darab kell, rajtuk múlik az életem.

– Most menj, Niki Tescovina áll az ablak mögött.

Géza Hutira háza felé félúton kicsit kiszélesedett a völgy, az oldalból vörös
ér csordogált a patakba. Zümmögő forrásnak nevezték a helyet, mivel a csalán
közé eldobott üres üvegek száján éjjel-nappal a szél dudorászott. A zümmögő
forrásból pezsgő, vasas ásványvíz buzgott elő, rozsdásra festette a kicsi
medence falát, vörös lepedék borította a fenyőkéreg vályút, a köveket,
gyökereket, amerre elcsorgott. Még a szaga is olyan volt, mint a vérnek.

Ráadásul véres is volt egy kicsit. Bebe Tescovina a forrás fölé hajolva
löcsbölte magát. Mackóruhája levetve egy kö 71vön hevert, és bár az árnyékból
mindenünnen a fagy és a dér lila fényei tűztek elő hűvösen, ő derekán csak
valami kicsi pelenkaféleséget hordott. Vézna gyermekcombján, csontos lábszárán
keskeny, szertefutó erekben vér csordogált.

Géza Hutira egy tuskón ült, földig érő hajáról most aligha lehetett volna
ráismerni, mivel nyaka alatt betűrve, ruhája alatt viselte. Messzire illatozó,
kakukkfűvel tömött pipát szívott, lába közelében egy üres üveg dudorászott. A
füst ellebbenő fátylán át Bebe Tescovina cingár testét méregette, a vér
kacskaringós ösvényeit a víztől gyöngyöző csenevész combokon. Andrejt, aki a
patak robaja mögött észrevétlenül közeledett, csak akkor vette észre, amikor
válláról az alumíniumpózna a szemébe villant.

– Van szerencsém – köszöntötte. – Gondoltam, ma keresni fog valaki. Máris
indulunk, hogy idejében visszatérhessen. – Fölkelt a kőről, pipával a foga
között nyújtózott, és odakiáltott Bebe Tescovinának: – Van egy kis dolgom az
úrral. Ha ráérsz, holnap ilyenkor gyere ugyanide.

Bebe Tescovinának most csak rövidre nyírt piros haja világított, szeme, mint
az áfonya, hamvas és kék volt, le nem vette Géza Hutiráról. Csalódottan,
lassan öltözött, amikor azok ketten elindultak.

Az ösvény el-eltünedezett, elmerült a patakmederben, látszott, az az egy
ember, aki használja, mindig gumicsizmában jár. A kicsi öblök peremét már
jéghártya borította, körülöttük az üveges mázzal borított köveken, csillogó
ágakon billegetők mártóztak.

– Egyedül él? – szólalt meg Andrej a kaptatótól kicsit lihegve.

– Hogy én? Ha szabad kérdeznem, ezzel most hová akar kilyukadni?

– Csak érdeklődöm, milyen az élete. Magam is a magány szerelmese vagyok. Talán
rokon lelkek.
72

– Rokon lelkek. – Géza Hutira megállt, elnézően végigmérte a másikat. – Az
más. Akkor megmondom. Van nálam egy ember. Úgyis találkozni fog vele.

Géza Hutira háza a völgy legvégében, túl az erdőhatáron állt, ahol
sziklagörgetegek és a közöttük vakító ezüst csermelyek találkoztak. Az erdő
szövedéke arrafele hirtelen megritkult, az árkoktól szabdalt meredeken már
csak néhány csapzott, szakállzuzmótól őszülő fenyő kapaszkodott. Nemrég
emelkedhetett föl onnan a felhő, a zsindelyfedél még szivárványos cseppektől
gyöngyözött. A közelben a meteorológus négylábú fehér műszeres szekrénye
csillogott, távolabb, néhány szabad ég alatt álló, megfigyelésre való eszközön
mozdulatlan varjak gunnyasztottak.

A ház küszöbén Béla Bundasian, Andrej fogadott fia ült imára kulcsolt kézzel,
malmozó hüvelykujjakkal, lába közelében fölborult denaturált szeszes üveg.
Mint az örmények, korán kopaszodott, fényes barna homloka magasan hátraborult,
bozontos szemöldökétől, a vastag szemüveglencséktől kicsit baglyos volt a
tekintete. Szemüvege mögül merev, közönyös arccal bámult mostohaapjára, nem
látszott rajta átsuhanni sem öröm, sem csodálkozás. Alig mozdult, amikor az
vállán az alumíniumrúddal megállt előtte.

– Maga az? – morogta, mintegy csak magával tudatva, amit lát. – Hogy az
ördögbe kerül ide?

– Téged kereslek – súgta Andrej. – Öt éve a nyomodban vagyok.

– Az enyémben? De hát minek?

– Úgy néz ki, sikerült kifognom rajtuk. Látni akartalak, és most itt vagyok.

– Azért, hogy lásson, csak úgy?

– Nincs senkim kívüled.

Béla Bundasian belekortyolt az üres üvegbe, türelmesen tartotta a szája előtt,
míg az utolsó cseppek is belecsorog 73nak. Aztán nyálát hosszan eleresztve
maga elé köpött, s megcsóválta a fejét. – Borzasztó.

Géza Hutira kihozott a házból egy távcsövet, körbehordozta a rideg völgykatlan
fölött, végül egy friss havazástól még fénylő magaslatra szegezte. A
hegygerincen kőhalom éles taraja remegett: műanyag zsákokkal letakarva Borcan
ezredes hevert ott. Közelébe kellett kitűzni az alumíniumrudat.

– Látom, régi ismerősre bukkant – mondta nyíltan Géza Hutira. Átnyújtotta
Andrejnak is a távcsövet, nézzen bele. – De számíthat a diszkréciómra.
Kérdezősködni nem fogok.

– Ezt máris köszönettel nyugtázom. Nem tagadom, az illetőt valóban ismerem.
Lesz majd vele egy kis megbeszélnivalóm.

– Amíg tárgyalnak, én majd elfordulok, bedugom a fülemet, esetleg odébb
megyek.

– Azt már ne – szólt közbe Béla Bundasian. – Arra kérlek, igenis tartsd nyitva
a füled. Hogy veszi ki magát, hogy titkaim vannak előtted.

A fennsíkot, ahova nemsokára mind a hárman kikapaszkodtak, már vékony, mákosan
szürke hótakaró borította. Alkonyodott, a szemközti meredeken a hasadékok
mélyéről jég világított, arrafele kanyarogtak Géza Hutira ösvényei a
szirtekre, ahova kijárt a műszereit leolvasni. Most is, miután jégsarkakat
kötött, derekát körbetekerte acélhuzallal, vállán az alumíniumrúddal egymaga
vágott neki a meredeknek.

Mire elérte a nyerget, később azt a kőhalmot, amelynek tövében Puiu Borcan
ezredes hevert letakarva, már szürkülődött. Andrej fogadott fiával szótlanul
várakozott az aljban, mindketten a távoli, égre vetülő alakot figyelték,
mígnem egyik percről a másikra elmerült az ereszkedő alkonyatban. Az estével
egyszerre a fennsíkra hatalmas fekete denevér ereszkedett, árnya egy darabig a
deres törpe 74fenyők, borókák fölött imbolygott, majd távozóban az is elmerült
a szürkületben. A megboldogult erdőbiztos gazdátlan, kóbor esernyője volt az.

Váratlan elült a szél, csend támadt a katlan csupasz falai között, mint egy
üres üvegben. A magasból időnként fémes kopácsolás, csilingelés hallatszott,
ahogy Géza Hutira kövek közé ékelte a rudat, ahogy körülötte cövekeket vert a
földbe, végül, amint a kifeszített huzalok megpendültek. A patakok morajlása,
mint az eső utáni pára, függönyökben emelkedett a völgyekből.

– Beleolvastam a naplódba – kezdte Andrej. – Gondoltam, kiderül belőle, mibe
keveredtél.

– Azt nagyon, de nagyon rosszul tette.

– Így aztán előbb Connie Illafeldnél kerestelek, de már hiába. Abból jöttem
rá, a baj sokkalta nagyobb.

– Nem tudom, miféle bajról beszél. Bajnak én inkább azt nevezném, hogy
beleolvasott a naplómba, a kutyafáját neki.

– Meg kellett tennem. Gondoltam, kitalálom belőle, mi történhetett veled.

– Látja, nem történt semmi. Tudja maga jól, mennyire utálom az ilyesmit.

– A végén mégiscsak rád találtam. Évek óta kereslek, jelenleg Dobrinban élek,
a közelben. El foglak vinni innen.

– Erről tegyen le, sürgősen. Többet ne törődjék velem. Jól megvagyok egymagam.

– Úgy számítom, valamikor tavasszal, legkésőbb nyár elején érted jövök.
Mondom, nincs senkim kívüled.

– Csak nem képzeli, hogy magával megyek. Itt maradok, és ha nem hagy békén, ki
nem találná, mit teszek. Gondom lesz, hogy kitudódjék, mit is keres maga olyan
nagy hévvel a tilalmas övezetben.

Géza Hutira, úgy látszik, befejezte a munkát, a pózna kövek közé ékelve,
huzallal kifeszítve állt a tetőn, mert egyszerre csak hallatszott, amint a
gerincen átbukó szél 75fuvolázni kezd a furatokon. Közeledő léptei alól kövek
indultak el a meredélyen. Aztán már viharlámpája suhogása is hallatszott, de
annyira ismerte a hegyoldal minden domborulatát, hogy csak az aljban gyújtotta
meg, amikor a két várakozó férfi közelébe ért.

Akkor váratlanul az egész lejtő sziporkázni kezdett. A vékony hólepel alatt
fölizzottak a kövek, s ameddig a lámpa fényköre a havon szétterült, kék, zöld
és rézfényű csillámlás hullámzott végig.

A Dobrin lejtőin, mielőtt a területre medvéket hoztak, ércet bányásztak. A
fennsíkról kötélpálya vezetett a völgybe, az iparvasút rakodójához, s a
tartóoszlopoknál, ahol a csillék a csigákon átdöccentek, mindig kipotyogott
néhány rög. Selymes derengéssel a hó alól most az elhullott érc villózott.

Miután a vágatokat bezárták, a kőből, gerendából tákolt kunyhóba, ahol addig a
kötélpálya karbantartója húzta meg magát szerszámaival, Géza Hutira
meteorológus költözött. De mohos köveivel, zuzmóval benőtt, ködszívta
gerendáival a viskó, mintha csak magától nőtt volna oda, mindenestől a
hegyoldalhoz tartozott. Ahogy a viharlámpa fénye szétterült, a helyiség
megtelt zsibongó árnyakkal.

– Ne tartson tőlük – mondta Géza Hutira. – Mind a menyét, mind pedig a földi
kutya az ember barátja.

Béla Bundasian mindjárt egy sarokba, egy halom föltúrt rongyra heveredett,
üveget bontott, s a házban szétáradt az ital illata. Azé a szeszé, amiben egy
ideig a pettyes tárnics gyökerét áztatták.

– Hozni fogok neked egy pokrócot – próbálkozott beszédbe elegyedni Andrej. –
Nem lesz könnyű szerezni, de majd lopatok valamelyik raktárból.

– Azt ne, utálom a pokrócokat.

– Legközelebb már biztos jó hírekkel jövök. Van egy ismerősöm, sofőr. Ráadásul
külföldi, gondolom, érted.
76

– Mondja, mi van a fejében? Hiszen most már maga is közéjük tartozik. Másként
nem lehetne itt.

– Csak így kerülhettem a közeledbe.

– Akkor hát ne lássam többet. – Béla Bundasian fejére húzta a lyukas pokrócot,
a göncöket, s a falnak fordulva folytatta: – És vegye tudomásul, külföldiekkel
nem érintkezem. De belföldiekkel sem. És nagyon jól tudom, mit kell tennem,
hogy megszabaduljak azoktól, akik be akarnak ugratni valamibe.

– Akkor most menjen – lökte oldalba Andrejt Géza Hutira –, amint látom,
zavarja. Ismerem, kicsit érzékeny fiú. Egyébként is már várja magát odalenn a
kantinban Nikifor Tescovina.

Andrejnál volt ugyan zseblámpa, de nem használta. A völgy alján, a sötétség
közepén, a csillagok visszfényétől derengő patak, mint egy vég elgurult
selyem, mutatta az utat a kantinig. Nikifor Tescovina viharlámpást himbálva
várta az éjszakai vendéget.

– Összetoltam neked két asztalt – mondta. – A gyerekek telehintették frissen
nyesett fenyőgallyal. Enni reggel szoktunk, úgyhogy máris elteheted magad
holnapra.

A széles priccs felől, ahol Nikifor Tescovina három gyermekével aludt, a
denaturált szesz és a tárnicsgyökér fanyarkás illata áradt. A lámpát
kioltották, a kályha tüze is rég kihunyt, inkább csak hallatszott, amint az
üveget egymásnak adogatva sűrűn kortyolgatnak. A sötétség mélyén Bebe
Tescovina szeme világított.

A falakban pókok, lárvák sisteregtek, a padláson elindultak a pelék, menyétek,
denevérek, a terem padlóján is kicsi körmök kopogtak. A házat átszőtték az
álomteli szuszogás fonalai. Amikor Nikifor Tescovina mezítlábasan, a deszkát
recsegtetve átlépkedett a helyiségen, már hajnalodott. Andrej is leugrott az
asztalról, ahol addig aludt, és odaállt a kantinos mellé a küszöbre.
Leszegzett fővel, egymás 77mellett állva vizeltek a lépcsőre, bámulták, ahogy
a habos, gőzölgő, kacskaringós erek feketén szétfutnak a deres földön. A völgy
zugaiban a homály burkai alig repedeztek, de a magasban a Dobrin gerincén,
mint a virradat csillaga, az alumíniumpózna villogott.

Amint kivilágosodott, előbújtak a pokrócok alól a gyerekek, Nikifor Tescovina
tüzet rakott. A forró kályhalapon fonnyadt pereszkét, mogyorót, makkot
pirítottak, egy fazék vízben áfonyaindák áztak, az illatos gőz végigsuhintott
a hideg ablakon. Bebe Tescovina tenyerével néha beletörölt, hogy kiláthasson.

– Ez az ember itt fog lakni? – kérdezte.

– Még nem tudom – mondta az apja.

– Ha kell, átadom a helyem, elköltözöm. Géza Hutira azt ígérte, magához vesz.
Talán még ma örökre elmegyek innen.

– Menj, ha gondolod. Én elengedlek.

Reggeli után Andrej elköszönt Nikifor Tescovinától, de az melléje szegődött,
együtt vágtak át a keskeny, deres réten a vágányok felé.

– Mit szólsz – kérdezte Nikifor Tescovina – az illetőhöz, aki Géza Hutiránál
lakik?

– Semmi különöset.

– Nem először látod, ugye.

– Hm, ahogy vesszük.

– Hogy tudd: az éjjel a faluban járt. Pedig nem is szabad neki, és nem is
szokott.

Andrej éppen az ütközőbak fölé hajolva a kézihajtány láncát oldozta el, lassan
kiegyenesedett, megtapogatta a gyomrát. Kitátotta a száját, és maga elé hányt
az ülésre. A csillogó sűrű nyálban, alvadó vérfonalak alatt pereszkék
remegtek.

– Elrontottad a gyomrod.
78

– Dehogyis. Csak ahogy lehajoltam, véletlenségből kicsurgott belőlem.

– Mint az agyvelőd, olyan.

Andrej tenyerével törölte le az ülést, elhelyezkedett rajta, és a
kézitekerővel kiengedte a féket, hogy a járgány a lejtőn magától
elindulhasson.

– Magam sem értem – magyarázta Nikifor Tescovina –, mitől világít a gyerek
szeme a sötétben. De csak mostanság, hogy megjött az első vérzése.

– Jó, rendben. Majd ezt fogom mondani.

– És akkor majd arról is tudjál, hogy készül tőlem elköltözni. Jó, ha odafönn
idejében tudomást szereznek az ilyen változásokról.

– Igen, magam is hallottam. Majd ugyanúgy elmondom.

– És persze, azt se hallgasd majd el, hogy Géza Hutira az, aki befogadja.

– Jó – mondta Andrej Bodor –, biztosíthatlak, mindezt pont így tudatni fogom.
79
\section{(Hamza Petrika szerelme)}

A két Hamza Petrika, aki az ősz egyik legutolsó éjszakáján fölnyársalta magát,
a Dobrin természetvédelmi területen, Oleinek doki medvészetében dolgozott. Pár
nappal az eset előtt még a faluban látták őket – a forradalom ünnepére
valamennyi erdőkerülő rendkívüli kimenőt kapott –, egész délután a késdobálók
sátra előtt ácsorogtak a Sinistra partján táborozó mutatványosoknál, s az
elvillanó, becsapódó pengéket figyelték. A közönség közben inkább őket
bámulta: senki nem látott még két ennyire egyforma, kék bőrű, piros szemű,
árvalányhajú fiatalembert. Albínó ikrek voltak, és annyira hasonlítottak, hogy
ugyanazon a helyen ráncosodott rajtuk a vastag kezes-lábas medvészöltözék, az
orrukból eregetett pára útja azt mutatta, még a levegőt is egyszerre veszik,
és ami a dolgok teteje: a nyakukban csüngő bádogtáblácska szerint
mindkettőjüket Hamza Petrikának hívták.

Az a pár ember, aki a drótsövénnyel, palánkkal elkerített rezervátumban
dolgozott, amellett hogy csak külön engedéllyel járhatott be a faluba, nevét
fémtáblácskára vésve láncon a nyakában viselte. Telente, bár hébe-hóba
beoltották őket, az erdőlakók gyakran megbetegedtek – a járványt arrafelé, ki
tudja, mi okból, tunguz náthának nevezték –, s ha valamelyikük elkóborolva
kinyúlt a bozótban, később bizony jól jött a nyakában csüngő felirat. A
Sinistra partjait ős, vadon erdők szegélyezték, nem mindig találtak idejében a
halottra.

Dobrin erdőkerületben egyetlen orvosi rendelő műkö 80dött, és amikor elterjedt
a hír, hogy Puiu Borcan ezredest a tunguz nátha terítette le, ennek udvarát
ellepték a fadöntők, útkaparók, gombászok, és természetesen megjelentek a
medvészek is. Mindenki az oltását követelte. Négy vagy öt napon át várakoztak
a bezárt rendelő előtt, a tornácra vezető lépcsőkön vagy egyszerűen az udvar
kövein üldögélve, a szerencsésebbje – bár azok is egyre sápadtabban
kornyadoztak – a vöröskeresztekkel telepingált kerítés tövében kapott helyet.
A felcserek zavarodottan lestek kifele a gézfüggönyök mögül, valamelyikük
időnként kiállt a küszöbre – pecsétes, rongyos fehér köpeny volt rajta, alatta
fakózöld katonanadrág, mezítlábra húzott szandál, piszoktól barna
griffmadárkarmokkal –, és türelemre intette a várakozókat: még nem érkezett el
az oltás hivatalos ideje. Közben akárhogyan is, de már ősz vége felé járt, a
déli verőfényben a lehelet ezüstös párája lengedezett az udvar fölött.

A negyedik vagy ötödik napon, már úgy estefelé, egyszerre az alkony lehangoló
fényeivel megérkeztek a szürke gúnárok, és mindenkit hazaküldtek. Ők a Coca
Mavrodin emberei voltak; mind hosszú nyakú, gombszemű alakok, bőrük vékony
volt, áttetsző, könnyű pókhálóhaj a fülük körül, arcuk teljesen ránctalan.
Ezzel a sok hasonlatossággal valami libaszerű tényleg volt a külsejükben.

Most kihirdették, az idén a téli járvány elmarad, oltásra sem lesz tehát
szükség, mindenki térjen békében haza. Miután a felcsereket előcsalogatták a
rendelőből, saját kezűleg hordták ki az udvarra a gyógyszeres dobozokat, és az
egészet széttaposták. Ropogott a sok ampulla a talpuk alatt, az oltószer
kesernyés illata terjengett a kerítések mentén, megült a kertekben a
szilvafák, boglyák között, elkeveredett az ázott avaréval.

Jó hír volt ez, szétszéledtek az erdőkerülők, a hasonszőrű vadócok, szinte
lábujjhegyen, a megkönnyebbüléstől kicsit 81zavarodottan. Az ereszkedő estében
még sokáig hallatszott, amint gumicsizmás, bocskoros talpak neszelnek a
harmatos ösvényeken. Mindenki elment, csak Géza Kökény maradt ott, a lépcsők
tövében pipázva, ő, akiről egyébként is azt beszélték, nem fog rajta semmilyen
betegség.

Magam is elindultam a sötétbe borult főutcán, amelynek végében a vasútállomás
fényei derengtek. Még ott, a rendelő közelében Oleinek dokival, a főmedvésszel
találkoztam, és az egyik Hamza Petrikával. Alkalmi ivócimborámat, az előttem
bandukoló dokit ismertem fel előbb, természetesen a szagáról. Véletlenségből
sem orvosságszaga volt, neki csak a neve volt doki, mindig is medvékkel
foglalkozott. Vad, émelyítő állatszaga volt, mint egy lehugyozott bokornak. A
természetvédelmi területen, egy romos kápolnában s az elhagyott bánya
vágataiban hatvan-hetven, de lehet, százhatvan-százhetven medvét tartottak.
Ivócimborám, a főmedvész és az albínó ikerpár gondozta őket.

Oleinek doki meghívott egy korty italra, s amint ott a harmattól csöndes, puha
úton az állomás felé baktattunk, egyszerre csak észrevettem, közelünkben az
egyik Hamza Petrika árvalányhaja világít. A dokival volt persze, de mint
valami kis selyemkutya, körülötte somfordálva, csak úgy tisztes távolból
kísérte. A másik Hamza Petrika nyilván az erdőn maradt, a medvékkel.

A rezervátumba keskenyvágányú pálya vezetett, a medvéknek kisvasúton
szállították az eleséget. És amíg az első hó le nem hullott, az a pár ember,
aki a telepen dolgozott, kézihajtányon járt be Dobrin Citybe. Ezek ketten most
nyilván az állomás felé tartottak.

A rakodó ereszén viharlámpás lógott, fénykörében a pára sárga burája alatt
csomó ember várakozott. Sinistra felől este érkezett a szárnyvonalon közlekedő
vegyesjárat, ami a két harmadosztályú teremkocsi mellett egy tehervagonnal
közlekedett. Hetenként egyszer, vasárnap este, a ra 82kománnyal denaturált
szesz is érkezett, annak egy részét ott helyben ki is osztották. Persze, csak
a jogosultak között. A doki előkereste az italosszelvényeket, Hamza Petrika
kezébe nyomta őket, elküldte sorba állni, hogy váltsa ki kettejük adagját,
mihelyst befut a vonat.

A denaturált szesz – kenyérbélen, szivacsos gombán vagy törött áfonyán
átszűrve fogyasztják – az erdővidék kedvelt itala. Ha véletlenségből nincs
kéznél áfonya vagy szivacsos vargánya, megteszi egy darabka kicsüngő kapca is.
Vagy egy marék föld.

A keskenyvágányú pálya a nagyállomás túlsó feléről indult a természetvédelmi
terület felé, odáig át kellett kelni a váltók lila lámpái, a harmatos fénnyel
izzó sínek fölött. A kisvasút vágányai a deszkalerakat kerítése mentén
kanyarogtak kifelé a faluból. A palánk cövekeit, hogy a gyakori lopásokat
valamelyest megnehezítsék, nemrég kihegyezték, az égre vetülve, távoli fények
permetezésében most mézesen csillogtak. Alattuk állt a kézihajtány, a sínek
végét jelző bakhoz kötözve. Oleinek dokival rátelepedtünk, lestük az esti
vonatot. Már hallatszott zakatolása a távoli hidakon, füttyjelei a
szélcsendben föl-fölröppentek a Sinistra-völgy meredek falai között.

– Ezek elhalasztották a járványt – jegyezte meg Oleinek doki.

– Megtehetik.

– Maga ezt elhiszi?

– Mért ne.

Ha egyszer megült valami furcsa hangulat, és nem kívántam beszélni, velem nem
volt mit kezdeni. Holott itt lett volna az alkalom a dokit kifaggatni,
kérdezősködni, hogy és mint állanak a dolgok odafönn a rezervátumban; talán
tud valami olyasmit Béla Bundasianról, a fogadott fiamról, amit másként sosem
derítek ki. De még jobb érzés volt hallgatni.
83

A doki is inkább medveszagába burkolózott, nem erőltette a beszélgetést, mi
olyan hallgatag ivócimborák voltunk. Nagy ritkán váltottunk egy-egy közömbös,
semmitmondó szót, félmondatot, jobbára egyetértően köhécseltünk. De amikor
közeledni hallatszott Hamza Petrika, tarisznyájában az egymáshoz koccanó
üvegekkel, a főmedvész mindjárt fölugrott, és elébe sietett.

– Akkor hát ide figyelj – szólította meg halkan, kissé fojtottan, mégis szinte
melegen. – Szabad vagy. Most azonnal elmehetsz.

– Tréfálsz, doki.

– Egyáltalán nem. A végén még elkapunk egymástól valamit. Hallhattad a két
füleddel, többet nem oltanak. Jobb, ha elválunk, mindenki dolgára megy.

– Nélküled egy lépést sem tennék szívesen. Mi a testvéremmel örök időkig veled
akarunk maradni. Ha tartasz tőlünk, hát egy darabig meghúzzuk magunkat,
megígérjük, nem nyúlunk hozzád. Kivárjuk, amíg elmúlik ez neked.

– Hiába, én már döntöttem. De azt megígérem, amíg egérutat nem nyertél, én sem
jelentem az esetet a szürke gúnároknak.

És Oleinek doki, hogy hajthatatlanságát jelezze, a tarisznyájából az egyik
üveget, ami a Hamza Petrika adagja lehetett, elébe tette a földre. Sarkon
fordult, és visszatelepedett mellém a kézihajtányra. Onnan még odakiabált
neki:

– Igyál, amennyi beléd fér, aztán igyekezz, szedd a lábad, pucolj el
valamerre. Reggel, ha már árkon-bokron túl leszel, majd jelentem az esetet.

Hamza Petrika ismerhette erről az oldaláról Oleinek dokit, többet nem
alkudozott. Azt még úgy-ahogy láttam, leül a töltés oldalába kortyolgatni. A
doki is lepattintotta az üveg szájáról a bádogkupakot, iszogatni kezdtünk.
Aznap este nem volt kéznél se gomba, se áfonya, a zubbony mandzsettáján
szűrtük át a szeszt.
84

Nyirkos, szurkos csönd ereszkedett a völgyre, mélyén, a deszkarakások között
megnyikkant néha egy-egy bagoly, a tanyákon kutyák ugattak, s később
hallatszott, ahogy a három vagonból álló szerelvény a lejtős pályán lassan
kigurul az állomásról, vissza, Sinistra felé. A sötétből időnként Hamza
Petrika hüppögése is hallatszott. Szívta az orrát, szaporán szörtyögött, mint
egy sértődött, kényes kiskutya. Az albínók, gondoltam, gyenge idegzetűek,
könnyen elveszítik a fejüket.

– Nem zavarja a szagom? – kérdezte előzékenyen Oleinek doki, biztos csak
azért, hogy megtörje, talán éppen a szagától keletkezett csendet. – Vallja meg
őszintén, tudom jól, kissé büdös vagyok.

– Szó sincs róla.

– Mert volt már néhány hülye esetem. – Nagyon is normális a szaga.

– Ne áltasson, a fehérnépek rendre kiadták az utamat. Meg is mondták,
kimondottan a szagom miatt. Nem mintha olyan nagyon izgattak volna. Aztán
odahelyezték nekem az ikreket.

– Az iker sok szempontból jó.

– Ahogy mondja. Az iker finom. Mi hárman sok örömöt szereztünk egymásnak, mint
egy boldog kis család éltünk odafönn, mind a mai napig. De most vége. Fő az
egészség. – Fölkelt a járgányról, szinte megkönnyebbedve kiáltott oda Hamza
Petrikának: – Hallod-e! Azért jó modor is van a világon. Méltóztass köszönni,
mielőtt világgá indulsz.

De a sötétségben, ahonnan az előbb még Hamza Petrika gyerekesen fölcsukló
sírása hallatszott, most csak a töltés kövei mocorogtak. A medvész helyén
átlátszatlan feketeség lebegett, érezni lehetett, nincs benne senki, teljesen
üres.

Oleinek doki körbesétált, lábával széles íveket kaszálva átkutatta a
hulladékkal, gazzal, száraz kóróval tele terepet. 85Útközben egy üres üveget
is fölrúgott, végül egy pár gumicsizmával tért vissza.

– Az övé – morogta, miközben többször alaposan beleszagolt –, ráismerek. De
hát mi üthetett bele, hogy levetette. Mezítláb hová a picsába mehetett?

Azért visszaült a hajtányra, és továbbra is a zubbony mandzsettájára
töltögetve szívogattuk a denaturált szeszt. Egy idő után a doki hanyatt dőlt
kényelmesen, én is elnyúltam kicsit bódultan a deszkaülésen. Aztán egyszerre,
egy időben vettük észre, hogy a palánk tetején ellobban egy szál gyufa, majd
egyszer-kétszer fölizzik a cigaretta parazsa. Az égre, a csillagok, mennyei
ködök közé, mint valami titokzatos fekete üreg, a Hamza Petrika árnya vetült.
A magasban tanyázott hát, ott füstölt a kerítés tetején.

– Jól elbújtál, mondhatom – kiáltott oda neki Oleinek doki. – Már szörnyen
izgultunk, mi lehet veled. A barátom egy kicsit meg is sértődött, hogy csak
úgy elmész köszönés nélkül. – És mivel Hamza Petrika nem válaszolt, még hamar
hozzátette: – Ugyan bizony megtudhatnám, miért nem kínálsz meg minket is a
dugi szivarodból?

Erre Hamza Petrika annyit mondott:

– Csak.

Mint egy csobbanás, úgy szólt ez. Mint amikor, mondjuk, egy zsebóra hull
éjszaka a patakba. Nemsokára lehullott kezéből a cigaretta is. A gaz között,
mint egy szentjánosbogár, pislogott.

– Hm.

Oleinek doki fölkelt, megkereste a cigarettavéget. Szipkába tette, s egymást
kínálgatva nyugodtan végigszívtuk.

– Hát igen, ezek a kurva ikrek – morogta. – Ilyenek. Pár órára elszakadnak
egymástól, és már badarságokat művelnek. A fene igazodik el rajtuk.

Azért valamennyire maga is furcsállhatta a dolgot, mert a hajtány peremén ülve
egyre szólongatta Hamza Petrikát. 86Miután semmiféle választ nem kapott, a
gumicsizmákat letette az ülésre, és elsétált a palánkig. Föl-alá járkált
mellette, a végén megragadta az egyik cöveket, és idegesen megrázta.

– Na!

Amikor valamivel később elengedte, ujjai finom, neszelő hangon váltak el,
mintha ragasztó csiriz tartaná őket vissza. A friss vér hangja ilyen.

Visszaült a hajtányra, mérgesen fújtatott – ptü! –, maga elé köpött, tenyerét
az ülés deszkájához kente. Megkereste az üveget, és most már minden
elővigyázatosság nélkül, egyenesen belekortyolt, majd felém nyújtotta.

– Nyaljon bele maga is hamar – súgta. – Aztán azt ajánlom, távozzunk innen. A
fiú, azt hiszem, karóba húzta magát.

– Az mi az ördögöt jelent?

– Na mégis, mit! Megkeresi a fenekén a lyukat, beleilleszti a rúd hegyét, és –
zutty! – beleül.

– Nem jön, hogy higgyem.

– Hiszi, nem hiszi, jöjjön, menjünk innen.

A doki eloldozta a kézihajtány láncát a baktól, kiengedte a fékeket,
megragadta az evezőkart, és máris indított. Hamza Petrika ott maradt a kerítés
tetején, árnya a csillagok közé vetült, alatta udvaros fénnyel a váltók lila
lámpái hunyorogtak.

– Az lesz a legjobb, ha most magammal viszem egy darabon – mondta a doki. –
Jobb, ha egy ideig együtt maradunk.

– Jó – feleltem, – vigyen el, mondjuk, Jean Tomoioaga ezredes őrházáig. Tudja,
ő a barátom nekem.

– Persze, tudom. És közben bevedeljük, ami még az üvegben maradt. Vagy maga
mire gondol? Van valami egyéb ötlete? Mit lehetne tenni.

– Nekem most nem jut eszembe semmi.
87

– Nekem sem. Legjobb, ha innen minél távolabb kerülünk.

– És mondja, doki, úgy egyébként, hogy szokták ezeket onnan levenni?

– Sehogy – felelte mérgesen. – Nem szokás. És szerintem, ha valahogy el is
kapja az ember alulról a lábát, azzal csak jobban belerángatja.

– Csak úgy eszembe jutott.

– Ne foglalkozzék vele. Ez az ő dolga, magának nincs joga megmásítani. Ne is
jusson többet eszébe. Egyébként, némelyikkel még napokon át beszélgetni lehet.

Az állomást elhagyva hamar emelkedni kezdett a pálya, keményen evezve
gurultunk ki a faluból. A kerekek hangja messzire előreszaladt a sínekben, a
töltés mentén s a hegyoldalon végighullámzott a kutyaugatás.

– Ezek szerint megüresedett magánál egy hely – jegyeztem meg útközben.

– Nem kizárt, kettő.

– Mert én kimennék magához az erdőre – folytattam. Beszélek a szanitéc
alezredessel, hátha sikerül protekciósan beoltatnom magam. Ha maga is akarja,
én szívesen odaszegődnék. Azt nem mondom, hogy értek a medvéhez, de
beletanulnék.

– Nem biztatom.

– De ha mégis.

– Majd meglátjuk. Azt hiszem, most egy jó ideig magam maradok.

Oleinek dokit a kézihajtányon a természetvédelmi terület bejáratáig kísértem,
ahol a síneket sorompó zárta el, rajta piros fénnyel viharlámpás égett. Az
őrszobán régi sakkozócimborám, Jean Tomoioaga ezredes lakott. Már induláskor
arra számítottam, elütöm nála valamivel az időt, talán még iszogatunk együtt
is egy keveset, s az éj folyamán, majd gyalogosan, a töltésen végigbaktatva
térek vissza a faluba.
88

A viharlámpás hamarosan a küszöbre került, Jean Tomoioaga ezredes kicserélte
rajta a vörös üveget fehérre, aztán elővette a sakkot. Házilag faragott,
otromba kicsi figurákkal játszottunk egy kockás ingén, amit ő a padlóra
terített. Az egészet egyetlen mozdulattal össze lehetett szedni; a
hegyivadászok nem szerették az ilyen játékokat.

Oleinek doki nem sietett, ő is a küszöbre telepedett a lámpa közelébe, és
várta, hogy felsorakoztassuk a bábukat. Úgy látszik, nem akarózott neki
mindjárt továbbmenni.

– Látom, egyedül térsz vissza – mondta neki Jean Tomoioaga ezredes. – A
barátunk még pluszban is egy kis kimenőt kapott?

– Úgy van, elengedtem.

Jean Tomoioaga ezredes elővett priccse alól egy üveget, a padlóra tette, hogy
mindegyikünk kényelmesen elérhesse. Kékesszürke ital lötyögött benne, olyan
denaturált szesz, amelyet faszenen szűrtek meg. A szén, azt mondják,
egészséges.

– És mikor tér vissza, ha szabad kérdeznem? Tudod jól, minden mozgást be kell
vezetnem a naplóba.

– Ha jön, itt lesz. Akkor majd beírod, ami kell. Ha nem jön, nem írsz be
semmit.

– Jópofa vagy.

A második vagy a harmadik partit játszottuk Jean Tomoioaga ezredessel, amikor
az ajtó előtt, a völgy fekete bársonyán föllobbant Hamza Petrika árvalányhaja.
Nem azé a Hamza Petrikáé, amelyik fölnyársalta magát, hanem a másiké. A hulló
harmattól hamvasan állt a nyitott ajtó előtt, és nem volt egy csepp vérszaga
sem.

– Hol van? – kérdezte szigorúan Oleinek dokitól.

– Magad is látod, nincs itt.

– Haladéktalanul beszélni szeretnék a testvéremmel.

– Most nem lehet.
89

Zsebre dugott kézzel állt az ajtó előtt, és mögénk is bepillantva körbejáratta
tekintetét a kicsi őrszobán.

– Ejnye, doki, látom, elhoztad a csizmáját. Akkor már nem is kérdem, hova lett
a testvérem lába belőle. – És ujjal rám mutatott: – Mondd, netán ez az ember
jön majd a helyünkbe?

– Az még a jövő zenéje – hagyta rá félig-meddig a doki. – De ha már kezded
érteni a dolgot, hát ide figyelj: tudatom veled is, elmehetsz. Szabad vagy,
kerekedj föl mielőbb. Valahol, tudod te, hol, vár rád a testvéred, Hamza
Petrika. Neki is megígértem, nem foglak mindjárt kerestetni.

Hamza Petrika leült a földre, belemarkolt a hajába, de az vékony volt,
hézagos, könnyű, a keze üres maradt. Tenyerébe köpött, eldörzsölte, aztán
fölkelt, nyújtózott egyet. Arca hirtelen kisimult, megnyugodott.

– Rendben doki. Megyek, összeszedem a holminkat. De te is ígérd meg, nem
indulsz mindjárt utánam.

– Ha ez a kívánságod, tessék. Mennyi legyen? Úgy húsz perc megfelel? Vagy
legyen egy félóra?

– Pont annyira gondoltam. Annyi ideig szeretnék teljesen egyedül lenni.

– Oké, fiú. Igazad van, készülj csak el kényelmesen.

Hamza Petrika hóna alá kapta testvére gumicsizmáját, és köszönés nélkül
elindult vissza, a medvészet felé. Mint akiből éppen a lélek távozik, útközben
nagyokat szellentett. Pár lépés után összecsapott mögötte a patak zúgása, a
sötétség bársonya.

Oleinek doki, bár óra nem volt a közelben, készségesen várakozott, s már
kétannyi idő is eltelt, amennyire ígéretet tett, amikor nyújtózkodni kezdett.
Egykedvűen lódította vállára az italosüvegekkel teli tarisznyát, és elindult a
járgány felé.

– Volt szerencsém.
90

– Akkor arra kérem – szóltam utána –, gondolkozzék a dolgon.

– Oké, oké, majd meglátjuk.

Valamivel később, a töltésen baktatva magam is elindultam Dobrin City felé.
Van, akit a talpfákon való járás megnyugtat, másokat bosszant, egyeseket
gondolkodásra késztet. Én egyszerűen a fejembe vettem, elérve a falu szélét,
nem indulok a Baba Rotunda-hágóra, hanem az állomás felé kerülök, hátha
válthatok még egy-két szót Hamza Petrikával. Hogy mit, arról persze fogalmam
sem volt. Gondoltam, majd csak adódik valami. De aztán nem lett a
beszélgetésből semmi.

A sínek mentén baktatva virradatkor érkeztem a dobrini állomásra. A
messziségben, a hágó lila vonala fölött sárgulni kezdett az ég, és én lépésről
lépésre vártam, hogy Hamza Petrika madárijesztőárnya egyszerre csak
rávetüljön. Elhaladtam végig a kerítés mentén, de sehol nem láttam. Szemközt a
rakodó rámpáján, sorban egymás mellett, lábukat lóbálva, nyakukat nyújtogatva
üldögéltek a szürke gúnárok.

Azon a helyen, ahol az este Hamza Petrika cigarettára gyújtott, a kerítésből
egy cövekkaró fele hiányzott, el volt vágva derékban. Töve körül a földet
illatos fűrészpor borította vastagon, már csak az éles hajnali fuvallatnak
volt kicsit fémes szaga – kicsit sós, kicsit édes –, pont, mint a vérnek.

Világosodott, gondoltam, most már le se pihenek, inkább fölkeresem a szanitéc
alezredest, hátha tényleg beolt protekciósan csak engem. Itt volna a nagy
alkalom, hogy elszegődhessek medvésznek.
91
\section{(Connie Illafeld szőre)}

Azon a tavaszon, amikor segédhullaőr koromban végre megismertem Connie
Illafeldet, nem sok örömöm tellett a találkozásban: jóformán már egyetlen
nyelven sem beszélt. Keverte őket összevissza, és csak az értett vele szót
valamelyest, aki ukránul, németül, románul és magyarul egyformán tudott, de
nem ártott, ha az illető például a ruszin és a cipszer tájszólásokat is
ismeri. Kevés ilyen ember élt Dobrin erdőkerületben, egyikük a
természetvédelmi terület főmedvésze volt, cimborám, Oleinek doki.

A Connie Illafeld afféle művésznév volt, a nőt – a bukovinai bojárok, az
Illarionok ivadéka egyszerű hegyi lakók között élt az egykori birtokon –
eredeti nevén Cornelia Illarionnak hívták. Az még talán előfordulhatott, hogy
valakiket külön-külön Cornelia Illarionnak, esetleg Connie Illafeldnek hívnak,
de a két nevet együtt csak egyetlen személy viselhette.

Úgyhogy amikor az írnoki asztalra készített, vöröskereszttel ellátott
iratgyűjtőn megpillantottam Cornelia Illarion nevét, amelyet macskakörmök
között, pirossal kicirkalmazva a művészneve követett, tudtam, ő az, szinte
rokonom, fogadott fiam egykori szerelme. Égtem a kíváncsiságtól, hogy végre
magam is a két szememmel lássam a lényt, aki évekkel korábban annyira
megvadította.

Akkoriban állományon kívüli civilként bedolgoztam a dobrini hegyivadászoknak,
kisebb-nagyobb titkos megbízatás mellett a kerületi halottkém helyettese,
ahogy ott nevezték, segédhullaőr voltam. A halottasház a laktanya 92udvarán
állott egy nyirkos, mohos zugban, és ha éppen az ürességtől kongott, azaz nem
volt munka, Titus Tomoioaga ezredesnek segédkeztem az irodán. Mivel a terület
mindenestől a hegyivadászokhoz tartozott, ő vezette a nyilvántartásokat
mindazokról, akiket Dobrinba küldtek, munkára irányította az újonnan
érkezetteket. De hát ő is csak egy lassú, álmodozó, szarvaslelkű hegyivadász
volt, órákon át tétlenül bámult kifele az ablakon, a fekete fenyők fölött
elhúzó szürke felhőkre, madarakra, még az ilyen szűkszavú kísérőokmányok
elolvasása is nehezére esett.

A kérdéses napon a déli vonulaton, a még jeges gerinceken nehéz illatoktól
rakottan átbukott az első meleg szél, virágszirom, barkapor úszott a
patakmeder fölött: állítólag az óhitűek húsvétja volt. A tavasszal egyszerre
két újdonsült internált érkezett Dobrinba.

Amikor a verőfénytől, ragyogástól, virágportól szédülten a homályos
irodahelyiségbe léptem, és a vöröskeresztes iratgyűjtőn megpillantottam
Cornelia Illarion nevét, azt hittem, a bódulat teszi ezt velem. De ott állt
ráadásul, szépen kisatírozva a művészneve is; mindez a vöröskereszttől rikító
borítón, s ez mindjárt elárulta, hogy az illetőt a Colonia Sinistra
gyógyintézet irányította oda.

Bár magamat mindig is hidegvérű, fegyelmezett embernek ismertem, akkor
hirtelen elfogott a nyugtalanság. És ami ilyen helyeken egyáltalán nem szokás,
puhatolózni kezdtem Titus Tomoioaga ezredesnél, mit tud. Hogy kerül ide a nő,
kicsoda, miféle szerzet?

– Ó, senki – morogta kicsit álmosan. – A sárgáktól kaptuk, ők küldték, csak
amúgy kedvességből ide. Ha olyan nagyon érdekel, mindjárt megismered, amikor
fölveszed az adatait.

Hát így álltunk. A Colonia Sinistra híres hely, az elmegondozó pavilonjai – az
is tudja, aki sosem látta őket – sárgára voltak festve, éjszaka világítottak.
Sárgáknak egy 93más közt az intézeti adminisztrátorokat, instruktorokat
hívtuk.

– És mi a terved vele? Már tudod, hová helyezed? – érdeklődtem.

– Nagyjából igen. Coca Mavrodin-Mahmudia ezredesnek az a kívánsága, kerüljön
egyenesen a medvékhez. Nincs ugyan valami jó bőrben, de Oleinek doki majd
elboldogul vele. Képzeld, mindenféle nyelven beszél összevissza, mint a
bolondok.

Szóval, Coca Mavrodin Connie Illafeldet a medvészetbe szánta. Arcomon
bizonyára látszott: nem éppen egykedvűen hallgatom ezeket a dolgokat. Titus
Tomoioaga ezredes még hozzá is tette megnyugtatólag:

– Meglátod, így lesz ez rendjén. A doki minden létező nyelven beszél, biztos
szót ért majd vele.

Connie Illafeld, mielőtt kezelésbe vették, egy magaslati településen élt, háza
Punte Sinistra felső végén, közel a vízválasztóhoz, a vasútállomás mellett
állott. Nem is volt az igazi állomás, csak afféle megállóhely, két sínpárral,
kitérővel, ahol a hegy két oldaláról fölkaptató vonatok kipihenték magukat,
vizet vettek, és menetrend szerint bevárták egymást. Az északi lejtő felé
mindjárt egy alagútban tűnt el a pálya, amely egy-egy vonat után még órákon át
lila füstgomolyagokat eregetett. Connie Illafeld házának északi falát is
megkapta már egy kicsit a korom.

A Connie Illafeld tehát művésznév volt, az elvonultan élő utolsó
Illarion-ivadék üvegfestéssel foglalkozott. Ókori jeleneteket pingált,
életképeket a régmúlt időkről, kicsi, akár zsebben is elférő üveglapokra;
csernovitzi és lembergi zsidóknak dolgozott megrendelésre, ki tudja, hogyan,
mi módon juttatta át őket a határon. Jó negyvenes volt, szeme zöld, bőre
fehér, haja fekete.
94

Erdőkerülők, utászok, átutazó hivatásos vadászok néha bizonyára
megkörnyékezték, de a látszat szerint valakinek tartogatta magát. Az alagútőr,
aki sosem aludt, azt állította, idegen vigéc csapja neki a szelet – Galícia
felől jövet éjjelente állítólag átkelt a Tiszán –, titokban néha meglátogatja.
De ez csak az álmatlan bakter meséje volt: mindenki tudta, a folyópartot, ahol
a határ húzódott, áthatolhatatlan drótsövény szegélyezte. Így is, úgy is
mindegy, ha volt is Connie Illafeldnek titkon valakije, annak azon a tavaszon
kiadták az útját. Akkor jelent meg életében az igazi: Béla Bundasian, a
fogadott fiam.

Punte Sinistra állomására egy estefele befutott a távolsági személyvonat, Béla
Bundasian csomag nélkül leszállt róla inni, a friss forrás ott buzgott a sínek
közelében. Ahogy fölébe hajolt, hátán fölcsúszott az ing, a zubbony gallérja
nyakára borult, eltakarta fülét, így nem hallotta, ahogy a kövek megreccsennek
a talpfák alatt, s lassan elmozdul a vonat. A pálya innen mindkét irányba
lejtett, a mozdonyvezetők indításkor csak kiengedték a fékeket, azzal a
szerelvény magától gurulni kezdett. Aznap a távolsági személy valamilyen okból
nem várta be párját a túlsó oldalról, úgyhogy amikor fogadott fiam
kiegyenesedett, és kezdte volna vidáman törülgetni a száját, épp az utolsó
vagonokat látta eltünedezni az alagútban.

Távolsági személy ezen a vonalon naponta csak egyetlenegy közlekedett, s ha a
lemaradt utas kitartott úticélja mellett, másnap estig mindenképpen várakoznia
kellett.

Tavasz volt akkor is, virágvasárnap, valami ilyesmi, a levegő tele illatokkal,
a tisztások mögött feketéllő fenyőerdőkből alkonyat után is szédítő
madárcsattogás áradt. Connie Illafeld föltűrt szoknyában az ablakpárkányon
térdelt, az üveget fényesítette, fehér karja világított a szürkületben. Biztos
hívogató lehetett az a hang is, ahogy a nedves 95papír csúszkált az üvegen.
Képzelem, mint egy végrehajtó, Béla Bundasian lecövekelt a kapu előtt.

Megérezte-e Connie Illafeld, merről fúj a szél? Biztos. Keze az ablakon
lelassult, szája résén át látszott, hegyes fehér fogai édes nyálban úsznak,
izzó zöld szeme féktelen örömben, mindez fogadott fiamnak szólt. Béla
Bundasian, lévén félig örmény, pergamenszínű volt, szeme fehérje kissé olajos,
szemöldöke máris csupa bozont, bárkinek első látásra megtetszhetett. Tudta ezt
magáról, mindjárt előadta a lemaradt utas történetét: kottapapírért igyekezett
Moldvába, a putnai papírgyárba, amikor balszerencséjére – vagy most már ki
tudja? – leszállt a vonatról vizet inni. Azt, hogy így történt, maga Connie
Illafeld is látta, behívta hát, pihenje ki magát, s ha netán még szomjas, ott
a vízzel teli vödör, most már kedvére ihat.

A ház padlóját puha csergepokróc borította, Béla Bundasian cipőjét
illedelmesen a küszöbön hagyta, s úgy harisnyásan véletlenségből mindjárt
Connie Illafeld csupasz lábára lépett, s mivel jólesett, rajta is hagyta. A
falakból, a parasztosan ácsolt bútorokból, a szőttesekből és párnákból
hívogató tésztaillat áradt. Tésztaszaga volt magának Cornelia Illarionnak is,
pihés hónaljának, párnás, gyöngyházfényű combjának, holott korát tekintve Béla
Bundasian mamája is lehetett volna. A féktelen vágy illata volt ez, ami most,
mint a kelésnek indult kovász, előbuggyant belőle. Pár perc elteltével már
eszeveszetten nyalták-falták egymást.

Connie Illafeld egyik fiókjában szárított telekiát tartott – virágot, levelet,
összemorzsolt szárat –, most az egészet kiszórta a csergére, és ebben a bódító
fanyar illatban nyújtózkodtak egyhuzamban két vagy három héten át, ablakaikat
elborították a szerelem párái. Jóval később – mindennek rég vége volt már
akkor – belepillantottam Béla Bundasian naplójába, amelyet a szerelmes
hetekről, hónapokról veze 96tett, onnan tudom mindezeket. Azt írta, lehetetlen
volt betelni vele, ha csak ránézett, az az érzése támadt, még a lábujjkái
között is egy-egy kis éhes punci lapul, a legjobb lett volna az egész nőt
kihörpinteni, mint egy pohár vizet. Egy ilyen szerelemnek persze hogy titkos
utakon már közeledett a vége.

Azokban az időkben nevelt fiamat ritkán láttam, ha kottapapírért Moldvába
utazott – tulajdonképpen kottamásolással foglalkozott –, hetekig nem került
elő. Kedveltem a fiút, de hagytam, tegyen kedve szerint, essen át a
tűzkeresztségen, egyébként sem az én vérem. Elvem az volt, csak legvégső
esetben avatkozom a dolgaiba.

Annak is elérkezett az ideje. Egyik ilyen távolléte alatt idegen, szürke úr
kereste, szeme sárga volt, szája keskeny, egy spárgával átkötözött
újságpapírcsomagot hagyott nálam a részére. A csomagot a szürke úr távozása
után nyomban fölbontottam: lengyelül írott, sokszorosítással készült füzetek
voltak benne. Természetesen azonnal elégettem őket, a pernyét föloldottam
vízben, szétlocsoltam a kertben. De ez már mindegy volt, Béla Bundasian
belekeveredett valamibe.

A lengyel füzetekkel való eset után fogadott fiam többé nem került elő. Gyanút
fogtam ugyan már a kezdet kezdetén, de hol kereshettem volna; fölültem a
moldvai távolsági személyre, s egy napnyugta után, csikorgó fagyban, szédítő
szénaillatban érkeztem Punte Sinistrára. Estére elült a szél, az istállókból
fölkelt a langyos szénaillat, és szétterült a zúzmarás réteken. Mégsem töltött
el valami különösebb jó érzés, az alagút felé guruló vagonok fényében mindjárt
észrevettem, Cornelia Illarion ajtaján pecsétek sötétlenek, a kilincsén pedig
vöröskeresztes szalag lobog. Akkoriban, akit a vöröskereszt meglátogatott,
tudhatta, nem áll jól a szénája. A vöröskereszt ajtón vagy kapun a legrosszabb
jelnek számított.
97

Az alagútőr nem volt éppen beszédes kedvében, annyit mégis elmondott: hogyne,
Cornelia Illarion valóban ott, a szemközt sötétlő házban lakott. Igen, csak
lakott. Nemrég, alig pár napja vagy hete, két úr kereste hivatalos papírral,
amelyen az állt, hogy bolond. Mindjárt magukkal is vitték kezelésre a Colonia
Sinistra néven ismert elmegondozóba.

Ami nevelt fiamat, Béla Bundasiant illeti, vele jó négy évvel később
találkoztam itt, a Dobrin-rezervátumban, a meteorológus Géza Hutira házában.
Kiderült, ugyanazon a napon, amikor Cornelia Illariont elszállították, őt az
este érkező vonatnál régi pártfogója, Velman ezredes várta. Egy amolyan
kéretlen, hívatlan jó barát, aki néha fölkereste, bizalmas jóakaróként ellátta
különféle tanácsokkal. A lengyel füzeteket egy szóval sem említette, arra
figyelmeztette Béla Bundasiant, hogy kétes nőügyei miatt kellemetlenségeknek
nézhet elébe. Úgy hírlik, idejét mostanság vidéken tölti, és rendszerint egy
beszámíthatatlan női személynél alszik – majd kiderül, egy takaró alatt vagy
sem –, s ez már közel áll ahhoz, amit a törvény erőszaknak nevez. Ő, a régi
pártfogó igyekszik majd a dolgot elsimítani, hátha megússza pár év
internálással.

Ennyit tudtam hát Connie Illafeldről, amikor az iratgyűjtőn megpillantottam a
nevét, és nemsokára az iratait is magam elé terítettem. Mindez szép lassan
átpergett az agyamon, azzal együtt, hogy a nő Coca Mavrodin ezredes
kívánságának megfelelően a medvékhez került a rezervátumba.

Béla Bundasian is ott lakott, ugyebár fönn az erdőhatáron, a meteorológusnál.
Megtanulta leolvasni a műszereket, a szélkakasok állását, s ha félévenként,
ünnepkor kimenőt is kapott, nem mozdult ki onnan. Legfönnebb a medvészekhez
járt át kockázni, malmozni, feketepéterezni.
98

„Meglátod, így lesz ez rendjén” – ezt mondta Titus Tomoioaga ezredes.

– Akkor is – válaszoltam volna neki valamit, de tanácstalanul abbahagytam a
mondókámat.

– Mi van, mi a bajod? – Titus Tomoioaga ezredes most már gyanakvóan
méregetett.

– Semmi.

Újra átböngésztem a kibocsátási iratokat, majd engedélyt kértem, hogy
távozhassam a vécére, ami a folyosó legvégében állt. Persze hogy kíváncsiság
fogott el, érdekelt az a magamban sokszor elképzelt illatos húsgombóc,
fogadott fiam pompás nője, akit látatlanban, a napló olvasása közben már-már
elirigyeltem tőle.

Mondhatom, nem sok sikerrel jártam. Odakünn, a várakozók padján egy szürke
bőrű, köhögős, bányászsisakos férfi hevert elnyúlva, mellette, rongyos
pufajkában egy csupa csimbók, csupa bunda szőrmók alak imádkozott.
Összekulcsolt kezét, arcát is egybefüggő szőrzet borította.

– Figyelj – mondtam Titus Tomoioaga ezredesnek –, nem tudom, miféle nőről
beszélünk mi itt. Egy árva fehérnép sem várakozik odakünn. Vagy talán
megpróbált meglépni?

– Ott van az.

– Egy bányász heverészik ott, meg egy másik, valami bundás. Kívülük nincs ott
senki.

– Akkor hát mégiscsak ott van.

Connie Illafeld valóban odakünn, a folyosón várakozott. Titus Tomoioaga
ezredes nemsokára maga vezette be. Először nevén szólongatta, de mindjárt
rájött, hogy azt talán nem is érti, úgyhogy kiment érte, és a hóna alatt
tartva betámogatta. A bundás volt az.

Arcát is selymes fekete szőrzet borította, a bolyhok között zöld szeme izzott.
Még a nevét sem tudta. Én azért igyekeztem a jópofa oldaláról nézni a dolgot.
Próbáltam 99összenézni Titus Tomoioaga ezredessel egy-egy közös, sokatmondó
szemvillanásra. És bár sok kedvem nem volt, el-elmosolyodtam, mint általában a
hibbantak fölött.

– Odabenn mindent elfelejt az ember – magyarázta Titus Tomoioaga ezredes. –
Kicsorog az emberből minden, mint a fos.

– De hogy még a nevét se…

– Nem lehet az olyan rossz.

– És te talán másként vélekedel, de nekem például szőrből is egy kicsit sok.

– Szó, ami szó – kacsingatott Titus Tomoioaga ezredes –, alapos kezelésben
részesült. Valamit bizonyára túladagoltak. Nem csodálnám, ha kinőtt volna
egyebe is neki.

– A pucukájára gondolsz?

– Ühüm. Lehet, lesz, aki majd megkeresi neki.

Miután a papírmunkával végeztem, Titus Tomoioaga ezredes megkért, kísérjem át
a leendő medvészt a lakatosműhelybe. Ott készültek azok a kicsi
bádoglapocskák, névjegyek, amiből egyet ő is a nyakában fog viselni.

De közben beállított az irodára Oleinek doki, a főmedvész, aki állítólag
minden lehető nyelvet beszélt. Mindjárt szóba is elegyedett Connie
Illafelddel, és úgy látszott, hamar szót értenek egymással. A végén a doki
kísérte át a lakatosokhoz.

– Látom, hogy izgulsz – mondta Titus Tomoioaga ezredes –, izgulsz valamiért,
de teljesen feleslegesen. Jó kezekbe kerül az illető.

– Az ördögbe is – fakadtam ki, újra óvatlanul.

– No, mi van?

– Becsszóra semmi.

Cornelia Illarion, amikor Oleinek doki oldalán visszatért az irodába, szőrös
nyakán messzire világító fényes táblát viselt, egy vadonatúj óraláncon
függött, aminek a végét elhegesztették, hogy többet onnan soha senki le ne
vehesse. 100 A nevet, amely valaha egy buja tündéré volt, most mi tagadás, egy
állat viselte.

Távozásuk előtt Oleinek doki – egyébként régi ivócimborám – nekem is szentelt
néhány percet. Tőle tudtam meg, fogadott fiam, Béla Bundasian, aznapra, bár
nem járt volna neki, kivételesen kimenőt kapott. Együtt jöttek le
kézihajtányon a rezervátumból, és most a végállomáson iszogat, ahol az
erdőkerülőknek délután denaturált szeszt osztottak.

– Tarts velünk, ha találkozni akarnál vele – mondta búcsúzóul Oleinek doki. –
Megisztok együtt egy-két kortyot. Az óhitűek húsvétja van ma.

– Nem – feleltem –, ma nincs kedvem.

– Talán üzensz neki valamit.

– Nem, pillanatnyilag nincs mit.

Oleinek doki el is indult a folyosón, a szőrmók Connie Illafeld, mint egy hű
állat mindjárt követte. Nyakában himbálózott a bádogból való névjegy, az
udvaron csillogni kezdett, visszfénye a falakon, fatörzseken villózott, még
vadonatúj volt. Mostantól, aki csak ránézett, máris tudhatta, kivel áll
szemben. A keskenyvágányú állomása felé tartottak, ahol fogadott fiam, Béla
Bundasian várakozott a kézihajtánynál.

Engem nemsokára a hullaőrségből is elcsaptak, át kellett engednem a terepet
utódomnak, Toni Tescovinának. Aznap reggel, amikor bevezettem a halottőrzés
apró fortélyaiba, Connie Illafeld, alias Cornelia Illarion testét találtam a
szürke kőasztalon kiterítve. A nyakán, ahonnan valaki képzelem, nem kis
haraggal – névjegyét letépte, sötétkék volt a vér, mint az alvadt áfonyalé,
vagy mondjuk mint a ruszin bojároké, az Illarionoké. Mire a kamrába került,
szutykos, kemény gönceit késsel vagy ollóval levágták róla, ha néha-néha
hozzáért az ember, biza hidegebb volt, mint a 101 kőasztal, amelyen feküdt.
Bundája fényét veszítve, mint valami fekete zúzmara, halkan zizegve pergett le
róla, mígnem műszak végére ott feküdt nekünk teljesen csupaszon.

– Hol szoktál lemosdani? – kérdezte kifelé menet Toni Tescovina. – A térre
készülök. Géza Kökény mondta idejövet, holnap szent húsvét napja. Én meg csupa
szőr vagyok.

– Afene. Kicsit sok lesz a húsvétból – morogtam. – Ami pedig a szőrt illeti,
nem árt, ha megszokják körülötted. Tudják csak meg, szőrös munka ez.
102
10. (Géza Hutira füle)

Abban az évben az év leghidegebb napja tavasz elejére esett. Előtte való
éjszaka Géza Hutira már nem aludt, attól kezdve, hogy a tűz kihunyt, s a
kéményen át beereszkedett a házba a katlan hidege, ő Bebe Tescovinát
melengette. Egy ideig ölében szorongatta, majd, miután föllelhető göncöt,
rongydarabot rárakott, végigfektette magán, és hajával, szakállával is
betakarta. Ha elszenderedett is kis időre, félálomban hallotta a völgy felől a
hó cincogását, ahogy a befagyott, néma patakmeder mentén közeledik valaki. A
léptek nemsokára a küszöb előtt topogtak, s amikor zseblámpájával rávillantott
és fölismerte a zúzmarával lepett, orrából bősz páracsóvákat eregető, jeges
agyarakkal villogó alakot – Nikifor Tescovina volt –, azt hitte, a gyerekéért
jött, hogy hazavigye.

De a kantinos a lányára most ügyet sem vetett, ő Géza Hutirát kereste.

– Kapjon magára valami meleget – mondta –, legyen a zsebében dohány,
rágnivaló. Pár napra eltávozunk.

– Éjjel nemigen járok el itthonról – morogta Géza Hutira. – És még az sem
fordult elő, hogy ne olvastam volna le időben a műszereimet. Mit gondol, ki
regisztrálja nekem a méréseket?

– Jöjjön csak. Tudja jól, a kutyát sem érdeklik a megfigyelései.

– És ha már muszáj, akkor merre, hova?

– Majd megmondják.

Még csak a hó világított, amikor a küszöb előtt hótalpat 103 kötöttek. A
meteorológus háza már az erdőhatár fölött állt, kikapaszkodtak a közeli
hegyvállra, átkeltek a keskeny fennsíkon, túloldalán beereszkedtek a Baba
Rotunda-hágóra. Az útkaparó Andrej házában várt rájuk Coca Mavrodin ezredes a
dobrini hegyivadászoktól.

– Beteglátogatóba készülünk – mondta, amikor két embere megérkezett. –
Körülnézünk egy kicsit a Kolinda-erdőn, ahol a nyugalmazott erdőkerülők
laknak. Úgy hallottam, nincsenek valami jó bőrben. Sőt ami az egészségüket
illeti, csak a legrosszabbakat mondhatom. Lássuk, mit lehet tenni értük.

A Kolinda-erdő alján kicsi, behavazott falu terült el, naptól, ködtől fakó,
szétszórt házai közé a lejtőkről szélfújta hótorlaszok ereszkedtek. A
terepjáró meg-megfarolva, bukdácsolva érkezett egy kicsi fatemplom elé, ahol
az út is hirtelen véget ért. A paplak tornácán fiatal, sápadt férfi,
Pantelimon pópa könyökölt, az udvaron párába burkolózva három fekete ló
várakozott.

A pópa nem megszokott papi köntösét viselte, úgy öltözködött, mint bárki más
errefele vagy éppenséggel a hegyivadászok emberei: hátára terítve fekete műbőr
kabát, alatta durva kötésű fűzöld szvetter, rongyos katonanadrág és mezítlábra
húzott szandál. Meztelen lábujjáról a konyhában sem olvadt le a hó.

– Keveset várnotok kell, az ember még nem érkezett meg. Nem kizárt, elakadt
útközben – mondta. Újságpapírba néhány darab dermedt főtt krumplit, pár fej
vöröshagymát, ráncos héjú almát csomagolt. – Gondoltam a hasatokra, ki tudja,
mikor végeztek.

Átment az udvaron, a hóba taposott lila ösvényen, a nyitott sekrestyeajtó
felé. A lovak fölött sűrű pára gomolygott, benne, mint valami távoli harmónium
hangja, ajtó nyikorgása kavargott, az udvar fölött időnként az egészet magával
ragadta a szél.
104

Az év leghidegebb napja volt, de a konyha ajtaja mégis mindvégig tárva-nyitva
maradt, előtte csak a pára szivárványos függönye lengedezett. A falból a
vakolat pergése, kaparászás hallatszott, a repedésekből hűvös egérszag áradt.
Az asztalt ragacsos viaszosvászon borította, közepére vastag tintaceruzával
volt berajzolva a malomjáték hálója. Coca Mavrodin ezredes köpenye zsebéből
máris elővette a fekete és fehér korongokat, sorban az asztal peremére
készítette őket.

Pantelimon pópa a templomból jövet nyergeket hozott, kettőt a vállán, egyet a
hóban húzott maga után. Átvetette őket a lovak hátán, hasukon meghúzta a
szíjakat. Műbőr kabátja vállán elszórtan madárpiszok világított, akár az
ezredesi csillagok. A két ezredes egy darabig szótlanul, állva malmozott a
konyhaasztal mellett. Az ajtó most is mindvégig nyitva maradt. Künn, a fagyban
topogtak a lovak, közéjük a gőzölgő friss trágyára időnként verebek, varjak
ereszkedtek.

Dél elmúlt, amikor a hótorlaszok között nagy hurrogással motoros szán
érkezett, olyanszerű karcsú, gyors alkalmatosság, amilyet a dobrini
hegyivadászok is használtak. De ezt most nem katona, csak egy vattakabátos,
kucsmás, gumicsizmás ember vezette, és egy üvegektől csikorgó, tömött
átalvetőt hagyott a küszöbön. A szánt mindjárt meg is fordította, és elhajtott
vele.

Az üvegeket sebtében dugaszolhatták, eresztettek egy kicsit, a konyhában
nyomban szétáradt az olcsó rum illata. Pantelimon pópa az italt átmérte két
műanyag tartályba, és közben arra kérte a két férfit, aki segédkezett neki,
hogy még az ujjukat se nyalják meg utána.

A tartályokkal teli átalvető az egyik nyereg kápájára került, mindhárman lóra
ültek. A pópa gyufaszálat rágva nézett utánuk a paplak tornácáról. Hóba
taposott keskeny ösvényen kifelé tartottak a faluból.
105

Coca Mavrodin arra kérte két emberét, lehetőleg egymás nyomában, libasorban
haladjanak, és mindig csak az út jobb oldalán; vágjanak a hóban jól látható
csapást. A Kolinda-erdő alaktalan, lomha hegy volt, hosszan elnyúlva, laposan
nyújtózott el a többi magasabb vonulat között. A Baba Rotunda-hágóról, bár
onnan alig egy-két órányi távolságra feküdt, mindig barnás ködökben elmerülve
úszott a keleti látóhatáron. Tetejét is erdő borította, nem tarkállott rajta
csipkézett szegélyű mezőhavas, oldalában csak néhány irtás, embervágta
szögletes tisztás világított.

A fagy délutánra sem engedett, a bádogszínű ég alatt, a zúzmarás fenyőkön,
mint valami óriási tobozok, dermedt varjak gubbasztottak. A Coca Mavrodin
posztóköpenyére fagyos gyöngyökben csapódott ki a pára, s ha valamelyik ló
szellentett, fenekéből is sistergő, forró gőzcsóva csapott elő.

Az út alig emelkedett, s hogy éppen merre vezet, azt inkább a hó alatt
bugyborékoló patak morgása jelezte, mígnem egy helyen a meder teljesen
kisimult, fölötte hullámzó fenyőlombokkal összecsapott az erdő. Onnan keskeny
vágat vezetett át egy nagyobbacska tisztásra. Körös-körül zengett-bongott az
erdő csarnoka: a Kolinda-erdő tele és tele volt titkos búvópatakokkal.

A tisztás közepén, hótól körbefújva állt a nyugalmazott erdőkerülő menhelye.
Mint egy titokkal tele doboz, ajtaja-ablaka gerendával, deszkával volt
beszögezve. Még a zsindelyfedelet is ácskapoccsal odarögzített nyers fahasábok
borították, nehogy bárkinek is eszébe jusson ajtón-ablakon próbálkozni. Közben
azért laktak is benne: fedelén, a zsindely rései fölött halvány füst hínáros
indái lebegtek. Meg is hallották odabenn, léptek közelednek a ház felé.

– Halló, jár itt valaki? – hangzott egyszerre tompán és öblösen a falak közül,
mint egy csukott dobozból. – Ki az, és mit keres?
106

– Csak mi vagyunk azok – kiáltott fakó hangján kurtán Coca Mavrodin ezredes.

– Jaj de jó. Megismerem a hangjáról a kisasszonyt. Akkor el tetszettek jönni,
kiengedni minket.

– Még nem éppen. Hiszen tudják, van egy kis baj az egészségükkel. Most egy
ideig nem árt az óvatosság. Nincs is valami jó idő, idekünn csak még jobban
megnáthásodnának.

– Azért mégiscsak jó, hogy maguk azok. Jólesik hallani az ismerős hangot.

– Természetesen mi vagyunk azok. Hoztunk egy kis itókát maguknak. Nem is
tudom, likőrt vagy rumot. Mindjárt kitaláljuk, milyen úton-módon juttatjuk be
maguknak. Van velünk egy vascső, az volna a legjobb, ha azon át sikerülne
betöltögetni.

– Likőr vagy rum? Hát azt előre is nagyon szépen köszönjük. Itt mellettem
bökdösi az oldalamat Toni Waldhütter bácsi, és kérdezteti, Jamaika az a rum,
vagy Portorikó, mert ugyebár neki az ilyesmi nem mindegy. Örülök, hogy az
öregnek megjött a hangja, persze csak viccből kérdez ilyesmiket.

– Megértem Toni Waldhütter bácsit, az italomra kényes vagyok magam is. Azt
üzenem neki, mindjárt megkóstolhatja. Velünk van Géza Hutira, leleményes,
ügyes ember, biztos talál majd egy rést a falon, ahol a csövet bedughatja, és
akkor a túlsó végét máris megszophatják. Legjobb, ha keresnek hamar egy
edényt, amit majd a cső vége alá tartanak.

– Köszönjük szépen. Bár való igaz, már az élelmünk is fogytán van kicsit.
Maximum két napra, ha elegendő.

– Megnyugtatásukra mondom, többre most már nem is kell. Elég, ha azt beosztják
maguknak.

– Ó, de jó. Akkor most már megleszünk valahogy.

A tisztás szélén, szemben a beszögezett házzal állt a 107 három lovas.
Körülöttük pára lengedezett, hajuk, szakálluk, borostájuk csupa zúzmara. Coca
Mavrodin fülében a sárga vatták is kifehéredtek.

– Mi akar ez lenni? – súgta oda Géza Hutira Nikifor Tescovinának.

– Na, vajon mi. Találja ki.

– Tényleg, na.

– Hagyjon engem.

– Érdeklődjék csak nyugodtan – szólt közbe Coca Mavrodin. – Én majd elmondom:
egyikük megbetegedett, tudja, az a bizonyos nátha. Ezért most vesztegzár alatt
vannak, elkülönítőben.

– Nem, én nem kérdeztem semmit.

Géza Hutira tenyerébe köpött, hiszen hallhatta, ő lesz az, aki a nyugalmazott
erdőkerülőket mindjárt megitatja. Leszállt a lóról, leakasztotta Coca Mavrodin
nyeregkápájáról az átalvetőt, benne a két tartály rummal; mellettük mindjárt
fölfedezte a szóban forgó vascsövet is. Megkereste a ház falán azt a rést,
amelyen át a bentiek beszéde kihallatszott, azon át bedugta addig, amíg
megérezte, odabenn megfogják, hallatszott, amint vége fémesen koccan az alája
tartott edényen. Akkor lassan töltögetni kezdte az italt. A rum útközben
megfagyott, lassan, mint a méz, ikrás csillongással csordult elő a tartály
száján.

Coca Mavrodin ezalatt, mintha kiránduláson lenne, köpenye zsebéből elővette az
úti elemózsiát. Az újságpapírt a kérges hóra terítette, négy sarkát ellátta
nehezékkel, jégdarabokkal, nehogy a szél váratlanul fölkapja. Körmével
megtépdeste a dermedt krumplikat, majd elkérte Nikifor Tescovina zsebkését, és
fölszeletelte a hagymát. Aztán, mint aki a férfiak javára lemond az adagjáról,
visszaült a nyeregbe, a ló nyakára dőlt, és elbóbiskolni látszott. A tisztás
fényei éppen kihunyóban voltak, az erdőből az alkony közeledett, kelet felől
maga az este.
108

– Ha véletlenségből én is elkapnám ezt a nyavalyát – morogta –, tudják, mit
teszek? Végigjárom a laktanyát, és minden hegyivadásznak beleköpök a szájába.

– Biztos, az volna a legjobb – mondta Nikifor Tescovina –, de én úgy tudom, ha
nem haragszik, olyankor már semmi kedve köpködni az embernek. Pedig a szája
éppen habos nyállal van tele.

– Maguk még mindig nem ismernek engem, hiszen csak tréfálok. De maga ezt a
dolgot a nyállal, honnan tudja?

– A medvész doki mondta. Az ember szája megtelik száraz, sűrű nyállal, olyan,
mint a szivacs. Nem lehet kiköpni.

Géza Hutira a kiürült tartályokat elhajította, hosszan utánuk bámult, ahogy
üresen, könnyen messzire siklanak a havon. Aztán Nikifor Tescovinával együtt a
leterített újságpapír mellé térdelt, s falatozni kezdtek. Esteledett, a
felhőkből áradó színes fények sápadozni kezdtek körös-körül a hó szélverte
hullámain.

– Nézze csak – jegyezte meg az étkezés vége felé Nikifor Tescovina. – Hadd
mutassak magának valamit. Ez a karika hagyma itt pont olyan, mint egy fül.

– Fül? Jópofa.

– Nézze csak meg jobban.

– Tényleg, egy fül. Egy igazi fül, de hát hogy kerül ide?

A rideg krumplihéj, a felvágott hagymaszeletek, a fonnyadt almacikkek között
az újságpapíron egy egész fülkagyló hevert. Kicsit szőrös, kicsit véres,
nemrég törhetett le valahonnan, frissiben.

– Ha nem veszi bizalmaskodásnak, hiszen ez igazán csak a magánvéleményem –
súgta oda Nikifor Tescovina –, azt hiszem, a magáé.

Géza Hutira két kézzel a fejéhez kapott, arra a helyre, ahol behúzott
korcsolyasapkája lefogta volt a fülét. Kitapogatta magát, aztán arca felé
tartotta a tenyerét. Egyik keze 109 száraz maradt, a másik ragacsos volt,
maszatos, kicsit barna.

– Hm. A kutyafáját neki. Biztos nekimentem valaminek. Becsszóra, fogalmam
sincs, hogyan történhetett – nyöszörögte szinte mentegetőzve. – Lehetséges,
azzal a vascsővel, ahogy húztam kifelé a gerendák közül. Mintha egy kicsit
meglökődött volna.

Coca Mavrodin nem aludt, most hirtelen kiegyenesedett a pára csóvái közül,
megköszörülte a torkát, és odakiáltott:

– Most hülyéskednek, vagy tényleg az elvtárs füle az? Mert azt én is
megnézném. Hadd lám csak.

Géza Hutira füle csonkja köré görbítette tenyerét, hogy megértse, mit is kíván
Coca Mavrodin. Kicsit eltöprengeni látszott a hallottakon, aztán miután mégis
megértette, szomorúan megrázta a fejét.

– Sajnos, már nem lehet.

Egy kicsi, avarszínű négylábú állat, akkora, mint egy mókus vagy egy menyét,
lábát a kérges havon sebesen kapkodva, szájában a füllel éppen tovaiszkolt.
Távolabb a párja várta, és nemsokára hallatszott, ahogy fogaik között Géza
Hutira porcogói ropognak.

– Majd kitalálok valamit – mondta jóval később Coca Mavrodin, ereszkedőben a
Kolinda-erdőről –, valamit, amivel kárpótolhatnám; tudtommal a szovjetek már
műfület is csinálnak. Azért, engedelmével, maga is jobban vigyázhatott volna
rá.

– Nem tesz semmit.

Most is libasorban haladtak egymás nyomában, csak ezúttal az út bal oldalán
hagytak maguk mögött jól látható csapást. A két nyomvonal között érintetlen
maradt a hó.

Pantelimon pópa konyhájában este tűz égett, a forró kályhalapon
krumpliszeletek, gombakalapok, héjukban egész 110 almák pirultak. A két ezredes
egy darabig megint állva malmozott, miközben az ajtó egész idő alatt
tárva-nyitva állt. Addig játszottak szótlanul, a korongokat tologatva az
abroszon, amíg a hótorlaszok mögött újra fölbúgott a motoros szán. Most maga
mögött teherárunak való lapos parasztszánt vontatott, benzines-,
gázolajoskannák csörömpöltek rajta. Lehet, ugyanaz az ember vezette, aki a
délben a rumot hozta, de nem lehetett felismerni: vastag, csillogó öltözéket
viselt, fején, mint a tűzoltók, rézsisakot, térdét is beborító magas szárú
csizmát. Le sem szállt a vezetőülésről.

– Ki bírok menni vele végig? – kiáltott oda. Hangja olyan túlvilági volt, mint
Géza Kökényé. Coca Mavrodin és a pópa kimentek elébe a tornácra.

– Abszolute végig. A fényszóró kévéiben csak az árnyékot figyeld. Kétoldalt
csapást hagytunk az úton, ha mindig közöttük haladsz, pontosan odaérsz.

– Most arra kérem magukat – szólt reggel Coca Mavrodin, amikor két emberével
elindult lóháton a Kolinda-erdő felé, –, bármennyire is kellene, útközben ne
pisiljenek. Amíg én nem szólok, hogy tessék, most lehet, kérem, tartsák
mindenképp vissza, elvégre férfiak. Nem kizárt, szükségünk lesz majd egy
kevéske langyos folyadékra.

Géza Hutira tenyeréből kagylót formált füle csonkja köré, hogy hallja, miről
is van szó. Így is Nikifor Tescovinának kellett elmagyarázni, mi éppen Coca
Mavrodin kívánsága. Most előző napi csapásaik között, a motoros szán nyomain
haladtak egészen addig a helyig, ahonnan már csak egy keskeny vágás vezetett
át a tisztás felé. Ott a lovak maguktól megálltak, és többé semmilyen
nógatásnak nem engedelmeskedtek. Csak kantárszáron lehetett őket átvezetni a
tisztásra.
111

A hely az eltelt éjszaka alatt valamelyest megváltozott. Először is a hó nem
fehér volt, hanem szürke, szederjes, néhol egészen fekete, tele kemény, lila
fényektől gerjedő hólyagokkal, vasszürke pikkelyekkel, és fölötte olyan szag
terjengett a fagyos levegőben, mint amilyen az elhagyott tűzhelyek, kidobott
kályhacsövek körül szokott. Mintha errefele egész éjszakán át csak pernyét
havazott volna.

Az elmúlt éjszaka eltűnt helyéről az a beszögezett boronaépület, amelyben a
nyugalmazott erdőkerülők laktak. A tisztás közepén néhány elsenyvedt, görcsös
fekete gerenda között most a szél érintésétől bársonyosan hamu és korom
hullámzott. A hó körülötte – egyszer megolvadt, majd pernyével tele keményre
fagyott – a felhők behulló fényében márványosan derengett. A magasban, mint a
tűzvész ott rekedt füstje, egy sereg csóka körözött. A havon mindenféle
eldobált benzines-, olajoskannák hevertek.

A fölkavart koromtól prüsszögni kezdett egyik-másik ló, Coca Mavrodin sálat
tekert az állára, körbekocogta az üszköket, majd lovát hirtelen megnógatva,
átgázolt a romokon. A paták alatt csilingelt a gerendákból kihullott sok
kampó, ácskapocs és szeg, csörömpöltek a nyugalmazott erdőkerülők eszközei,
bádogedényei. Az ezredesnő a tisztás pereméig lovagolt, ott várakozón megállt,
hogy két embere beérje.

– Jöjjenek csak bátran – kiáltott oda nekik. – Mind megsültek a bacilusok.

– Mit mond? – Géza Hutira fölkapta a fejét, Nikifor Tescovina tekintetét
kereste, várta, találkozzék egy villanásnyira a pillantásuk. De a másik
megérezte szándékát, és félrenézett.

– Hagyjon ki ebből – szólt kicsivel később figyelmeztetőn. – Nincs véleményem
a dologról.

– Csak gondoltam, észreveszi, mibe keveredett.
112

– Na, mibe! Igazán nem tudom, mire gondol. Mind a ketten Coca kisasszonynak
dolgozunk.

Eközben maguk is szinte észrevétlenül átgázoltak az üszkökön, lassan kocogva
elérkeztek a tisztás peremére. A lovak, mintha forró lenne, sebesen kapkodták
a lábukat a hóból.

– Harapjunk valamit – ajánlotta Coca Mavrodin. – Ma nem is akármi várja
magukat, ebédre konzervet hoztam: hagymás pontyot árpakásával. Aztán, ha
jóllaktak, szeretném, ha előkeresnék a dögcédulákat. Személyenként hármat
kérek: egyet a nyakukban viseltek, mint maguk, egyet-egyet még külön a csuklón
s a bokájukon. Megköszönném, ha sikerülne valamennyit betakarítani.

A konzervet köpenye zsebében hozta, meg sem kellett bontani, mert a folyadék
útközben megfagyott, és a csavaros üveget szétvetette. A cserepek, szilánkok
közül kibontotta a fagyott árpakásahengert, amelyből sötétkék uszonyok
meredeztek, körmével tépdeste, darabokra, hogy tenyeréből csipegetni lehessen.

Körös-körül a kérges havon mindenfelé pörkölt tollú madarak – varjak, csókák,
rigók – hevertek. A tűz ébresztette őket, és biztos, még a levegőben
megsültek, de a meleg sokáig a magasban tartotta őket, s csak akkor hullottak
le, jó messzire becsapódva, amikor a tisztás kihűlt alattuk.

Amint az evést befejezték, Nikifor Tescovina fenyőgallyat, nyírfavesszőt
nyesett, Géza Hutira vastagabb mogyorórudakat tördelt. Előbb botokkal
tapogatták végig, mint valami kincskeresők, a terepet, majd a hamarjában
zsineggel körbetekert vesszőkötegekkel kiseperték a hamut a gerendák, földi
maradványok közül.

– Erre mondják azt, hogy fekete munka – morogta Géza Hutira. – Kár, hogy nem
hordok magamnál tükröt. Látná csak, hogy néz ki.

– Most megint mi baja? Úgy veszem észre, a lányom 113 kimondottan rossz
hatással van magára. Tartsa meg magának a bölcsességeit.

Géza Hutirának azon az oldalán, amerről Nikifor Tescovina beszélt hozzá, nem
volt füle, aligha hallotta, miket mondanak neki. Fejét jobbra-balra
tekergette, zavartan nézelődött, keresgélt az üszkök között.

Végül tizenkét darab láncon fityegő, korommal vastagon bevont bádoglapocskát
találtak. Akkor végre Coca Mavrodin megengedte, hogy vizeljenek. A langyos,
sós oldat, mondta, feloldja a bádogra tapadt fekete taftot, aztán már csak le
kell dörzsölni a kérges havon, és a belevésett adatok bárki számára
kiolvashatók lesznek.

Az üszök és korom alól ugyanannak a négy embernek a dögcédulája került elő,
négyüknek mindhárom névjegye. Csakhogy a nyugalmazott erdőkerülők öten laktak
az erdei menhelyen. Aron Wargotzki kiégett porhüvelye és a dögcédulája
hiányzott.

Hogy később ezt az Aron Wargotzkit nekem kellett előkerítenem, annak Géza
Hutira elveszett füle volt az oka. A három utas hazatérőben az utászházamban
újra megpihent, és mialatt Nikifor Tescovina és Géza Hutira mosakodott –
Elvira Spiridon öntözőkannából locsolta őket, ők meg, mint fáradt lovak az
esőben, egymás vállára ejtett fővel áztak –, Coca Mavrodin tanakodni a
tornácra hívott.

– A fennvaló elvette tőle a fülét, márpedig nekem egy jól halló emberre lenne
szükségem – mondta. – És őrajta kívül senki nem ismeri úgy ezeket az erdőket,
mint maga.

– Az már nem az én területem – vonakodtam. – A kisasszony tudja jól, a
Kolinda-erdő már nem az enyém, életemben nem jártam ott.

– De én ezt most magától kérem, Andrej. És most már csak ezt, semmi egyebet.
Akkor szemet hunyok a dolgai 114 fölött. Találja meg nekem ezt a fertőző
beteget – jegyezze meg jól: Aron Wargotzki a neve –, és akkor megígérem,
fogadott fiával együtt elmehet innen.

Attól fogva reggelente a küszöb előtt fölcsatoltam a sítalpakat, kedvesemet,
Elvira Spiridont magam elé állítottam, derekát fél kézzel átkulcsoltam, és
útban a Kolinda-erdő felé hazáig siklottam vele. Severin Spiridon, a férje,
ilyenkor a kapuban várakozott; ő figyelmeztetett, a tizedik nap után legyek
résen nagyon, addig az ember valahogy eltengeti magát száraz tobozon, sótlan
jégcsapon egy nyirkos odúban, de akkor vége: négykézláb is elindul feladni
magát. És útközben tenyerével, térdével nyomokat hagy maga után a havon.

Coca Mavrodin pedig ezzel búcsúztatott:

– Tudja, mire figyeljen, Andrej, nagyon? A szarra.

Nem butaságot beszélt, hiszen a tapasztalt erdőjáró ember tudja, az a bizonyos
szar, amire Coca Mavrodin gondolt, ha eltakarja is mondjuk, az éjszakai
havazás, reggel a fehér takarón át a nap melegét magába szippantva leveti
magáról a hazug álcát, és újra ott pompázik nemes barna mázával.

De Aron Wargotzkinak sem lábnyomát, sem tenyere, sem térde nyomát, sem
hátrahagyott piszkát nem találtam sehol a Kolinda-erdőn. Végül nem gyarló
emberi szüksége, hanem ostoba, fényűző kedvtelése volt az, ami elárulta.

Egy délután – akkor talán már a második hete kerestem – hazatérés előtt a
nyugalmazott erdőkerülők újra meg újra behavazott hófehér tisztásán pihentem,
a búvópatak lankasztó mormolását hallgattam, amint rövid történeteket
ismételgetve bolyongott a föld, a hó, a jég alatt, amikor orromat megérintette
a perzselt kakukkfű semmi egyébbel össze nem téveszthető illata. Ilyet szívott
mellszobra árnyékában Géza Kökény, ezt füstölték a medvészek is, ha
115 dohányuk elfogyott, néha maguk az ezredesek is. Megkóstoltam nemegyszer
magam is.

A fenyők mögül betűző napfény lemezein, a szélcsendes percekben a pipafüst kék
nyelvei lebegtek. Előttem a havon a rejtőző patak útját árnyas horpadás
jelezte, végében, néhány csupasz, nyirkos kő mögött fekete üreg tátongott.
Onnan kunkorodtak elő időről időre a füst vékony fonalai. Amíg én a havas
erdőn – hogy Coca Mavrodin szavával éljek – egy darab szart kerestem, Aron
Wargotzki ült a föld alatt, és pipázott.

– Aron Wargotzki – szólítottam meg. – Ígérd meg ünnepélyesen, nem mozdulsz
onnét. Nagyon kérlek, nyugton maradj. Repülni, remélem, nem tudsz, akkor meg
bárhova is mennél, úgyis elárulnának a nyomok.

Aron Wargotzki még jó ideig eregette a füstöt, és csak estefelé válaszolt,
amikor látta, válasza nélkül a közeléből nem tágítok.

– Jó, megígérem. De csak azért, mert amúgy sem bírok mozdulni. A fél lábam
leégett.

– Nagyon helyes. Maradj veszteg, kíméld magad. És hogy ne érezd magad
magányosan, hamarosan, legkésőbb holnap reggel visszatérek.

Fölcsatoltam a sítalpakat, és már majdnem az egész tisztáson végigsiklottam,
amikor hangja utolért, egyszerre zümmögött tőle körös-körül az erdő, búgtak a
mélyben a búvópatak járatai.

– Géza kolléga, először is meg szeretnélek kérni valamire.

– Összetévesztesz valakivel, én nem ő vagyok. De mondd, mi az óhajod.

– Küldd ide nekem Géza Hutirát, ha ismered. Beszélnem kell vele.

– Nem hinném, hogy mostanság ráér. Mégis, ha találkoznánk, mit mondjak neki?
116

– Azt, hogy Aron Wargotzki sürgősen kéreti. Siessen, és hozzon magával egy
bögre meleg tejet.

– Oké. Meglehet, összefutok vele. Ha nem felejtem el, megmondom neki.

– És én akkor most kivel beszéltem?

– Ugyan, Aron Wargotzki, gondolhatod, mondhatok neked bármilyen nevet. Igazán
nem fontos ez.

Sítalpaim nyoma napról napra mélyült a havon, a végén, ha dolgom végeztével
estefele beleálltam a vályúba, az magától hazavezetett a hágóra, ahol Elvira
Spiridon várt rám redős homlokkal, felhős tekintettel.

Annál inkább örült nekem Coca Mavrodin, amikor az erdőbiztos irodán
jelentkeztem. Patkányfogót mutatott, erős rugóval működő gyilkos csapóval,
amit a jó hírre várva fiókjában készenlétben tartogatott. Fog belőlük
készíttetni, mondta, ötöt vagy hatot, persze hatalmas nagyokat. Majd
odatelepítjük őket a búvópatak nyílásai közelébe, arra az esetre, ha Aron
Wargotzki mégis fölépülne és lábra kapna.

De Aron Wargotzki, ahogy megígérte, nem mozdult, a hó a búvópatak
lélegzőnyílásai körül nap mint nap érintetlen maradt, csak egy avarszínű,
karcsú kicsi állat suhant el arrafele néha, biztos az, amelyik Géza Hutira
fülét megette. Hol a fanyarkás füst illata kunkorodott elő az üregekből, hol
az erdei ember sokéves gyantaszaga terjengett a rések fölött.

– Ide figyelj, Aron Wargotzki – szólítottam meg. – Beszéltem Géza Hutirával,
elfoglalt ember, nem ér rá, sajnos. Velem kell beérned. Mondd, mi a panaszod,
talán segíthetek valamiben.

– Csakis abban, hogy ráveszed, mielőbb jöjjön ide. Négyszemközt akarok vele
beszélni. És addig is, amíg lesz ráérő ideje, küldjön veled egy kancsó meleg
tejet.

– Az álmok világában élsz, Aron Wargotzki. Mondd csak, ezt az egész tejügyet
honnan veszed? Továbbá arról van-e tudomásod, hogy még a nyáladat sem bírod
lenyelni? Tele a szád kemény, száraz habbal. Dögrováson vagy.
117

– Én?

– Abbiza. Beteg vagy sajnos, nagyon.

– Én? Azt mondod? Nincs nekem semmi bajom. Csak teleettem magam földdel, és
most nyomtatékul egy kis tejet kívánok.

– Ejnye, Aron Wargotzki, ne bolondozz. Engedd meg, hogy biztos forrásból
tudjam, mi is a bajod. Már csak azért is, kérve kérlek, saját érdekedben
maradj nyugton.

– Mintha már mondtam volna, mozdulni sem bírok, akkor hová mennék. A jobb
lábamról, vagy ha szemből nézed, a balról, leégett a hús. Pont az a darabka
kurva comb hiányzik, ami mozgatta.

– Akkor egyről beszélünk. Nyugton maradj, most már bírd ki tej nélkül azt a
pár napot.

Addig is, amíg a csapóvasak készültek, a napot kora reggeltől napszálltáig a
tisztáson töltöttem. Reggel sítalpammal a nyomokba álltam, s azok egyenesen
Aron Wargotzkihoz vezettek. Néha a búvópatak duruzsolását túlkiabálva,
hosszasan szólongatnom kellett, míg hajlandó volt tudomásul venni, hogy
látogatója érkezett, máskor, mint egy kutya, a vágyakozástól lihegve várt a
nyílás mögött.

– Mesélj – kérlelt. – Mit szólt Géza kolléga, hogy itt vagyok?

– Semmi különöset, Aron Wargotzki. Senki nem szól a dologhoz semmit. Ez a
dolgok rendje.

– Aztán, amikor a tejet hozod, ügyelj ám, ki ne lötyögtesd. Majd én
irányítalak, hova, merre töltsed. Azt sem bánom, ha a szádban hozod, és
beköpöd nekem itt, valamelyik lyukon. Csak tej legyen.

– Ejnye, csak nem képzeled, hogy olyan közel megyek? Azt igazán nem
kívánhatod, hogy azt a nyavalyát elkapjam tőled.

– Már mondtam, semmi bajom nincsen. Csak a lábam 118 rossz, meg talán kicsit
túl sok földet ettem. Jó volna most már szájat öblíteni egy kevéske tejjel.

– Földet ne egyél, arra vigyázz. Még valami egyebet is fölszedsz.

– És tudod, még az is bánt: őszintén sajnálom a Géza Hutira fülét. Magam sem
tudom, pontosan hogyan történt. Komisz, rossz íze volt annak a bizonyos
rumnak, és én mérgemben nekilöktem a vascsövet, amelyen át betöltötte volt
nekünk. Láttam is mindjárt, szegénynek letörött, már csak egy kis cérna bőr
tartja. Szeretnék tőle bocsánatot kérni, egy fül nem akármi.

– Jó, Aron Wargotzki, majd tudatom vele, hogy megbántad. Elnéző, nagyvonalú
ember, meg fog bocsátani neked.

Azon a napon, amikor úgy volt, a hegyivadászok furgonja az utászházig
szállítja a csapóvasakat, a kapu előtt, az alkony szeleiben, fényeiben Elvira
Spiridon haja, ruhája szegélye lobogott. Közelében, a kerítés mentén,
sátorlappal letakarva, hogy valami kóbor eső el ne áztassa, ötven-hatvan
cementeszsák hevert sorban lerakva. Úgy tetszett, Coca Mavrodin-Mahmudia
megváltoztatta elhatározását.

– Ha ezt is megteszi az úr – hajolt közel Elvira Spiridon, hogy lehelete
szagából a fenyegetést is megéreztem, – azt szeretném, ha az alatt az idő
alatt nem találkoznánk.

– Rendben, amint parancsolod. Menj, el vagy engedve. Éld az életedet.

A néhány nap alatt, amikor hátamon cementeszsákokkal naponta többször is
megfordultam az utászház és a Kolinda-erdő között, tavaszodni kezdett. A
napsütötte kupacokon az olvadó hó hártyái alól előbukkant a sápadt fű,
nyomában a sáfrány, s a növekvő zöld foltokon át a sítalpak nyomai vezettek
márványfehéren. A nyílások fölött, amelyeken 119 keresztül a búvópatak
lélegzett, a kakukkfű kék lidércfényei lobogtak a napsütésben.

Amikor valamennyi zsák cement ott állt lerakva a patak üregei körül, fűszálat
dobtam a víz tükrére, hogy mozgását meglessem. Elővettem már előre kifent
késemet, miközben föltűrtem zubbonyom ujját, a pengén megvillanó napfény az
odúk sötétjét is megjárta. Akkor utoljára Aron Wargotzki megszólított:

– Azt hiszed, nem tudom, mire készülsz? Azért is, most már tudni szeretném a
neved. Mondd meg végre, ki az ördög vagy?

– Aron Wargotzki, úgy érzem, nem ez a megfelelő időpont a bemutatkozásra.
Annyit azért elárulhatok, Andrej Bodor álnéven éltem a körzetben. Kérlek, egy
ilyen nevű embernek nézd el ezt az egészet.

Coca Mavrodin-Mahmudia kissé elszámította magát a cementtel: a zsákok fele még
érintetlenül hevert, amikor az üregekben a víz mozgása máris lassult,
megszürkült, mint a savó, színén elültek a buborékok, és annak jeléül, hogy
kezd megkötni az oldat, és dugulnak el a kürtők, a föld alól a patak egyszerre
több helyen kibuggyant a rétre.

A tisztás peremén, azon a helyen, ahol levetett sítalpaimat hagytam, Elvira
Spiridon állt, szélben lobogó új tavaszi ruhájában, frissen mosott, száradó
tavaszi hajával, a nap vakított hatalmas rézkarika fülbevalóiról.

– Ugye, hogy előkerültél – mondtam lihegve, amint a közelébe értem.

– A mai napon kezdett hiányozni nekem az úr.

Szó, ami szó, most már ő is nekem. Szokás szerint magam elé állítottam a
sítalpakra, s amint kétoldalt elindult mellettünk az erdő, s egyre gyorsabban
siklott hátra, a nyugalmazott erdőkerülők tisztása felé, körömmel és foggal
téptem le róla az új tavaszi ruhát, saját derekamról késsel nyisziteltem le a
cementtől páncélos nadrágot, míg végre bársony fenekét újra az ölemben
éreztem.
120
11. (Severin Spiridon meglepetése)

Az az ütött-kopott kicsi autóbusz, amely három hegyvonulaton átkelve naponta
egyszer fordult Sinistra és a Kolinda-erdő között, főként járőröket, szemléző
hegyivadászokat szállított, és azt a néhány civilt, aki engedéllyel a zsebében
valamelyik magaslati településen dolgozott. Így a busz, amikor utasa volt, a
Baba Rotunda-hágón is megállt, a helyet rozsdaette, festékfoltos vaspózna
jelezte az út mentén. A hágón, a talajt is érintve, egyre-másra átgördült
egy-egy felhő, a póznán az irányjelző tábla mindig vízcseppektől rakottan
himbálózott a szélben. Nyikorgása a csukott ablakon is behallatszott az
utászházba, ahol az útkaparó Andrej Bodor lakott.

Egy délután, már jóval később, hogy az autóbusz a Kolinda-erdőn térülve
elhajtott Sinistra felé, egy férfi haladt át a hófoltos, sáfránytól,
kikericstől tarka tisztásokon. Furcsán, oldalazva járt, mint egy félszeg
kutya, a feketén csillogó olvadékvizeket kerülgetve, jobbra-balra pislantva
baktatott át a réten. Az utat elérve tétován megállt, inkább csak kihajolt
fölébe, mintha attól tartana, hogy sodrása valamelyik irányba elragadja. Egy
ideig tanácstalanul ácsorgott, mígnem fölfigyelt a megállót jelző tábla
nyikorgására. Tövébe telepedett, mint egy vándor, aki a menetrend szerinti
járatra vár.

Fekete műbőr zekét viselt, fényesre piszkolódott nadrágot, fején fekete
ellenzős bányászsisakot. A vállán átvetett görbe vándorbotról kerekre tömött,
fekete aktatáska lógott. Bőre szürke volt, arca csupasz és fényes, inkább csak
álla 121 körül motozott valami gyér, ritkás borosta. Árnyékos, lila üregek
mélyén csillogott olajos szeme.

Az útkaparó Andrej az utászház ablaka mögül lesett kifelé, megfigyelőhelyéről
távcsövön is szemügyre vette az idegent. Nyolcszor harmincas messzelátója,
amelyet Coca Mavrodintól kapott, mindig készenlétben, az ablak kilincsén
lógott. Fonalkeresztjén most a szürke idegen vergődött.

A bányászsisakos férfi időnként fölkelt a pózna mellől, hallgatózott,
tekintetét újra meg újra gyanakvón hordozta körül a táj fölött, néha idegesen
az elhúzó varjak után kapta a fejét. Mérgesen meredt a délutáni napba is,
amely bágyadt sárga fénnyel tűzött vissza a keskeny, pióca alakú felhők közül.
Szeme sarkából az útkaparóházat is méregette, mintha csak attól tartana,
ablaka mögül figyelik éppen.

Andrej figyelte is. Az előző éjszakát átvirrasztotta egy halott medvész
mellett – bár a halottkémségből felfüggesztették, besegíteni azért gyakran
fölkérték –, s reggel, amikor Titus Tomoioaga ezredes felváltotta, megittak
együtt egy kis üveg vízzel hígított denaturált szeszt. Az ezredes újságolta,
Sinistrán életbe lépett a kijárási tilalom. Talán hamarosan itt, Dobrin
Cityben is bevezetik – az éjjel valaki ledöntötte Géza Kökény szobrát –,
legjobb, ha máris mindenki otthon marad. Attól fogva, hogy sinistrai
bábjátékosok főpróbát tartani az utcára vonultak, és a hegyivadászok közébük
lövettek, járőr cirkált a falusi utcákon is. Bármerre nézett az ember, hosszú
nyakú, gúnárképű fiatalemberek lestek át a kerítések fölött. A kapukon,
palánkokon szénnel firkált feliratok sötétlettek, ilyenek; „velünk vagytok”,
vagy „téged is vár a liga”. Egy deszkába forró vassal egyszerűen ennyi volt
beleégetve: „disznók”.

Március vége felé járt, a levegő tele nyugtalanító illatokkal, barkaporral,
cikázó legyekkel. A völgy alján, medrében olvadékvizektől zavarosan kavargott
a patak. Andrej félú 122 ton járt a hágó felé menet a szerpentineken, amikor a
Kolinda-erdő felől szembejött vele a délutáni autóbusz. Valamennyi ablaka ki
volt törve, nem hegyivadászok, hanem szürke bőrű, bányászsisakos emberek ültek
benne. Fullasztó, nehéz szag maradt utánuk.

Az utászházban első poharát hörpölgette, amikor a távoli hófoltok előtt
elvillanva az idegen föltűnt a tisztás végében. Bányászsisakja nemsokára az út
közelében villant, amikor odatelepedett a pózna tövébe. Erre-arra kémlelt,
többnyire a tisztás egyik hajlata felé, ahol néhány fenyő mögött Severin
Spiridon házfedele füstölgött, aztán a marton elsétáló kóbor kutyát figyelte,
megint gyanakvón pislantott az utászházra, amelynek ablaka mögül Andrej, az
útkaparó is leste. Valahányszor fölrezzent, reccsent rajta a kemény műanyag
kabát.

Telvén az idő, lila párákkal az alkony reménytelen csendje ereszkedett a
tisztásokra, az idegen megelégelte a várakozást, fölkelt a pózna mellől, és az
enyhén emelkedő ösvényen az utászház felé indult. Egy tünedező felhő
visszfényében sisakja éle villózott. Fölment a lépcsőkön, és tenyerével éppen
beárnyékolni készült tekintetét, hogy az üvegre tapadva bepillantson, amikor
Andrej kinyitotta előtte az ajtót.

Pont olyan volt, ahogyan távcsövön át már megismerte: fényes és szürke bőrű,
borotválatlan, de olyan csupasz fajta, akinek csak állán nő tétova, ritkás
borosta. A szaga meghökkentő, fojtó, mint a várótermeknek.

– Mit tudsz a buszról? – érdeklődött halkan. – Ugyan bizony miért nem jön?

– Mivelhogy már eltávozott – felelte az útkaparó.

– Úgy. És a következő?

– Az majd csak holnap délután.

Az idegen úgy lépett át a küszöbön, hogy közben Andrej mellét is súrolta,
körbejárta a ház közepén álló asztalt, 123 aztán maga csukta be az ajtót, és a
zárban megfordította a kulcsot. Aktatáskája, melyet a botja végéről a priccs
lába mellé ejtett, súlyosan puffant a padlón, mintha kővel lenne tele.

– Akkor itt alszom – mondta. Megoldotta zubbonyát, akkor szaga, mint olaj a
víz színén, egy pillanat alatt szertefutott a házban. Leült az asztal mellé,
csak úgy recsegett-ropogott rajta a kemény, repedezett műbőr kabát. Ahogy
megnyílt a hasa fölött, látszott, nadrágját derékszíj helyett vastag drót
tartja össze, a csat helyén kövér hurok, belefoglalva éles kődarab. Belső
zsebéből előkerült egy italosüveg.

– Kérsz? – pillantott kurtán az útkaparóra, miközben kidugaszolta.

– Később talán megkóstolom – hárította el Andrej, és csuprot tolt elébe.

De az idegen egyenesen az üvegből ivott, szájából buborékok törtek az ital
belsejébe. Az első pár korty után levetette magáról a műbőrkabátot, sisakját
az asztalra helyezte. Fényes, szürke fejbőrére izzadt, vékony szálú haj
tapadt.

– Pihenj le máris – mondta neki Andrej –, itt sokáig nem maradhatsz. Estéimet
nem egyedül töltöm, egy asszonyt várok ide.

– Azt mondtam, itt maradok.

Andrej megtömte a kályhát fűrészporral, fenyőtobozzal, kemény borókagyökérrel,
és alágyújtott. A tornácról nedves tuskókat hozott be szikkadni. Az idegen
eltávolodott, átült a székről a priccs szélére.

– Miattam igazán ne tüzelj. Apám jegesember volt, kijártunk jégtömböt
fűrészelni a Kolinda-erdőre a jégbarlangokba. Abbiza, az öregem szalma közé
kötözve saját hátán cipelte a hideg hasábokat a piacra, a gazdagoknak. A mi
családunk nem fázós fajta.

– Először hallok a kolindai jégbarlangokról.
124

– Ezentúl talán már nem is fogsz. Úgy hallom, mindenféle népség búvóhelyül
használta, úgyhogy betömték őket. Mind be lettek öntve cementtel.

Lábát kinyújtotta az asztal alatt, karján föltűrte fűzöld szvettere ujját,
alatta ingét is. Kék erekkel befutott, szőrtelen, szürke karja volt. Nyaka
vékony, inas, álla tompa, szeme olajos, mint a bodzabogyó.

– És a papírjaid rendben vannak? – kérdezte az útkaparó.

– Az enyémek?

– Ha nem tudnád, ez határövezet. Azonkívül a közelben kormányzati medvéket
tartanak.

– A papírjaim! Ó, nagyon is, tudd meg. A legnagyobb rendben. Még hogy az én
papírjaim. Nos, azok felől egyszer s mindenkorra megnyugodhatsz.

– Jó, akkor hát pihend ki magad. Ha valamikor éjszaka közepén elindulsz,
reggel eléred Dobrinban a hajnali vonatot.

– Magam is így képzeltem.

Az útkaparó leakasztotta az ablak kilincséről a távcsövet, körbehordozta az
olvadékvizek alatt ázó tisztások fölött. Esteledett, az erdőből már
ereszkedett a rétre a szürkület. Ilyenkor szokott elindulni otthonról Elvira
Spiridon.

– Belenézhetek én is?

– Tessék, parancsolj. De te nem látsz benne semmi érdekeset.

– Csak azt akarom látni, amit te.

– Fogjad, kukucskálj vele. És mondd, persze, csak ha nem tartozik a külön
bejáratú titkaid közé: merre tartasz?

Az idegen a messzelátóval előbb végigvándorolt a tisztásokon, majd a
hegyvonulatokon, végül arra a gyöngyházasan derengő felhőre emelte, amely már
a hold fényében várakozott a hegygerinc fölött.

– Hogy merre? Reggelre Sinistrán kell lennem, a piacon. Lesz ott valami.
125

– Ugyanis nem láttalak még errefelé.

– Ugyanis? Igazad van, ugyanis. Én ugyanis, hogy szavaddal éljek, nem
idevalósi vagyok. Apámmal jeget fűrészelni csak a Kolinda-erdőig
merészkedtünk. Korábban, tudod ott vonult a határ. Ismeretlen vagyok errefelé.
Rövidíteni akartam, de eltévedtem. Ezalatt elment az a szemét busz.

Alkonyodott, Elvira Spiridon elindult otthonról. Bár a távcsövet még az idegen
tartotta kezében, látszott, merre jár az asszony, az előtte felröppenő
madarakból. Amikor léptei már a közelben neszeltek a sárban, Andrej lement
elébe az útra.

– Egy ember van nálam – mondta.

– A fekete táskás, ugye?

– Az.

Elvira Spiridon sarkon fordult, elindult vissza, a ház felé, ahol
tulajdonképpen lakott. Az útkaparó csak vágyakozva tekintett utána, ahogy
feneke dombján ide-oda ringanak a szoknya redői. Szemmel tartotta, amíg el nem
tűnt a fenyőkkel tűzdelt hajlatban. Este volt, a felhők mind eltakarodtak az
égről, a hágóra az éjszaka hidege ereszkedett, neszelni kezdett a sár a kövek
között, ahogy a hirtelen érkezett fagyban megkeményedett.

– Figyeltem a jelenetet – mosolyodott el az idegen. – Kár volt elküldened.
Oszt hova ment ilyenkor este szegényke.

– A férjéhez.

Az idegen pálinkája keserű volt, miközben ivott, orrából pára csapott elő, a
tűz fényében delejes fénnyel villózott. Az útkaparó pléhtányért helyezett
elébe, egy bádogcsuporba kevés vizet.

– Ha megéheztél, majd a magadéból egyél. A szagodon érzem, valami sajtféleség
van nálad. Felétek mostanság kapni ilyesmit?

– Ó, nem. Sajtot azok kaptak, akik ma útra keltek. 126 Bizony jó lett volna még
ma este megérkezni Sinistrára. Mondtam, holnap lesz ott valami.

– Pihend ki magad, kelj útra föltétlenül még virradat előtt. És mondd, a
hadsereggel tartasz?

– Hogy a sereggel-e? Azt is megtudom. A liga majd megmondja holnap, kivel
tartunk.

Derekán megoldotta a nadrágot, s akkor a drót végén, a hurokból az éles kő a
padlóra hullott és elgurult. Az útkaparó csípőfogót vett elő, s miközben
segített az idegennek visszailleszteni a követ a foglalatba, látta, a drót
többszörösen tekeri körbe a derekát. Az csak tűrte, matassanak rajta.

– Oszt mivel foglalkozol?

– A hegyivadászoknak dolgozom – mondta Andrej. – Például én ügyelem fel az
országútnak ezen szakaszát.

– Hallod-e! Azt, hogy pont ezen szakaszát, azt magam is gondoltam. Hű,
sejtettem, fontos ember vagy.

– Ahogy mondod. Tudod, ez itt a közelben csupa tilalmas terület. Itt minden a
hegyivadászoké.

– Ó, persze. Oszt ki a hegyivadászok parancsnoka?

– Coca Mavrodin a neve.

– Csak nem egy nő?

– Eltaláltad.

– Mert akkor nem kizárt, még holnap, mihelyst hatalomra kerültünk – ahogy az
ilyesmit te mondanád –, közösülök egyet vele.

– Ha meglát – mérte végig Andrej a vendéget –, a dolgot majd ő is nagyon
akarni fogja.

Az útkaparó megnyesegette a viharlámpás kanócát, és miután meggyújtotta,
hosszú karó végére tűzte, hogy azzal akassza ki a ház ormára. Ezen majdnem
összeszólalkoztak: az idegen azt szerette volna, ha a ház azon az estén
mindenképp jeltelen marad. De Andrej leakasztotta a falról az útkaparók
szolgálati szabályzatát, és az orra elé dugta.
127

– Jól van, na. Csak nem képzeled, hogy most elkezdek olvasni – hárította el
nyitott tenyérrel az idegen. – Vidd innét. De azt sem szeretném, ha valami
jöttment a pislákolás után pont ide találna. Persze, nem az asszonykára
gondolok.

– Később lehet, majd érte megyek. Megvárom, amíg elalszol, aztán lehozom, hogy
mire távozol, a közelemben találjam a meleg fenekét. Addig is elnyújtózhatsz a
priccsemen.

– Azt már nem. Meg ne próbáld ezt a viskót elhagyni. Vésd az eszedbe,
mostantól fogva itt mindenféle járkálás megszűnt.

– Pisilni ki szoktam állni az ajtóba.

– Majd oda elkísérlek. Még a nyakamra hozol valami jöttmentet. Elvégre nem
tudhatom, ki vagy.

Az ablakon át figyelte, amint az útkaparó kitűzi a ház ormára a viharlámpást,
és körbejárja a falakat. Már a hegygerinc fölött járt a hold, a sár
megfagyott, az út túloldalán a kóbor kutya léptei kopogtak a marton. Amikor
Andrej visszatért a házba, az idegen éppen a régi falinaptárt böngészte. Sok
év előtti volt, légyjárta, sárgán kunkorodó sarkokkal, még az előző útkaparó,
Zoltán Marmorstein idejéből.

– Ez mi akar lenni? – kérdezte az idegen. – Mondd csak, miféle számok itt
ezek?

– Csak az év napjait mutatják.

– Talán magyar vagy?

– Félig.

– Hm. Az semmi.

Elheveredett a priccsen, gumicsizmás lábát a támlára helyezte. Hanyatt fekve
is kényelmesen ivott, csak ádámcsutkája ugrált vadul, miközben az ital az
üvegben kövér buborékoktól fortyogott. A tűz lassan összeroskadt, elnémult, s
a kihűlő kályhacső pattogásai ellankasztották az idegent. Feje oldalra
bicsaklott, szája félig kinyílt, elindult 128 a nyála, és csillogó csíkot húzva
csorgott a vállára. A liga embere elaludt.

Aktatáskájában csak úgy magukban mocorogtak az éles kövek.

Az útkaparó Andrej lábujjhegyen kiosont a házból, leakasztotta az oromról a
viharlámpást, és óvatosan, hogy a hártyásra fagyott tócsákat ne nagyon
ropogtassa talpa alatt, átvágott az úton. A túlsó marton a kóbor kutya árnya
imbolygott, szeme néha összevillant a lámpással. Egy darabig az útkaparó
nyomába szegődött, de már félúton megérezte Severin Spiridon kutyáját a
sötétben, és előreiramodott. Mire az útkaparó a kapu elé ért, a két kutya már
némán barátkozott. Severin Spiridon a ház oldalát támasztva, az eresz alatt
gubbasztott.

– Többször is eszembe jutottál – mondta Andrejnak. – Szívesen éjszakáznék az
utászházban, a helyedben. Le nem hunytam a szemem, egyre csak rád gondoltam.

– Az illető alszik.

– Mondom, már mindjárt az elején is lemehettem volna. Csak hirtelen nem jutott
eszembe. Tudd meg, nem félek ezektől. És teérted egyébként is bármit
megteszek.

– Kösz.

– Ha most nálam akarnál maradni, még mindig lemehetek. Biztos szót értek vele.
Sejtem én, kiféle, miféle. Holnap alakul Sinistrán a liga.

– Nem bánom, menj, ha akarsz. De egyre vigyázz: táskája éles kővel tele. A
legélesebbet derékszíján, a csat helyén, drótba foglalva hordja.

– Bízd csak rám. Mondom, szót értek én ezekkel.

Severin Spiridon elment, Andrej a lépcsőn állva nézett utána. Mellette
hangtalanul fölbukkant Elvira Spiridon, csípőjük máris összeért, arcuk előtt
összekeveredett a pára, amíg várták, hogy a zseblámpa fénye eltünedezzék a
deres rét végében.
129

– Már az övé voltál? – kérdezte Andrej.

– Csak egy kicsit, uram.

– Inni kellene most valamit.

Egérízű szederbort ittak, csuporral merték egy uborkásüvegből. A helyiségben
csak a tűzhely nyitott ajtaja világított. Fényében néha egy ágas-bogas
fémtárgy idomai villámlottak az asztalon. Egy hajnyíró gép volt az, kicsi
szarvaival a fogantyún fenyegetőn, egymaga állt a terítő közepén. Tapintása
finom és hideg. Andrej óvatosan megérintette, majd sokáig méregette, mígnem
megkérdezte:

– Ez pedig mi akarna lenni.

– A férjem hozta a hegyivadászoktól. Holnap életbe lép a kijárási tilalom. Aki
pedig otthon marad, az megnyiratkozik.

– Nyírásról egyszer már volt szó, még a borbély Vili Dunka idején. De akkor
elmaradt valamiért.

– Ezt most mi magunk végezzük. Mondani is akartam az úrnak: holnap este, ha
majd az ágya felé közeledem, már nem lesz rajtam a hajam.

Az útkaparó néha megmerítette csuprát az uborkásüvegben, iddogált, Elvira
Spiridon ezalatt ruhátlanul a takaró alá bújt, és annyi helyet hagyott maga
mellett, hogy a férfi is elférjen. De az még jól megtömte a kályhát gyökérrel,
telehintette fehéren égő tobozzal, hogy világítson. Akkor a nyírógép után
nyúlt, és átült a fekvőhely szélére.

– Életemben nem nyírtam – suttogta. – Még idejövet sem gondoltam volna, hogy
ma lesz a napja.

Átölelte Elvira Spiridon vállát, a gépet homloka közepére illesztette,
nyiszitelni kezdett, és megállás nélkül végighaladt vele a nyakáig. Visszafele
már a csupasz nyakszirt felől indult, át a tarkón, le a homlokáig. A lenyírt
tincseket, mint valami frissen vasalt selyemszalagokat egymás mellé, a szék
támlájára helyezte. Amikor befejezte, és a helyiségben már az asszony csupasz
feje is világított, újra mert magának az uborkásüvegből, ivott.
130

De csak rövid szünetet tartott: ahogy kiürült a csupor, Elvira Spiridont
kitakarta, és a nyírógépet a hasára, köldöke alá helyezte. Lassan, apró
harapásokkal indult el vele lefelé, ahol a moha sötétlett.

– Ha ott is kellene, tudatták volna, uram.

– Először nyírok – suttogta az útkaparó –, most kérlek, ne mozogj.

Elvira Spiridon rövid időre megfeszült ugyan, de amint a nyírógép kései
átmelegedtek a bőrén, ernyedten átadta magát, kitárulkozott, hogy Andrej
minden kis porcikájához, szőrpamacsához odaférjen. Andrej a végén az ölébe
vette és gondosan végigfújdogálta egész testét, lefújt róla minden szőrt és
hajat.

– Ha egyszer elmennék innen – súgta hajnalban a nő fülébe –, lehet, elkérlek
majd Severin Spiridontól. Ha úgy döntenék, hogy magammal viszlek.

– Próbálja meg az úr – válaszolt szintén suttogva Elvira Spiridon. – A férjem
biztos magának ad.

– Jól értetted: elmenni, mostanság ilyesmiket forgatok a fejemben. Csak erről
nagyon szépen kérlek, hallgass.

– Hallgatni? Ilyesmit ne kérjen tőlem az úr.

Reggel Elvira Spiridon fejére kendőt kötött, teletömködte a levágott
hajtincsekkel. Az útkaparó ezalatt tüzelőt aprított, hadd érje egyéb
meglepetés is Severin Spiridont. Öreg fenyőtuskókat hasított, a lepattanó
kéreg alól kövér lárvák potyogtak a fagyos szürke földre. A röpködő forgács
között is nehéz varjak ereszkedtek rájuk.

– Nem fázik a hasad? – kérdezte az útkaparó, amikor egymás nyomában haladtak
az utászház felé. – De kérlek, egészen őszintén.

– Nincs melegem, uram.

– Ördög tudja, mi ütött belém. Baj van az idegeimmel. Fura napom volt nagyon a
tegnap.

– De én úgy vélem, csupaszon is kedvel engem az úr.
131

– Ühüm, nagyon.

Az utászházban Severin Spiridon ruhástól feküdt félálomban a priccsen. A padló
ringásától alóla elindult egy felborult italosüveg, és a helyiség közepére
gurult. Az asztalon még érintetlenül állt újságpapírba csomagolva a liga
emberének a sajtja. Ahogy az átnedvesedett papírt lefejtették róla, szürkén
rátapadva rajta maradtak a betűk. Mire este az utászházba érkezett, mesélte
Severin Spiridon, a liga emberét már nem találta ott. Csak az a sajt állt az
asztal közepén s az üveg, alján kevés keserű pálinkával. Az egész ház tele a
várótermek félelmetes szagával.

– Kapcsold be a rádiót, kérlek.

– Most inkább nem – mondta az útkaparó.

– Tudni szeretném, mi történik. Sinistrán ma alakul a liga. Kapcsold be,
kérlek.

– Nem szokásom a rádiózás. Azonkívül ezt most, sajnos, amúgy sem lehet. Nézd
csak, kérlek, nincs benne elem. – Az útkaparó meg is mutatta a táskarádió
hátulján a feltört, üres rekeszt, ahonnan valaki kitépte az elemeket.

Elővett egy kisebb üveg denaturált szeszt, töltögetett belőle vízzel félig
telt csuprokba, aztán késsel forgácsokat hasított kavargatni, hogy a gyanta
zamata ázzék benne.

– Ha megengeded, most magammal vinném az asszonyt – mondta déltájt Severin
Spiridon.

Andrej végigpillantott rajtuk, Elvira Spiridon hajjal kitömött asszonyos
kendőjén, majd tekintetével az ölén állapodott meg.

– Kérlek – bólintott.

– Kicsit földúlt ennek az embernek a szaga, és ma nem szeretnék magamra
maradni. Engedd, töltse velem az egész napot, este majd szokás szerint útnak
indítom.

– Vidd, hiszen a tied.

Andrej még egy darabig egymaga iszogatott, a tavaszi legyektől zsibongó
ablakon kitekintett a tisztásokra, a fölöt 132 tük elhúzó felhőkre. Később,
délután maga is megjárta magát a hólétól vizenyős, süppedős réteken. Teremtett
lélek nem mozgott a hágó két-három tanyája körül, csak az erdő alatt aszalódó
hórakások, lila jégkupacok mögött lobbant föl néha egy-egy lángoló rókafarok.

Hazatérőben volt, amikor a szerpentineken, égő fényszórókkal, teherautók
közeledtek hosszú oszlopban a hágó tetője felé. Valamennyi ponyvával volt
letakarva, egyikben-másikban dorongok, láncok, vasrudak csörögtek, némelyik
gunnyasztó, alvó emberekkel volt tele. Mögöttük a várótermek tetűporos,
szeszgőzös, fojtó illata örvénylett.

Már az utászház közelében, az egyik keréknyomban Andrej egy golyó szemet
talált. Egy magányos szemgolyót, amelyre rátapadt a sár darája, sárga leve, de
mégiscsak szem volt. Biztos az egyik leponyvázott teherautóról potyoghatott
le. Olyan olajos fényekben játszott, mint a liga emberéé.

Az utászház még mindig tele volt az idegen ember szagával, az útkaparó nyitva
hagyta maga mögött az ajtót, kitárta az ablakot. Alkonyatig kakukkfüvet
pipálgatva könyökölt a légvonatban. Akkor egy vödör vízzel, kicsi tábori
lapáttal leereszkedett az útra, gondolta, bárkié is volt, mégiscsak eltemeti
azt az egy darab szemet. De már nem találta ott.

A lila égbolton kettős narancsvörös pántlika, napsütötte kondenzcsík
világított, mint sínyomai a hágó tisztásain. Az útkaparó nyitott ablak előtt
ülve várakozott; a hágó fölött már a tavasz illatai úszkáltak, az erdőből
alkonyat után is madárcsattogás emelkedett. Severin Spiridon házának
zsindelyfedeléről nemsokára füst oldódott el, a szélcsendben fölfelé szállt,
ezüstösen beburkolta a holdat.

„Ezentúl hiába várom – gondolta magában Andrej Bodor, az útkaparó. – Hiszen
mától életbe lépett a kijárási tilalom.”

Kicsit bosszankodott, hogy rászedték, hiszen ravasz 133 szomszédja, amikor
feleségét magával vitte, titkon minderre számított. Azért elképzelte, némi
meglepetés őt is éri majd, amikor izzó fenyőtobozok éles fényében térdel,
szemben Elvira Spiridon csupasz hasával, szemben a meztelen valósággal.
134
12. (Nikifor Tescovina palástja)

Amikor az éjszakai tehervonat mellett, a peronon rátaláltak Petra Konnertra, a
mozdonyvezető lányára, még élt, de az volt a kívánsága, vigyék egyenest a
laktanya halottas kamrájába. Sinistráról egy fékezőfülkében utazott, s amint a
vonat megállt, kigurult belőle, és többet nem mozdult, csak sötét nedvek
eredtek alóla minden irányba. Apja, Peter Konnert nyikorgó, egykerekű
talicskába fektette, mint egy zsák gabonát, s az éjszaka kellős közepén
végigtolta a sötét falun.

Tavasz elején a halottasházban fölgyülemlett a munka, az útkaparó Andrejt
gyakran szólították be a hegyivadászok kisegítő ügyeletre – egy kis üveg
sósborszeszért vállalta –, ő segített most kiteríteni a lányt a szutykos
szürke kőasztalra. Petra Konnertből a sok nyíltól, dárdától, golyótól ütött
sebeken át hajnalig eltávozott az élet. Előző nap Sinistrán rendbontás
történt, s bábjátékosok, csepürágók színházi kellékekkel hadonászva csörtettek
az utcákon.

Andrej szokása szerint megnyitotta a kamra szellőzőit, de az üde harmattól
illatokkal rakott levegő helyett megszégyenítő bűz áradt befelé a nyílásokon.
Máskor tavaszi reggeleken az előző éjjel kinyílt farkasboroszlán bódító illata
üli meg a völgyet, de ez, ami most még a réseken át is beszivárgott, az emberi
ürülék szaga volt, a helybelieké, a hegyivadászoké, benne még a denaturált
szesz áporodott zamatai úszkáltak.

Amint kivilágosodott, és a köd is kiemelkedett a laktanya udvaráról,
körös-körül mindenütt ott bűzlött maga a magyarázat. A kerítésekre, falakra –
még a halottasházéra is 135– nyúlós mázzal odakenve ez állt: „az anyátok
picsája”. Úgy látszik, a sinistrai forrongók egyike-másika eljutott Dobrinba.

Miután az éjszakai ügyelet ideje letelt, eljött Andrejért Coca Mavrodin
ezredes. Úgy volt, aznap újabb dögcédulákat osztanak ki a természetvédelmi
területen, bokára, csuklóra valókat; a terepjáró a porta előtt várakozott,
hátsó ülésén kenderzsák hevert, tele a kész, névre szóló bádoglapokkal. Amint
kifordultak az útra, szembevilágított velük a kutya alakú hófolt, telekenve
hatalmas barna betűkkel: „az anyátok”.

– Ez Géza Kökény – mondta Coca Mavrodin-Mahmudia. – Ráismerek az írásáról.
Tudnám, honnan szedett össze ennyi szart.

Tavaszodott, olvadékvizektől zavarosan, bő zuhatagokban tombolt a patak, a
csillogó habos kövek tele billegetővel. A parton, a fű között, mint valami
titokzatos gyertyák, a hirtelen kivirult törpe tárnics lila lángjai lobogtak.

– Elárulom, Andrej, fáradt vagyok.

– Nem jó, ha a kisasszony a titkaiba avat.

– Kicsúszott a számon.

Amikor a völgyön terepjáró halad fölfelé, hangja egyszerre végigfut a falak
között, alig hagyja el a jármű Jean Tomoioaga ezredes őrhelyét, a sorompót,
odafönn tudni lehet, a hegyivadászoktól közeledik valaki. Mire az érkező
feltűnik az utolsó kanyarban, Nikifor Tescovina rendszerint az út végében, az
olajos kátyúk között ácsorog.

A tócsák között most csak néhány varjú szökdellt, nem lebegtek a fafüst kékes
fonalai a zsindelyfedél fölött, az ajtó a szél érintésétől, magában moccant
egyet-egyet.

Nikifor Tescovina háttal a bejáratnak, tarisznyák, átalvetők fölé görnyedve
matatott. Hallhatta jól a terepjárót a meredeken, a közelben cuppogni a sárban
az abroncsokat, Coca Mavrodin könnyű lépteit is a küszöbön, mégsem 136 fordult
hátra. Két fekete kislánya kendőkbe burkolózva, összeszorított lábbal ült a
priccs peremén. Vadonatúj, gumiabroncsból szabott bocskort viseltek, fehér
posztókapcát, mint a legnagyobb ünnepeken.

– Elutazik? – kerülte meg a csomagokat Coca Mavrodin.

– Én? – Nikifor Tescovina elkötötte a tarisznya száját, csak azután
egyenesedett ki.

– Csak ebből a sok batyuból gondolom.

– Dehogy. Szeretem a holmim néha átcsoportosítani.

– Azért kötözzem föl nekik? – kérdezte Andrej és a dögcédulákkal teli zsákot
kiborította az asztalra. – Vagy most már ne?

– Felőlem kötözze.

A gyerekek lába a kapcák alatt retkes volt, kicsit gombaszagú, amíg Andrej
motozott rajtuk, karjuk remegett, szemük sarkába bogyó nagyságú
kristálycseppek ültek.

– Újabban ez a mániájuk, elmennek – mondta Coca Mavrodin. – Úgy látom, még
maga is.

– Én csak a holmim rendezgetem.

De ezt Coca Mavrodin már nem hallotta. Kilépett a kantinból, elhaladt a
terepjáró mellett, az ösvény felé, amely Géza Hutira házához vezetett. Andrej,
vállára kapva a bádoglapoktól zörgő zsákot, maga is utána indult. Nikifor
Tescovina megvárta, míg Coca Mavrodin eléri az első fatörzseket, amelyek
eltakarták, aztán – psz-psz! – Andrej után küldött néhány hívogató jelet.

– Mi van?

Nikifor Tescovina szótlanul várakozott, amíg Andrej visszatért a közelébe,
akkor zubbonyát melle fölött megragadva magához vonta.

– Egyszer azt mondtad, meghálálod, ha beprotezsállak. Most volna az. Nagyon
szépen kérlek, adj egy húszast. Egy olyan húszdollárosra gondoltam. Amiből
tudom, van egy-néhány neked.
137

– Nem lehet – ingatta meg a fejét Andrej. – Csak ezt ne, Nikifor Tescovina.
Csak ezt ne kérd tőlem.

– Gábriel Dunkától még a pontos helyét is tudom, hol tartod. Ha akarom, most
megyek, és kiszolgálom magam. De én tisztellek, ezért inkább személyesen kérem
tőled. Adj nekem egy olyan húszast.

– Sajnálom, Nikifor Tescovina, de nem! Hamarosan égető szükségem lesz minden,
de minden garasra.

– Ne gondold, hogy ingyen kérem – szorította meg Nikifor Tescovina Andrej
melle fölött a zubbonyt –, remélem, sejted. Neked adom valamelyik gyereket,
elég nekem egy. Nagyon kérlek, válassz, bármelyik a tied lehet. Hogy én a
másikkal elmehessek.

– Öreg vagyok már hozzájuk. Meg aztán, ugye, az anyagiak. Egy garasról nem
bírnék lemondani. Nem, Nikifor Tescovina, kimondtam az utolsó szót ebben a
dologban.

Coca Mavrodin-Mahmudia a vörös forrásnál várta be Andrejt, sarjadzó lósóska és
csalán között ült a földön, körülötte szél duruzsolt az eldobált üres
üvegeken. Onnan már ki lehetett látni a völgy végébe, amerre Géza Hutira
házfedele csillogott. Coca Mavrodin távcsövön azt nézte, nem fordult az
útkaparó felé, csak megkérdezte:

– Pénzt kért?

– Megpendítette.

– Rosszkor, ugye. Pont most, amikor magának is annyira kell. Persze, gondolom,
nem egészen ingyen képzelte.

– Úgy valahogy.

– Maga meg sajnos már öreg az ilyesmihez.

A völgykatlan fölött nemrég felhő húzott el, a földön, lucskos fűcsomók között
a havas eső üveges ikrái peregtek. A ház kőfala is olvadozó szürke foltoktól
csillogott. Az ablakot belülről pára borította, néha beletörölt egy kéz, hogy
ki lehessen látni rajta.

A priccs az ablak alatt állt, a szakadozott szürke pokróc 138 alatt Géza Hutira
nyújtózott, Bebe Tescovinát fél karral átölelve tartotta, szakálla
összeelegyedett a gyermek hajával. A kitárt ajtóból közönyös arcukra hullott a
behavazott hegyoldal fénye.

– Még sosem járt errefele – mondta Géza Hutira Coca Mavrodinnak. – Valami
történhetett.

– Csak a várható időjárás felől érdeklődnék. Mit mutatnak a műszerek?

– Mostanság nem érek rá leolvasni őket. Megszállt a test ördöge.

– Most veszem észre – szólt közbe Andrej Bodor, és meglóbálta kezében Géza
Hutira dögcéduláját –, egyidősek vagyunk. Mindketten harminchatosak.

– Ne engedd, hogy ránk kötözze őket – szólt közbe Bebe Tescovina is. – Kérlek,
ne állj többet szóba velük.

– Igen, harminchatos, az egy nagyon jó évjárat volt – morogta Géza Hutira. –
Mind vittük valamire. – Megsimogatta Bebe Tescovina fejét, belemarkolt rövid
vörös hajába: – Hadd kötözze rád, ha akarja. Ha elmennek, levesszük őket.

Az asztalon már csak Béla Bundasian bádog névjegyei maradtak. Coca Mavrodin,
mintha mindegyiken más és más lenne a szöveg, a fény felé tartva sorra
kibetűzte, majd egyenként kihajította őket a nyitott ajtón a jeges fűre, a
kövekre. Levetette sapkáját, köpenye ujján, ki tudja, miért, kicsit föltűrte a
mandzsettát, és kapaszkodni kezdett a padlás felé a létrán, mígnem feje elérte
a mennyezetet.

– Szeretném, Andrej, ha végre bemutatná nekem a mostohafiát. Remélem, itthon
találjuk.

Fél kézzel kinyomta a csapóajtót, és a nyíláson bekukkantott a padlástér
homályába, amelyet csak a zsindely hasadékain bevillanó fény pengéi
szabdaltak. Közöttük Béla Bundasian szemüvege csillogott.

– Akkor most engedje meg – szólalt meg Andrej Coca 139 Mavrodin mögött a létrán
állva –, hogy bemutassam Béla Bundasiant, a mostohafiamat.

– Örvendek, Bundasian, és mindjárt egy vallomással kell kezdenem. Fene tudja,
mi van ma velem, az imént kidobtam a dögcéduláit. Nem kellenek többet. Azért
jöttem, hogy ezt tudassam magával.

– Miket mond?

– Töröltem a nyilvántartásból. Úgy tudom, van a papájának egy ismerőse, az
majd elviszi magukat messzire. Idegenek, menjenek innen.

– Nem ismerem magát. Nem tudhatom mit akar.

– Megígértem, menjen, amíg lehet.

– Momentán ilyesmiről szó sem lehet.

– Most ne faksznizzon.

– Maga is tudja, megöltem valakit. Nem távozhatok innen.

– Dehogyis ölt. Nagyon téved, Bundasian, mindenki él, virul maga körül. – És
amikor látta, hogy Béla Bundasian két kézzel markol a szénából, és a fülére
tapasztja, mint aki többet egyetlen szót nem akar hallani, még hozzátette: –
Ha szép szóval nem megy, denevéreket és baglyokat fogok ideküldeni, hogy
cincogjanak és huhogjanak a fülébe, amíg meggondolja magát. Ha egyszer
elengedtem, menjen.

Amikor a meteorológus házát elhagyták, Géza Hutirából és Bebe Tescovinából
semmi sem látszott ki a takaró alól. Csak a szerelem megcsukló nevetései
buggyantak ki a pokróc lyukain, egymáshoz simuló bokájukon, karjukon a
dögcédulák csilingeltek.

– Mihez kezd, ha elmegy? – kérdezte lefelé menet Coca Mavrodin. – Mivel keresi
majd meg a betevőt?

– Úgy gondoltam kisasszony, csontfaragványokkal.

– Ezt pontosabban hogy értsem?

– Jártamban-keltemben elég sok csontot találtam az erdőn, elárulom, már
próbálkoztam: virágok, őzikék, 140 gombák, egy-egy őrt álló ezredes. Veszik az
ilyesmit a népek.

A kantin valamennyi ablaka kitörötten villámlott, közöttük az üres teremben
madarak röpködtek. A küszöböt máris kezdte ellepni a moha, az ajtó hirtelen
megvénülve meg-megnyekkent a szélben. Ott lógott rajta Nikifor Tescovina
palástja, amelyet Gábriel Dunka mormotáiból darabonként drótokkal kötözött
egybe; még fejre való csuklya is készült rá, azon fityegett egy darabka
telefirkált nyírfakéreg. „Mégiscsak elveszek egy húszast, Andrej – állt rajta
–, kelleni fog a gúnyám neked, másként megfagysz a sok jeges birkahús között.”
141
13. (Gábriel Dunka neve napja)

Gábriel Dunka meztelen nőt életében először harminchét éves korában látott.
Igaz, törpe volt. Az épülőfélben lévő sinistrai börtönből a néptelen
országúton hazafelé tartott piros furgonjával, amikor Elvira Spiridon
leintette. Aznap kora hajnaltól havas eső zuhogott, a patak menti fenyők,
kutyabengék közé nyirkos köd ereszkedett, széltől hajtott foszlányok húztak át
az út fölött; közöttük porcelános fénnyel az ázott asszony alakja csillogott.
Nem volt rajta semmi ruha, csak súlyos haja tapadt, mint szakadozott rossz sál
a nyakára. Combja, ágyéka mintha kivirult volna a tavaszi fergetegben,
fenyőtűvel, kék, fehér és sárga virágszirommal volt tele.

Gábriel Dunka látásból ismerte az asszonyt, aki a Baba Rotunda-hágón lakott,
és gombás tarisznyákkal, áfonyás, szedres puttonyokkal a hátán a
gyümölcsbegyűjtő felé menet néha elhaladt a kerítése előtt. Arra azért soha
nem gondolt, hogy egyszer meztelenül integetve pont rá vár majd az országút
mentén. Nem szívesen ugyan, de fölvette.

Nem maga mellé ültette a vezetőfülkébe – a raktérre, az ablaküveg szállítására
szolgáló léckeretek közé fektette, nehogy valaki meglássa. A furgon nem a
sajátja volt, ő csak üveglapot szállított a Dobrin Cityben levő műhelye és a
sinistrai építőtelep között; a börtönigazgatóság és a hegyivadász-parancsnok
külön engedélyével furikázott az utakon. Furcsán vette volna ki magát, ha
valami hatósági ember vagy akár csak egy helybeli paraszt meglátja, hogy
szolgálati idő alatt, a sárga rendszámú hivatalos kocsival 142 meztelen nőket
fuvaroz. Elvira Spiridont a lécek mellé nyújtóztatta, és rácsapta az ajtót.

Gábriel Dunka a nőnek a hasáig ért, most kissé megszédült, hogy a
köldökszaggal együtt szívta be róla az eső illatát.

Amint hazaérkezett – egyszerű, kopár falusi udvaron lakott egy fészerben,
amely egyúttal műhelyül is szolgált –, egyenesen az ajtónak farolt, hogy az
asszony feltűnés nélkül bejuthasson. Tudta, a patak túlsó partjáról szomszédai
minden mozdulatát távcsövön figyelik; aligha lehet betelni egy törpe
látványával.

Amint az sejthető volt, Elvira Spiridon nem járt a legtisztább utakon. Aznap
reggel társával megpróbált kiszökni az országból, de számára már a kezdet
kezdetén balul ütött ki a dolog. Mustafa Mukkerman, a török kamionos, aki a
Beszkidekből fagyasztott birkát szállított a Balkán déli csücskei felé, és a
kampókon lógó húsok között, úgy mellékesen, néha mindenre elszánt embereket is
kicsempészett, őt nem vette föl. Beszálláskor derült ki, a sofőr még régen
megesküdött, nőt nem szállít; egyszer valami némber idegességében
telepiszkolta neki az egész kocsit. Úgyhogy társa, Andrej, az útkaparó elment,
ő, Elvira Spiridon ott maradt az úton, ráadásul anyaszült meztelenül.

Aznap kora hajnaltól esett az eső, a kamionra várakozó Andrej Bodor és Elvira
Spiridon meztelenre vetkőzött, hiszen ázott ruhával nem ülhettek be a
zúzmarás, fekete éjszakába, az maga a halál. A ruhát tehát egyazon, külön erre
a célra magukkal hozott fóliatasakba gyömöszölték, azzal, hogy majd útközben,
a raktéren felöltöznek. Az asszony kis ideig huzakodott ugyan a sofőrrel, de
mindhiába. Mire észbe kapott, a kamion Andrejjal és a becsomagolt ruhákkal
máris tovagurult. El délnek, a Balkán felé, amerre éjjel és nappal a szabadság
fényei villóznak. Elvira Spiridon ott maradt meztelenül a nagy szürke köd
kellős közepén.
143

Egy darabig bőgött, aztán erőt vett magán, tépett egy örökzöld repkénylevelet,
és a hasa alá biggyesztette. Biztos régi képeken láthatott ilyesmit. De hát a
szél hamar lekapta a levelet onnan.

Elvira Spiridon fenyők, csupasz nyírek, kutyabengék között kóborolt fél
délelőtt, mígnem a város felől, a köd mélyén föltűnt az üveges törpe furgonja,
lökhárítóján a messzire világító sárga kormányzati rendszámmal. Nem
tévedhetett, a hegyivadászok járművein kívül, amelyeket köhögő hangjukról
messziről föl lehetett ismerni, ez az egy piros, félig-meddig civil külsejű
gépjármű futott az egész hegyvidéki körzetben. Az asszony hajnalban már látta
Sinistra felé menet, tudta, valamikor a délelőtt folyamán visszatér.

Gábriel Dunka szokatlan termete ellenére állami alkalmazott, iparosember volt.
Az akkor épülő sinistrai börtön ablakaira ő homályosította az üveget.
Valahányszor egy-egy adaggal elkészült – hetenként harmincöt-negyven ablakra
valóval –, a rakománnyal ő maga indult a helyszínre. Fészere közepén hatalmas,
homokkal tele láda állt, a homok alatt síküveg lapult. A törpe mezítlábasan
addig sétált a ládában, amíg léptei alatt az üveg átlátszatlanra karcolódott.

A láda mellett is halomban állt a homok, Gábriel Dunka Elvira Spiridont arra
fektette.

– Türelmét fogom kérni – mondta neki halkan, elfogódott hangon –, mindjárt
keresek magának egy-két megfelelő ruhadarabot.

Félig kettőbe tépett egy barna papírzsákot, amit addig a padlón szőnyegnek
tartott, kibontotta, és a nőt betakarta vele. Elvira Spiridon réseiből,
zugaiból még mindig esővíz szivárgott, szétfutott alatta a homokon.

– Maga igazán nagyon rendes – jegyezte meg Elvira Spiridon. – Bajban ismerni
meg az embert.

– A baj, az megvan – hagyta rá a törpe. – Azért is 144 megkérem, ne nagyon
mozogjon, maradjon lehetőleg egyfolytában fekve. Én ugyanis nem látszom ki az
ablakon. Szomszédaim, ha bármi mozgást észlelnek, megtudnák, idegen van nálam.

Gábriel Dunka megtömte fűrészporral, fenyőtobozzal a vaskályhát, begyújtott, s
később a lobogó rakásra még egy jókora hordódongát is dobott. Nemrég
szüntették meg a közeli gyümölcsbegyűjtőt, s az udvarán roskadozó öreg
hordókat hamarosan széthordták tüzelőnek. A donga most kékes-zöldes
gáznyelveket eregetve, kábító gyümölcsillatot árasztva pattogott. A tépett
papírzsák alatt Elvira Spiridon vacogott.

– Én már rég megszáradtam volna – jegyezte meg kissé zavartan Gábriel Dunka,
félig-meddig tréfának szánva. Mivelhogy kisebb vagyok. Természetesen, a kisebb
dolog, például a törpe, hamarabb szárad.

– Máris tanultam valamit – hallatszott Elvira Spiridon hangja a papírzsák
alól. – Na de azért az, hogy törpe, az talán túlzás.

Gábriel Dunka minden vagyonát egy elnyűtt, kopott vulkánfíber bőröndben
tartotta, amely polcként szolgált fekhelye és a fészer nyirkos fala között.
Most fölnyitotta, belekotort, könyökig elmerülve a sok szürke-sárga foltos,
bogárjárta, egérszagú ruhanemű között. Néhány darabot kiválasztott és maga
mellé készített.

– Van elem a rádiójában? – kérdezte váratlanul Elvira Spiridon. – Talán
bemondanak majd valamit.

– Vanni van. De én szeretem hallani, ha valaki az udvaron át közeledik. Nem
kizárt, szomszédaim mégiscsak gyanút fogtak.

– Én nem félek maga mellett. Bármi történjék, biztos kitalál valamit. Ne vegye
sértésnek, ahogy elnézem magát, talpig férfi.

– Kösz. Tudja, néha meglátogat egy-két ezredes, kíván 145 csiak, hogyan készül,
mitől lesz olyan párás, hamvas az ablaküveg.

– Ó.

– Azért nem túl gyakran. Mindenesetre, ha sor kerülne egy ilyen megjelenésre,
ne látszódjék ki magából semmi. Még a szuszogás is legyen mellőzve.

Gábriel Dunka fölnyalábolta a halom kiválogatott ruhát, és Elvira Spiridon elé
borította. Föllebbentette az asszonyról a foszlott papírlepedőt, és
nekilátott, hogy felöltöztesse. Egy pántos rövidnadrággal próbálkozott
először, de bárhogy is erőltette, az egyik lábon is csak a térd alá jutott el
vele.

– Sejtettem, nem lesz jó magára. De az ember nem adja fel csak úgy,
látatlanban. És bocsásson meg, most nagyon furcsán érzem magam. Gondolom,
attól, hogy a bőréhez értem. Gyönyörteli érzés, kicsit szédülök most. Az is
lehet, mindjárt megfulladok.

Eltávolodott az asszony közeléből, a vízzel teli vödör fölé hajolt, egész
arcával megmerítkezve hatalmas kortyokkal lefetyelt. Nem törülközött meg,
hagyta, csorogjon végig a víz a nyakán.

Elvira Spiridon maga kezdett öltözködni, egyik, majd másik lábára is egy-egy
egész nadrágot húzott, két zubbonyt az ujjuknál fogva összekötözött, és
betakarta magát.

A törpe edényeket csörgetve foglalatoskodott. Egy lábasba vizet készített, a
kályha mellett állva kivárta, míg forrni kezd, akkor telehintette száraz
áfonyalevéllel. Egy ideig állni hagyta, majd két bádogcsuporba osztotta. A
kész teába jócskán töltött egy kék címkés literes üvegből. Kék denaturált
szesz volt benne. Végül az egyik gőzölgő csuprot Elvira Spiridon keze ügyében
a homokba állította.

– Proszit. Isten hozta. A fennvaló akarhatta, hogy így történjék.
146

– Egészségére, Dunka úr. Úgy tudom, valamikor mostanság van Gábriel napja.

– Az könnyen meglehet.

– Remélem, nem leszek terhére.

– Ha nem mozog, akkor azt hiszem, nem nagyon. Akkor addig marad, amíg jól érzi
magát. Vagy amíg nekem kell távoznom innen. Ez hamarosan bekövetkezhet.

– Igazán sajnálnám.

– Pedig előfordulhat, egy szép napon elmegyek. Nem a messzi délre, kedves
Elvira, hanem bennlakónak a sinistrai múzeumba. Nemrég eladtam magam nekik, én
most már az övék vagyok, hozzájuk tartozom. Eladtam a csontvázamat a
természetrajzi gyűjteménynek. Tudja, azok szeretik beszerezni az ilyesmiket.
És ami nem mellékes: előre kifizették.

– Ó, magam is hallottam, a múzeum telis-tele sok eredeti holmival.

– Nos, igen. És gondolom, biztos majd ők is futnak a pénzük után. Kérdés:
megvárják-e, amíg annak rendje-módja szerint földobom a tappancsot. Ki tudja,
egy szép napon értem jönnek. Ízlik a tea?

– Épp dicsérni akartam.

– Akkor talán jobb is, ha most elhallgatunk. Az üvegnek is füle van.

Odakünn alkonyodott, bekékültek az apró ablakok. Gábriel Dunka megvárta, amíg
annyira besötétül, hogy arca tükröződni kezd az üvegen, akkor föllobbantott
egy gallyat, lángjánál gyertyát gyújtott. Megborotválkozott, nedves bőrén
eldörzsölt egy fél marék kakukkfüvet. Régi, sárgás, puhára mosott ingecskét
vett magára, és régi iskolai egyenruhája zubbonyát, ami ráncosan, gyűrötten
évek óta hevert a bőrönd fenekén. Elérkezett a nap, amikor újra magára
ölthette.

– Biztos látja, izgulok egy kicsit – suttogta. – Ma este 147 először leszek
nővel. Nézze csak, még a haj is remeg a fejemen.

– Semmi ok az idegességre – felelte Elvira Spiridon –, nem olyan tudomisén
milyen, különös ügy. Hallomásból azért bizonyára tudja, miből is áll a dolog.

– Szégyenszemre, meglehetősen járatlan vagyok. Pedig a törpéknek meglehetősen
jó a hírük.

– Éppen ezért nyugodjék meg. Gondoljon csak arra, magam is kutya helyzetben
vagyok.

Gábriel Dunka megremegett, nagyot sóhajtott. Kétségkívül, a papírzsák alól az
asszonyi bőr egy-egy nyugtalanító fehér foltja egészen más volt, mint a
legelső látomás: amikor az asszony csatakosan, félelemtől lila állal,
kifehéredett orral, vértelen fülcimpákkal az országúton elébe állt.

– Ne vegye parasztságnak – szólalt meg újra halkan, felindultan –, de most kis
időre magára hagyom. Csak akkor térek vissza, ha megnyugodtam. Nem tudom, mi
van velem, de el kell mennem, mert rettentő furán érzem magam. Félek, még
megölöm magam.

– Jól van, Dunka úr, menjen. Fújja ki magát. Ha megengedi, míg maga oda lesz,
néha töltök magamnak. Mire hazatér, bemelegszem.

Dobrin City egyetlen főutcáját, amely a völgy alján a patak vonalát követve
kanyargott a hágó felé, már évek óta nem világították. Errefele az emberek, ha
sötétben találkoztak, a szagukról ismerték föl egymást. Gábriel Dunkáról, aki
a távoli fényektől derengő pocsolyák között botorkált, messziről azt lehetett
hinni, egy kutya. Csak a léptei cuppogtak a sárban másként.

Kisétált a faluból, a természetvédelmi terület bejáratáig, ahol a sorompó s
mellette az őrszoba állt. Máskor is előfordult, hogy Gábriel Dunka fölkereste
Jean Tomoioaga ezredest, aki évek óta egymaga szolgált a poszton, és az
őrszobán lakott. Ha megjelent nála a törpe, az ezredes 148 rendszerint padlóra
terítette zöld-fehér kockás ingét, előkerült a néhány házilag faragott
sakkfigura, különleges színű kavics, s ők lejátszottak néhány parti sakkot.

Most is így történt, de Gábriel Dunka szemmel láthatóan hamar belefáradt a
játékba. Jean Tomoioaga ezredes észre is vette, máshol jár az esze, és
elnézően figyelmeztette tévedéseire. Így is aznap este a törpe minden játszmát
elveszített.

– Semmi értelme a játéknak – tette szóvá a dolgot Jean Tomoioaga ezredes. –
Csak tönkreverlek. Mondd, mi bajod?

– Remélem, őszintén érdeklődsz, s akkor én sem hallgatok. Azért is kereslek
most, késő este. Tudod, jelenteni kellene valamit.

– Bennfentes vagy, magad is megteheted…

– Most azonnal Sinistrára kellene utazni a hírrel, márpedig én a furgont
napszállta után nem használhatom. A dolog sürgős: határátlépési ügy.

– Jó, majd gondolkozom.

– Nem majd, hanem most. Az illető nagyon rosszban sántikálhatott, még ruha
sincs rajta. Nálam található a műhelyben. Tegyél valamit, hogy azonnal vigyék
onnan.

Dobrinban Elvira Spiridont legutoljára Géza Kökény látta. De sok öröme neki
sem tellett benne. Mintha még mindig attól tartana, a szomszédok figyelik, az
asszony négykézláb, csodálatos hasát a földnek fordítva baktatott elő a törpe
műhelyéből, a szobor mellett várakozó terepjáró felé.

Gábriel Dunkát, amikor nemsokára hazatért, fészerében újra az üveglapok rideg
csöndje fogadta. Az ázott bőr, haj és a titkos rések illata ajtónyitáskor
kiszabadult, s a Sinistra szele örökre magával ragadta.
149
14. (Béla Bundasian tüze)

Élete utolsó napján Béla Bundasian arra ébredt, hogy Géza Hutira házában
egymaga maradt. A zsindelyfedelen egész éjszaka ónos eső pergett, s hajnalban,
amikor hirtelen elállt, az üres falak között kietlen csend sistergett tovább.
A tűzhelyen magában rebbent a hamu, búgni kezdett, a kürtő huhogott, biztos
megérkeztek a baglyok, amelyeket Coca Mavrodin ígért.

Leereszkedett a padlásról, látta, a kunyhó elhagyatott, hiányzott Géza Hutira
máskor kilincsen lógó, gumírozott, csuklyás viharkabátja, táskája, távcsöve,
nem csüngtek helyükön a jégsarkak sem. Az üres fekhelyen a széna éledezett,
félig-meddig mutatta Bebe Tescovina összegömbölyödött alakját, fölötte talán
még egy kevés korai tejszag is érződött. De a boldog pár már messzire járt.

Odakünn mindent, fadarabokat, köveket, a lépcső néhány fokát az ónos eső jeges
máza borította, Béla Bundasian a falba fogózva kerülte meg a házat. A
fészerben facsavarokat keresett, telefúrta velük bakancsa talpát, hogy maga is
mielőbb útra kelhessen. Az ónos eső távozta után a szemközti csúcsok,
meredekek üveggel leöntve, gyémántfénnyel csillogtak, a ház körül a fűszálak
mint összekoccanó poharak csilingeltek a szél érintésében.

A völgy felől, az erdő mélyéről is fémes kaparászás hallatszott, a jégsarkak
pengése, de az csak a visszhang volt: Géza Hutira Bebe Tescovinával akkor már
a gerinc meredélyeit járta.

Béla Bundasian föltette szemüvegét, hamarosan meg is 150 pillantotta őket a
magasban. Eleinte kettejükből csak egyetlen pontszerű alak látszott, amely hol
fölbukkant, hol eltűnt az éles tarajok között; de amint a felkelő nap
végigsütött az ormokon, váratlanul az egész vonulat körvonala az ónos eső
távozó felhőjére vetült, Géza Hutira óriásira nőtt árnyalakjával. Szárnyaló
léptekkel suhant a hegygerinc fölött, Bebe Tescovinát a vállán cipelte, fejét
előretartva, nehogy a gyerek hasa megnyomódjék. Felhők ragadták el őket
Ukrajna felé.

Béla Bundasian teletömte zubbonya zsebét szárított gombával, aszalt vörös
áfonyával, bükkmakkal. Aztán csákánnyal szétverte vele az ajtót, bezúzta az
ablakot, a zsindelyfedelet, néhány csapással a ház sarkán a kőfalat is
megbontotta, utat készített az eljövendő esőknek, szeleknek. Összekulcsolt
kézzel a romok elé térdelt, de amikor a szél egy pertlit sodort előtte a jeges
zománcon, fölkapta, szemüvege szárát tarkóján összekötözte vele, nehogy majd a
szembecsapódó ágak lesodorják.

Kövekbe, ágakba, fűcsomókba kapaszkodva ereszkedett a völgybe. Mögötte a
kunyhófedél csupasz bordáit megszállták a varjak. A zümmögő forrás közelében
ormótlan jégtuskó meredezett, belefagyva, szürke hegyivadászköpeny gallérjáról
medalionba foglalt vörös csillag világított. A kantin kitörött ablakai között
a teremben madarak suhogtak, a küszöböt, mint egy lábtörlő, moha borította,
két mormota ásítozott rajta. Az őrszobán Jean Tomoioaga ezredes priccsén
hanyatt dőlve hortyogott.

– Zokon ne vegye, hogy fölébresztem – suttogta fülébe Béla Bundasian –, de úgy
veszem észre, mindenki elmegy, s én itt maradok szabadon. Kérem, vegyen
őrizetbe.

– Nem tehetem, ilyesmit ne is kérjen. Magát törölték a nyilvántartásból,
minálunk megszűnt létezni. Azt mondom, távozzék, menjen innen.
151

– A kutyafáját, azért megpróbálhatná. Elvira Spiridont is lapátra tette.
Elvégre megöltem valakit.

– Ölt, nem ölt, a maga dolga. Ajánlom, kerülje el széles ívben Dobrint, mert
többé nem tud itt magáról senki. Idegen, menjen.

A Sinistra vize fölött barkapor úszott, rigófütty és a farkasboroszlán bódító
illata. Béla Bundasian a falu közelében letért az útról, megkerült egy törpe
nyírrel, kutyabengével teletűzdelt vizenyős rétet, megkerülte a jóformán
ismeretlen Dobrin Cityt, túlsó végében, a Pop Ivan tövében érte el az
észak-déli országutat. A lejtő egyik hajlatában a töltőállomás kék és sárga
falai rikítottak.

– Milyen nap van ma? – kérdezte, amint a fülke elé érkezett.

– Hétfő, kedd, valami ilyesmi – felelte Géza Kökény, a benzinkutas.

– Szóval nem csütörtök.

– Ó, nem, arról én is tudnék, biztosíthatom.

Béla Bundasian lepihent a közelben, egy darabig hanyatt fekve nyújtózott az út
menti marton. A fölötte elhúzó felhőket, madarakat bámulta, a cikázó
bogarakat, majd felült, és a hegyek alatt elkacskaringózó országutat figyelte.
Órák teltek el, nem haladt el előtte semmi. Föltápászkodott, nyújtózott,
megmozgatta elgémberedett tagjait, aztán körbesétálta a benzinkutat.

– Nincs kedve malmozni egyet? – kérdezte Géza Kökény, a kutas.

– Éppen ráérek. Ha megígéri, nem csal, lehet szó róla.

A malomjáték hálója az olajos földre volt rajzolva, lábbal tologatták rajta a
kavicsokat, fadarabokat. Nem zavarta őket senki, az állomásra egyetlen jármű
sem érkezett. Délután, túl az országúton magányos ló lépdelt át a réten az
itató felé, olyan színe volt, mint a közeli olvadásos hegyoldalnak, fakó,
borzderes.
152

Hosszasan utána bámultak: mint valami mennyei küldöttnek, sörénye körül titkos
fényjelek villóztak, ahogy hangtalanul elbaktatott.

– És ha netán csütörtök volna, azzal se menne semmire – mondta Géza Kökény. –
Mustafa Mukkerman nem jön többet. Úgyhogy várni mostanság nemigen van mire.

– Így most már kezd másként kinézni a dolog. Lehet, át kell szerveznem a
napomat.

– Nem hallgathattam. Itt nálam olajra, benzinre számíthat, egyébbel sajnos nem
szolgálhatok.

A játék befejeztével Béla Bundasian körbesétálta a kutat, megint elheveredett
az út menti marton, zsebéből elrágta a magokat, a szárított gombát,
elszopogatta az aszalt áfonyát. Keveset szendergett is, félálomban hallotta a
távolban a hegyivadászok csörömpölését, amint egy helyen járműveikkel
keresztezték az országutat. Aztán megint csönd lett, ettől magához tért,
kitapogatta kiürült zsebeit, a testét, és föltápászkodott. Nyújtózott egyet,
köpött vagy kettőt, szellentett, elballagott a benzinkútig, és felköltötte
Géza Kökényt.

– Jó, akkor adjon nekem egy kanna benzint, meg egy doboz olajat.

Azzal a húszdollárossal fizetett, amelyet egyszer mostohaapja hagyott neki. A
visszajárót hazai fémpénzben kapta, a rengeteg apróval színültig telt most
valamennyi zsebe. A kannát lóbálva áthágott az úton, beereszkedett a rétre, s
a borzderes szent ló nyomdokain maga is végigbaktatott. A keskeny lapály túlsó
végében a régi malom, az egykori gyümölcstároló épülete magaslott.

Akkor már gerince tövében is érezte, élete legutolsó léptei ezek, őrült, vad
gerjedelem szállta meg, nadrágja előbb csak kidudorodott, aztán fütyköse maga
elől a gombokat lepattintva a rongyok közül a levegőre tört, és az égnek
meredt.
153

Béla Bundasian megállt a patak partján, ahol a füzes eltakarta előle a falut,
a hamvas barkák fölött a lassan távolodó gyémánt hegycsúcsok világítottak.
Lába előtt a sárgászöld fűben, mint égő gyertyák, frissen kinyílt sötétkék
törpe tárnicsok lobogtak. Bakancsát levetette, gondosan egymás mellé helyezte,
belegyömöszölte a kapcákat. Mint lefekvés előtt, legszívesebben vizelt is
volna egyet, de hamar letett róla, férfi létére tudta, pózna nagy szerszámával
nincs mit kezdenie. Ha nem, hát nem, egykedvűen tépett egy tárnicsot,
húgycsöve nyílásába igyekezett tűzni, de sehogy sem sikerült, az mindegyre
kibuggyant belőle. Lába elé hullott, mint egy kék gyertya, tovább világított,
annak a tüze volt az, ami később lángra lobbantotta.

Még a körme is izzott, szikrát vetett az orra hegye, a füle, zsebei
kiszakadtak, s amerre a rengeteg aprópénz szertegurult, a perzselt fű és avar
füstölögni kezdett. Szemüvege kerete is megolvadt, de a forróságtól a lencsék
még jó ideig a szeme előtt lebegtek, így aztán, mielőtt elvágódott volna, hogy
mint könnyű zuzmópernyét magával ragadja a patak, még láthatta a körébe
sereglő kíváncsi bámészkodókat, tekintetükben az idegent illető üveges
közönyt, és biztos már-már kezdte bánni az egészet.

Évekkel később újra megfordultam Dobrinban, Géza Kökénnyel is találkoztam.
Váltig állította, nem is a patak ragadta magával, a szél volt az, ami a
Sinistra mentén apránként széthordta, és egy vagy két héten át – mialatt, mint
egy nedves tuskó sziszegett, füstölgött a viruló tárnicsok között – útban
itatója felé még a borzderes ló is elkerülte a rétet.
154
15. (Géza Kökény éjszakája)

Évek óta tartoztam Gábriel Dunkának négy darab húszdollárossal, egy napon
elindultam régi helyén fölkeresni, törlesztési szándékkal. Vadonatúj metálzöld
négykerék-meghajtású Suzuki terepjárómmal, zsebemben görög útlevéllel egy
tavaszi délután érkeztem a Baba Rotunda-hágóra. Sok év után, mint legelőször,
ezernyi színben, árnyékoktól tarkán pillantottam meg a Sinistra medencéjét,
végében a Dobrin kevély ormait, amelyek körül már az északi égbolt türkiz
színei izzottak.

Úgy gondoltam, a terepjárót az utászház előtt hagyom – elvégre útkaparóként
pár hónapon át egykor magam is ott éltem –, és körbejárom a tetőt, de sem az
utászházat, sem Severin Spiridon tanyáját nem találtam, helyüket csak
találgatni lehetett, néhány esőtől, hótól áztatott, sötétkék üszkös halomban.
Más egyéb a Baba Rotunda-hágón nem változott; egy közeli szirten fémpózna
csillogott, a nyugati látóhatár fölött rongyos szárnyú denevér lebegett,
keleten hatalmas, narancsvörös felhő. Még sítalpaim régi nyomai is ott
kanyarogtak a Kolinda-erdő felé. A felhő visszfényében a zöldellő füvön, a
tisztások zugát, gyantás fatörzseket beragyogva páros márványpántlika
tekerőzött a homályba.

Dobrinban annak rendje-módja szerint mindjárt bejelentkeztem a
hegyivadászoknál, az újonnan épült fogadót jelölték ki szálláshelyül, ahol a
tartózkodásra engedélyezett huszonnégy órát eltölthetem. A fiatal, még gyermek
ezredes – arca csupa púder, ajka merő rúzs – figyelmeztetett, 155 foglaljam el
szobámat hamarost, és aznap ki se mozduljak már, mivel napszálltával, mint
évek óta mindig, érvénybe lépett a kijárási tilalom. Valóban, alkonyodott,
Géza Kökény mellszobra bíbor fényekkel áldozott le túl a tavaszi lombokon.

A fogadó bárpultján konzervesdobozban, folyékony bűzös zsiradékban kanóc
lángja úszkált, fényei a csapos – egykori sakkozócimborám, Jean Tomoioaga
ezredes – borostás arcán pislákoltak. Ujjatlan pólóban, pecsétes, foltozott
katonanadrágban állingált ott, mezítlábra húzott bőrszandáljából a padlóra
konyultak a griffmadárkarmok. Megcsapta előkelő cédrusfaillatom, ezüst hajam
idegen fénye – most éppen sötétkék selyemszalaggal kontyba kötve viseltem –,
és ezek után többé sem arcom, sem hangom nem érdekelte. Megismert-e a pompás
színek, illatok burka alatt – csak találgathatom.

Én egy bizonyos Gábriel Dunka nevű törpe iránt érdeklődtem, kifejezvén abbéli
reményemet, hogy régi címén jó egészségben találom. Bár ismerte jól, hisz
egykor ő is sakkozó társai közé tartozott, most meg sem rezzent.

– Törpe? Nem is tudom. – Jean Tomoioaga csapos megvonta vállát, kinézett az
ablakon. – Törpét itt ne keressen. Ha lett volna is valamikor, bizonyára
elkerült innen.

– Talán elköltözött?

– Úgy is mondhatnám, uram. Ennyi most már legyen elég. Ha mégis bővebbeket
óhajt tudni, kimondom a végső szót: legjobb, ha személyesen érdeklődik
Sinistrán, a természetrajzi gyűjteményben.

Olcsó rumot mért, a pohár alján egy darabka fanyar tárnicsgyökér úszkált. Bár
kissé kaparta torkomat, kedveltem ezt a fajta zamatot, szívesen
felhörpintettem volna még egy fél vagy akár egy egész decit, de Jean Tomoioaga
csapos elhárított:
156

– Most már nem ártana az úrnak is nyugovóra térni, ha nem látná, már rég
elpihentek az idevalósi népek.

Szobám ablakából addig tekintettem kifele, a szürkületben távolodó dölyfös
hegyormokra, amíg végképp elmerültek a kelet felől közeledő lila sötétségben.
De a Dobrin mögött a hold is készülődött, rezes fénnyel árasztotta el
közelében az eget. Az egyik hegyhátra bundásan, pihésen kicsi felhő
domborodott. Színe pont olyan volt, mint azé a kicsi állaté, amelyik egyszer
megette a Géza Hutira fülét.

Így jutott eszembe Géza Hutira, aki állítólag huszonhárom évig nem
nyiratkozott. Róla az elcsapott dobrini borbély jutott az eszembe, Vili Dunka,
megint róla pedig egykori élettársa, Aranka Westin. Aranka Westin, akitől pont
hét éve búcsúszó nélkül távoztam. Talán még nem késő, utólag kimenthetem
magam.

Kiléptem az ablakon a fogadó csalánnal, lósóskával benőtt udvarára, s a
sötétség köpenyébe burkolózva keltem át ismerős kerteken a ház felé, ahol régi
barátnőm, Aranka Westin lakott. Úgy képzeltem, csak úgy négykézláb beosonok,
és mint egy alázatos kutya, ágya elé, a rongyszőnyegre heveredem, de ő résen
lehetett, megelőzött, és az orrom előtt hirtelen kitárta az ajtót. A
sötétségben nem láthatott, árnyalakomról sem ismerhetett fel, legfönnebb a
pöttyös tárnics fanyarkás illatáról, amely azon az estén, akár a régi időkben,
leheletemmel mindig előttem járt. Fölismert, és máris régi álnevemen
szólított:

– Tudtam, Andrej, hogy él valahol. És azt is, egyszer majd eszébe jutok.

– Azért is jöttem – mondtam elfogódottan –, hogy exkuzáljam magam.

Egy darabig még illendőségből erről-arról elbeszélgettünk, de a sötétségben
kezünk a gombok, pertlik oldása közben egyre csupaszabb helyeken találkozott,
mígnem 157 lekerült testünkről minden ruha. Bőre, mintha víz csordogálna
alatta, hűvös volt, hasa alól rég kipusztult a moha, csülkeink koppantak, mint
a borókagyökerek. Történt, ami történt, nem bánom meg soha.

Lankadtan, ütőeremet tapogatva heverésztem langyos közelségében, amikor
egyszerre csak a felhők között megnyikkantak a vadludak. Úgy látszik, végképp
odaszoktak, most utaztak a lappok földjére, hazafelé. Esküszöm, nincs
nyugtalanítóbb szó az övékénél. Az éjszaka csöndjében tisztán hallatszott, a
Kolinda-erdő felől közelednek, s a Dobrint elérve hirtelen északnak
kanyarodnak, a Pop Ivan felé. Gyomrom mélyén is hangjuk remegett.

Úgyhogy amikor nemsokára értem jöttek a hegyivadászok, és tudatták: mivelhogy
kijelölt szálláshelyemet elhagytam, idegen létemre visszaéltem a nép
vendégszeretetével, ezért tartózkodási engedélyemet bevonják, és Sinistra
körzetből örökre kitiltanak, én még el sem aludtam. Éberen vártam, mint őrszem
a reggelt, hogy végre távozhassam.

Terepjáróm a közelben, az éjszaka deres fényeiben vesztegelt, mellette, mint
egy szobor, maga Géza Kökény vigyázott.

Görögország felé megint csak a Baba Rotunda-hágón vezetett a legrövidebb út.
Éjnek évadján a lenyugvó hold csöndjében érkeztem a tetőre, sítalpaim ezüst
szalagja most is ott kanyargott tisztásokon át a Kolinda-erdő búvópatakjai
felé. Utoljára még átjárt egy kis jóleső meleg: azért mégsem tűnök el erről a
tájról nyomtalanul.  
\end{document}
