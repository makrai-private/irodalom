\documentclass{IEEEtran} 
\usepackage[utf8]{inputenc} 
\usepackage{times} 
\usepackage[T1]{fontenc} 
\usepackage[magyar]{babel}

\author{ Wass Albert}
\title{ A titokzatos őzbak (részlet)}
%\date{}

\begin{document}

\maketitle

Abban az évben, hogy apám megházasodott, sokan jártak hozzánk. Többnyire olyanok, kiket nem ismertem eddig s kik bár szeretetreméltóak voltak és kedvesek, egy kicsit mindig úgy néztek reám, mint ahogy egy alkalmatlan holmira néz az ember, aki teherként ül a boldogság nyakán. Éreztem, hogy az újonnan jött embereknek valami ilyen felesleges teher vagyok, ha nem is mondják soha s ezért inkább csak a fákkal barátkoztam s az állatokkal.

Szép nagy kertünk volt, nagyanyám ültette. Amikor csak lehetett, ott bolyongtam a fák alatt, elképzelt álmokat játszottam s hallgattam a madarak s a szelek beszédét.

Sokszor megkérdezte apám, ki dolgos embert akart faragni belőlem: – Mit csináltál egész délután?

– Semmit – feleltem én ilyenkor s apám mindannyiszor megharagudott.

De hogyan is mondhattam volna meg neki, hogy falvakat és városokat építettem, vagy kóborló vitéz voltam, ki sárkányokkal küzdött, vagy hogy a harkállyal beszélgettem, ki végigkopogtatta a beteg vackorfát, orvosságot írt fel neki egy gyertyánlevélre s azt kellett elhozzam a rétről.

Következő évben puskát kaptam apámtól, ami nagy dolog volt. Tízéves voltam már akkor s kis kétcsövű húszaspuskámmal elindultam egy zegzugos, izgalmas, magányos, de csodaszép ösvényen, melyet azóta sem hagytam el soha.

Vadász lettem.

Első vadászélményem abból az időből való. Még háború volt s a felső udvarban orosz foglyok dolgoztak a cséplőgépnél. Szép meleg nyár s mi tele vendéggel, kik vígsággal lepték a házat, mint a verebek. Augusztus vége lehetett, azt hiszem. Szabad volt nyúllesre mennem. Tátival, öreg vadőrünkkel haladtam a dűlőúton, tarlók és langyos kukoricaföldek között. Vállamon cipeltem a puskát, mit apámtól kaptam. Kegyetlenül nehéz volt, majdnem földig ért.

Táti ment elől s én mögötte. Sohasem felejtem el azt az utat. Meleg volt s én trikók fölé szürke szövetkabátot vettem, mert a trikó fehér s attól félnek a nyulak. Emlékszem, azt a szövetkabátot unokabátyámtól kaptam, kinek szűk volt már. Minden ruhámat tőle kaptam majdnem abban az időben s ez jó volt, mert már akkor ruházatom gondja egyedül az én gondom kellett legyen, más nem sokat törődött azzal, hogy mit húzok magamra. Arra a szövetkabátra pedig különösen jól emlékszem, mert évekig hordtam még, míg egészen széjjel nem szakadt. Nagyon szerettem azt a kabátot. Öv is volt rajta és négy jókora zseb.

Szóval izzadtam a kabátban s vállamat nyomta a fegyverszíj. Izzadt kezem ahogy a fegyver csövét fogta, valami különös szag keletkezett. Sokszor találkoztam azóta is ezzel a szaggal. El is neveztem magamban nyári-fegyverszagnak. Talán az acél, a fegyverolaj, a napsugár s az izzadt kéz találkozásából keletkezik, igazán nem tudom, mert nem gondolkoztam rajta sohasem. De akkor éreztem életemben először azt a szagot. S akkor mentem először igazán vadászni, komoly nagy vadra. Nem szarkára, varjúra, verébre vagy patkányra, mint addig. Nyúlra. Igazi komoly nyúlra, igazi komoly puskával. Talán onnan van, hogy ez a szag úgy beleszívódott az emlékezetembe s valahányszor megüti az orromat, langyos táblák között kanyargó dűlőutat látok, aláhajló napot, zöld, fülledt, puha erdőszélet, ahova akkor mentünk.

Soha a világ olyan békés, szelíd és mégis titkos izgalmaktól fűtött nem volt, mint akkor. Már este nehezen aludtam el, amikor megmondták, hogy másnap nyúllesre mehetek. Reggel alig virradt, már kiszöktem az ágyból, pedig a les csak esti mulatság s estig sok idő volt. Takarítottam a puskámat, válogattam a töltények között s mindegyre a napot néztem, hogy mikor kezd már lefelé fordulni. Aztán az is eljött, a szelíd, békés délután. Álmosan zümmögött a cséplőgép s mi mentünk a kanyargó ösvényen, barna mezők között szótlanul s a világ nagy nyárvégi békessége összekeveredett bennem az izgalommal.

Áthaladtunk a falu melletti kicsi erdőn. Hűvös avarszag csapott az orromba, titokzatosan suttogott a cserjés, langyos, jószagú szellő járt s én izgatottan figyeltem be a lombok alá. Azóta sokszor voltam vadászni zöld augusztusi erdőn, langyos nyárvégi melegben. Ismerem jól a búcsúzó erdőlomb kesernyés szagát, a napszítta levelek összesúrlódásából keletkező furcsa zajt, mely olyan, mintha láthatatlan kezek motoznának szerte a homályban. De az az érzés, amit akkor érzetem, mikor először haladtam keresztül fegyverrel a vállamon az augusztusi erdőn, örökké nyomon kísér. Felködlik bennem minden augusztusban újra meg újra.

Öreg Táti előttem bandukolt, lassú nyugodt léptekkel. Meglehetősen lármásan ment, hátán a fényescsövű fegyver minduntalan odaütődött a nehézszagú, feketére zsírozott vadásztarisnyához s koccant a fényes rézkarikán. Jól emlékszem még ma is arra a tarisnyára. Láncok csörögtek rajta és tele volt mindenféle titokzatos szent holmival, olyan volt, mint egy varázsló csodatarisnyája. Évekig, egy egész évtizedig vágytam, hogy legyen nekem is olyan szurtos fekete vadásztáskám, tele mindenféle nagyszerű és titokzatosan szép dologgal. Ma már nekem is van olyan. De nincs benne semmi szép és titokzatos holmi s ez néha nagyon szomorú. Pedig van ott is minden, ami Táti tarisnyájában volt: töltény és bicska és spárga, golyóstöltény vaddisznóra és külön goromba sörét rókának. És van kutyalánc is rajta és kis lógó szíjacskák és éppen olyan szurtos szaga van, mint annak a másiknak volt. És mégsem ugyanaz. Évekig tűnődtem, hogy mi lehet az oka. Aztán egyszer megleptem a fiamat, amikor fogason lógó tarisnyámban turkált. És akkor egyszerre ráismertem, hogy ez az a tarisnya, éppen ez az, amelyik akkor Táti vállán fityegett. Mert úgy csillogott a fiam szeme éppen s úgy nézett a sok titokzatos holmira s a tarisnyára, mint és akkor, mikor azon az emlékezetes augusztusi estén az erdőn áthaladtunk.

Egyforma, lassú léptekkel ballagott Táti, kezét a fegyver csövén tartotta, lefele nyomva azt, hogy a puska piszkos fekete agya hátul fölfele állt s elém mutatta a rajta levő vésést: Bartha János 1908. Abban az évben született a puska is, akárcsak én.

Lábujjhegyen mentem mögötte, habár senki sem mondta, hogy úgy kell menni. De éreztem, hogy másképpen nem lehet. Azóta is mindig úgy járok az erdőn, ha nem is vadászom. S valahányszor valaki előttem jár erdei úton s nehéz léptekkel zörögve halad, ugyanazt a bosszúságot érzem, mint akkor régen, mikor Táti ment előttem botladozva s köhögve is közben.

Átmentünk a kis falusi erdőn s a tisztás másik oldalán be a Gyertyánosba. Újra erdő közt haladtunk s a fák alatt titokzatos homály kísért. A Gyertyános túlsó felén, hol a szántóföldek kezdődnek, egy görcsös szélfa alatt megálltunk. Ma már nincsen meg az a fa, mint ahogy egyetlenegy régi fa sincsen meg azon az erdőn. De azért a helyét tudom most is. A nap már eldűlt, az ég alját söpörte. Fölötte piros felhőpászmák pihentek mozdulatlanul, mint száradó vásznai dolgos angyaloknak. Alattunk sárga, aszott tarlók terültek el, lentebb zöld kukoricások s köztük keskeny kanyargó úton emberek haladtak hazafele, teheneket s juhokat terelve.

– Na itt leülhetünk – mondta Táti dörmögő vén hangján.

Helyet keresett a bokrok között, topogott, ágakat igazgatott, tördelt, hajlítgatott. Aztán leültünk. Elvette a puskámat, megtöltötte. Sóhajtott, köhentett nehányat, a szurtos tarisnyából pipát, dohányzacskót kotort elő, majd kovát, taplót s pipára gyújtott. Vastagon omlott a füst a pipából, a szellő elkapta játékosan s vitte a tarlók felé. Táti utána nézett súlyos figyelemmel, aztán dörmögve megszólalt s komolyan, mint egy kinyilatkoztatás.

– Ma lövünk nyulat. Jó a szél.

Némán figyeltem minden mozdulatát. Szent és titokzatos szertartásnak éreztem az egészet, az ágak félrehajlítását, a leülést, a fegyver megtöltését, a pipát s a füstöt ahogy a tarlók felé szállott, mint könnyű szürke fátyol s hirdette, hogy nyulat lövünk azon az estén. Csendben ültünk. Csupán a pipa szortyogott s a füst szállt lenge gomolyokban. Mozdulatlanul ültem a földön, csupasz térdeimen keresztbe fektettem a súlyos fegyvert és vártam.

Titokzatos szép este volt. Talán soha azóta nem volt olyan este több. A nap beszállt a Fundatura gömbölyű magfái közé, mint pihenni térő arany sasmadár. Halkultak a fények, a szemközti hegyoldal álmos-barnára vált s hátunk mögött az erdő fái közt sötét árnyékok kezdtek sűrűsödni. Halványodtak a felhők, az ég gyöngyszínű lett. Láttam egy csapat vadrucát átsuhogni a tarlók felett s a tóról odahallatszottak a békák. Mondom, csodálatosan szép este volt. Talán soha nem volt több olyan. A csend puha volt és tág, mint egy nagy átlátszó harang. Mint maga az égbolt, mely a mozdulatlanságot őrizte felettünk.

Szúnyogok gyűltek elő a bokrok közül, kárörvendő gonosz kis népei az erdőnek. Gúnyos hangjukkal körül zünnyögtek, arcomra, kezemre, csupasz lábszáramra szálltak s véresre kínoztak mindultalan. Mégis alig figyeltem őket. Megbénított a súlyos hallgatás, amit az alkonyodás az erdőre nyomott. Nehéz volt megmozdulni, borzasztó nehéz. Mintha ólomból lettek volna a karjaim, mintha láthatatlan szálak kötöztek volna le, mintha félnem kellett volna attól, hogy a mozdulat valamit összetör, vagy megvágom magamat a csendben. Csak néha simítottam végig a lábamon, vagy arcomon, pedig úgy kínoztak az apró bestiák, hogy a könnyem is majdnem kicsordult már. De Táti ilyenkor is rosszallólag tekintett reám s én mindannyiszor megszégyelltem magam.

Aztán élni kezdett mögöttünk az erdő. Ez csodálatosan izgalmas volt. Rejtelmes furcsa élet indult a hátam mögött, itt is, ott is moccant valami, apró ágak reccsentek a sötétben, halk moszatolás indult, megszűnt, újra kezdődött. Szorongattam a fegyver csövét s néztem mozdulatlan szemmel az összehajló fák közé, ahol már a sötétség volt az úr s a szívem vert, idegesen és dideregve. Azóta is nagyon ritkán vert úgy.

Kint egyre szürkébb lett minden. Hűvös szellő támadt s elterelte az alkony langyos sültalma-szagát. A távolodó dombok kékes párákat szőttek a völgy fölé. Két szürkevarjú jött s károgva alábillent a fák közt. Aztán sokáig nem mozdult semmi.

Nem tudom mennyi idő telhetett el így. Egyszerre csak éreztem, hogy Táti megérinti a vállamat. Felrezzentem, a szívem hevesen verni kezdett. Az öreg csontos ujjával az erdőszélre mutatott. Talán száz méterre tőlünk egy nyúl bukdácsolt. Lassan, kényelmesen billegette magát, olykor megült, egészen görbe lett, mint a púpos ember. Majd felegyenesedve figyelt, hosszú kanálfüleit ide-oda mozgatta, billegett nehányat s újra megült. Legelt egy kicsit, aztán játékosat ugrott s elbukdácsolt a tarlók felé.

– Álljunk fel – dörmögte rekedtes hangon az öreg.

Engedelmeskedtem. Kezem szorosan markolta a puskát, szemem a szürkületben nyúló tarlót szántogatta a nyúl után. Csend volt, semmisem mozdult. Mintha az egész világ megállt volna azon az estén.

Aztán egyszerre megláttam valamit az erdőszélén. Éppen ott, ahol a nyúl ült az előbb. Olyan csodálatos volt, olyan nagyszerű, hogy megbizsergett belé a hátgerincem s a szemem kerekre kitágult. Soha, soha nem fogom elfelejteni. Az erdő szélén fent, élesen és keményen felrajzolódva a halványsárga nyugati égre, egy őzbak állott.

Talán a vér is megdermedt bennem egy percre, olyan valószínűtlenül gyönyörű látvány volt. Sohasem láttam még őzbakot addig. Csak az őzikét ismertem, ami képeskönyvekben van, meg a mesékben. A kiszolgáltatott, szelíd, ártatlan, gyönge őzikét. De az ott nem őzike volt. Szívdobbantó, nagyszerű szobor volt, büszke és fejedelmi szobor, gőgös és megsemmisítő, mintha az erdők fejedelme állt volna ottan, a titokzatos, a bátor, a büszke, a legyőzhetetlen. Pompás vonalai élesen váltak ki az égből, fejét feltartotta, egyenesen, szikáran. Karcsú agancsa felnyársalta az első csillagot.

Sokáig állt ott mozdulatlanul. Nem tudom meddig. Már nem volt idő, már semmisem volt. Csak őt néztem s a lélegzetem is elakadt, talán a szívverésem is. Nem tudom meddig állt ott, meddig néztem úgy. Lehet, hogy csak egy másodperc volt az egész, lehet, hogy hosszú percek sorozata. Csak azt tudom, hogy egyszer hirtelen megfordult s magasra tartott fejjel lépett nehányat, büszke, királyi mozdulattal s eltűnt az erdőben.

Még ma is látom. Néha igazán nem tudom, valóban élő állat volt, vagy csak egy látomás. Azóta sokszor jártam ott s azon az erdőszélen mindig az eszembe jutott. Azóta sok-sok őzbakot láttam már, erdőszélen vágottban, sűrűségben, azóta sok-sok őzbak esett már össze véresen puskám előtt. De egy se volt ő. Egy se volt az igazi, a fejedelem, akit akkor láttam.

Csodálatos volt. Még ma is érzem az igézetét annak a percnek. Talán ezt az igézetet keresem azóta is, talán őt keresem azóta is, a csodálatos fejedelmi őzet, akit akkor láttam, vagy megálmodtam, most már igazán nem tudom megmondani. Talán ezért várom olyan sóvárogva mindig, hogy május zöldbe borítsa az erdőket s halk léptekkel járni lehessen halványzöld vágottak szélén, ezüstkék hajnalokon, barna estéken piros őzeket lesve. Azóta várom úgy mindig évről-évre a perzselő júliust, mikor sípommal megidézhetem rejtett lombsátorok között a bakot, titkos lovagját nyári erdőinek, a szépet, a karcsút, az előkelőt, az elvarázsolt mesekirályfit.

Sokat gondolkoztam ezen s valahogy úgy látom, mintha egy ilyen csodálatos őzbakot keresnék az életben is. Fájó izgalommal néha megiramodom egy gondolat után, eszme-bozótok vadonjában néha meglelek egy békés tisztást és ott lesben állva várok napokat, heteket, hónapokat, amíg a vadat sikerül kicsalni rejtekéről és kilép elém. És akkor látom, hogy nem az, nem az én őzbakom, csak éppen hasonlít hozzá. És akkor tovább megyek, új bozótokon és új tisztásokon át új eszmék, új célok, új vágyak nyomait keresve.

Sokat, sokat jártam már így, szomjas szenvedéllyel, sok-sok estén, sok-sok hajnalon. Hatvannégy őzbak agancsdíszét hoztam haza erdők rejtett mélyeiről a mai napig és talán tízszer, húszszor annyit engedtem békésen elvonulni mozdulatlan tisztelgő fegyverem előtt. De azt, az elsőt, az igazit soha, soha nem láttam többet. Pedig őt keresem, tudatosan, vagy tudat alatt, de őt keresem attól a naptól kezdve minden kóborló nyugtalanságommal, attól a csodálatos estétől kezdve, amikor ott álltam szédülten és földbegyökerezve a szürkületbe hanyatló erdő szélén s néztem merev szemekkel a sűrűségbe, ahol eltűnt a nagyszerű jelenség.

Csak akkor riadtam fel, mikor Táti megérintette a vállamat.

– Ott… ott… – mutatott a tarló felé.

Sok idő telt, amíg magamhoz tértem s megláttam a nyulat. Táti bosszankodott is ügyetlenségemen, szinte hangosan mondta már, hogy a két vakondtúrás között nézzem, hiszen nincsen is messze. Mit jelentett nekem már akkor az a nyúl! Szegény öreg Táti hogyan is érthette volna. Felemeltem a puskát, kicsit céloztam, de nem is egészen komolyan, aztán elsütöttem. Hideg voltam és nyugodt, mintha nem is első nyulamról lett volna szó. A kezem sem reszketett, a szívem sem vert jobban mint máskor.

Akkor nem gondolkoztam rajta, talán észre sem vettem. De ma már tudom: az őzbak okozta ezt. A titokzatos őzbak, aki egy pillanatra megmutatta magát, hogy örökre eltűnjön azután. Már akkor nem élt bennem semmi vágy és semmi szenvedély többé, csak ő. Mint azóta is, mint most is. Hideg a kezem és nyugodt, mikor emelem a puskát rókára, vadmacskára, disznóra, szarvasra, akármire. De ha őzbakot látok valahol messze a vágottban állni, vagy halkan lopódzni a sípom felé a sűrűségen át: reszket a kezem, amint a látcsövet szememhez emeli s a szívem úgy kalapál, szinte kiugrik. Olyan vagyok olyankor, mint a reménytelen szerelmes, ki eltűnt kedvesét keresi. Mint a kincskereső, ki csillanni lát a fövenyben valamit. Mint babonás alchimista, kinek lombik-tüzében egy pillanatra megjelent a bűvös szalamandra. Aztán kiderül, hogy egyik sem ő, egyik sem az igazi. Szomorúan állok az erdővel szemben s az erdő titkos suttogással hív tovább, új ösvények, új tisztások felé.

Furcsa dolog ez. S furcsa volt azon az estén is, nagyon furcsa. Eldörrent a lövés, a fegyver megrúgott kegyetlenül. Sajgott a vállam. Füst lepte el az erdőszélet, tarló, vakondtúrások, nyúl, minden eltűnt. A dörrenés még visszaütődött a szemközti oldalról, aztán belefulladt a semmibe. A csönd, mint híg iszap, újra összezárult.

Táti mozdult meg először.

– Jól van – mondta hangosan, rekedten s a hangja engem is felsebzett, akár a néma erdőt. – Jól van, fekszik a nyúl!

Két vakondtúrás között valóban ott feküdt a nyúl, fehér hasa belevillant puhán a szürkületbe. Odamentem hozzá. Nem futottam, pedig úgy illett volna. Csak mentem s Táti mögöttem jött. A nyúl nem élt már, mire odaértünk. Szürkén, puhán, halva feküdt ott, alig látszott. Olyan volt éppen, mintha kicsike kölyke lett volna a reánk boruló roppant szürkületnek.

A nyúl mellett álltam, de nem örvendtem. Nem is a nyulat néztem. Fölfele néztem újra, az erdősarokra, ahol az őzbak állott. Üres, borzasztóan üres volt az erdősarok. De fent az égen, éppen ott, ahol ő felszegte a fejét, két fényes csillag égett. Olyan volt, mintha agancsa két hegyével átdöfte volna az alkony peremét.

Aztán mentünk hazafele. Szó nem esett. Én nem kérdeztem Tátit, hogy látta-e ő is az őzbakot s ő nem szólott róla. Nem is beszéltem erről soha senkinek. Sötét volt az erdő és titokzatos. Táti ment elől, pipázva. Puskája csövén csillant a felkelő hold, mely a fák ágai között akkor bukkant föl éppen kövéren és sárgán.

Én hátul mentem. Balvállamat húzta a fegyver, jobb kezemet a nyúl zsibbasztotta. Nehéz vén nyúl volt. Tenyeremet bevérezte, markomat görcsbe fárasztotta. Fájt a karom tőle, olyan nehéz volt. Összeharaptam a számat és mentem. Hosszú volt az út. Ötször, tízszer, százszor olyan hosszú, mint jövet. Néha megálltam és letettem a nyulat. Ilyenkor nehány lépés után Táti is megállt és visszanézett. S dünnyögte, hamiskásan, a bajusza alatt:

– Ki mit lőtt, hozza haza. Ez a rendje a vadászatnak.

Szegény öreg Táti, ki ma már alszol békés álmokat ott túl a szentegyedi szőlőben, köszönöm Neked, hogy azt a nyulat haza vitted velem akkor. Hogy nem vetted át, hogy nem segítettél. Hogy hagytál izzadni, gyötrődni, fájó vállal, elzsibbadt karral, görcsbe ránduló ujjakkal. Hogy haza cipeltetted velem a Gyertyános túlsó végiből azt az első nyulat. Pedig magamban szidtalak akkor, de most köszönöm Neked, öreg Táti, nyugodjál békében. Te faragtál vadászt belőlem, azon az egyetlen estén, azon a fárasztó, hosszú, kegyetlen estén a Gyertyános és a konyhaajtó között, ahol a nyulat végleg letehettem s odaülhettem melléje én is a küszöbre. Elcsigázva, kifáradva, de komor büszkeséggel, igazi férfias vadászbüszkeséggel a szívemben.

Mert vallom veled együtt attól a naptól kezdve: csak akkor öröm a siker, ha megfáradtunk érte. Nem az a vadász, aki halomra lövi a vadat és tűri, hogy más cipelje azt s más bajlódjon vele. Õ azalatt egyébre gondol, talán már új vadat keres, amit megölhessen. Nem az a vadász. A vadász egy vad után jár. Egy vad után, egy bizonyos vad után, amit jól ismer, vagy nem ismer, csak tudja, hogy van, hogy kell legyen valahol. Aztán elejti. Aztán leül melléje. Pipára gyújt, vagy csak úgy elnézi az erdőt s gondol egyszerű és szép gondolatokat. Aztán elrendezi a vadat úgy, ahogy illik. Vállára veszi és hazaviszi. Mert az ember nem ragadozó állat. Az ember, ha vadász, az élményt keresi, nem a gyilkolást. Köszönöm Neked, öreg Bartha János, hogy megtanítottál erre. Sokszor megköszöntem már ezt neked. Forró nyári délelőttökön, az Andrenyásza pojánáin, vagy a Lisztes alatt, amikor nagyszerű és egészséges érzéssel dobtam le vállamról az elhozott őzbakot a kalyiba elé s véres ingemet levetve megmosdottam a patak jéghideg vizében, érezve izmaimban a fáradt erőt s valami furcsa, szép vadászbüszkeséget. Te tanítottad meg nekem ezt az érzést, amikor a sikert az izmok fáradságán keresztül érzi az ember s azt a komoly, pompás büszkeséget, amit annyiszor éreztem azóta.

Akkor még nem tudtam, mi az, amit érzek. Fáradt voltam, kegyetlenül fáradt. Ültem egy keveset a küszöbön, hallottam, hogy a konyhában dicsérnek engem és a nyulat s ez jólesett. Aztán megmostam a kezemet és bementem a házba. Éppen akkor vitték a vacsorát. Mindenki ült már a helyén, amikor beléptem. Szótlanul a helyemre mentem. Senkise nézett rám s az érdeklődés nem is hiányzott. Természetes volt, hogy a nyúl, amit haza hoztam s aminek a súlya ott égett a karomban, az én ügyem egészen. Éhes voltam. Jóízűen, boldogan ettem. Vacsora vége felé az egyik vendégnek eszébe jutottam.

– Na! Hát mi volt a lesen?

– Lőttem egy nyulat.

Ennyi volt az egész. Mindenki lelkesedett akkor és csodálkozott. Még ma is mondják, kik visszaemlékeznek arra a vacsorára, hogy úgy látszott, az egész nyúl-ügy nem volt rám semmi hatással. Lőttem egy nyulat – mondtam halkan és közömbösen, mintha már a századik lett volna legalább.

Furcsa dolog ez nagyon, hogyan is magyarázzam. A nyúl az én ügyem volt. Én hoztam haza. Nem csupán meglőttem: megdolgoztam érte, megkínlódtam érte. Izzadva cipeltem haza a dicsőségemet, nyöszörögve és összeharapott fogakkal. Azért volt az a dicsőség olyan komoly, olyan igazi férfi-dolog, hogy nem lehetett ujjongva dicsekedni vele. Talán ha kocsin hoztak volna haza, ha a nyulat nem is kellett volna a kezembe vegyem, lehet, hogy hencegésbe csúcsosodott volna ki a bennem felgyülemlett izgalom. De az izgalmat levezette derekasan a fáradság és ez így különben is más volt. Komoly vadász-dolog, férfi-ügy s házunk kényelmes léha életétől nagyon távolálló.

Így volt az, mikor az első nyulat lőttem. Fáradt voltam és jól aludtam utána. Álmomban nagy sötét erdőket láttam – azóta is gyakran álmodom ilyent – nagy súlyos csendet hallottam és föld és ég között mozdulatlanul, mint pompás kőszobor állt a fejedelmi őzbak és csillagokat bökdösött az égbe.  

Attól a naptól kezdtem járni az erdőt. Kiültem esténként az erdőszélre, most már Táti nélkül, egymagamban s mindig ugyanazon a helyen, ahova először mentünk. Kucorogtam a bokrok között és lestem a szürkületet, mely lassan és méltóságteljesen közeledett felém olyankor. Szemem makacsul megakadt az erdőszélen, ott ahol akkor a titokzatos őzbak állott és vártam elszánt kitartással, néha bele a késő éjszakába, hogy újra megjelenjen. De nem jött többet. Nem jött többet és ez egyre jobban elszomorított. Pedig nem is tudom, miért vártam úgy rá, hiszen nem volt szabad őzre lőnöm akkor, csak nyúlra és fácánra.

Emlékszem, nehány egyre szomorodó őszi estén még ott dideregtem a görcsös fa alatt és vártam az őzbakot. Aztán nem mentem többet. Jött az ősz és beteg leheletétől a fák elsápadtak. Árva és magányos lett a kertünk, akárcsak én. Az ősz szomorkás hangulata mindig reám nehezedett. Igyekeztem még többet egyedül lenni és ennek nem volt semmi akadálya. Álldogáltam naphosszat a fák alatt, melyekről barnafoltos levelek hullottak csöndesen s játszottam magamban szomorú meséket, melyek azzal végződtek mindig, hogy meghalok.

Vidámság töltötte színültig apámék házát azon az őszön. Emlékszem, annyi kacagás és annyi léha tréfa gyűlt össze benne, hogy szinte szétfeszítette a falakat. Pedig Magyarország búcsúzott tőlünk azon az őszön. Pedig olyan súlyos és komoly idők voltak, amilyenek egy nemzet életében ritkán adódnak. Vihartól terhes fekete fellegek ülték kereken az erdélyi égbolt peremét s a pusztulás már ott sírt körülöttünk az őszben.

De nálunk lármás nagyvadászatok voltak s az élet középpontja abban rejtőzködött, hogy ki hogyan hibázta el a fácánt, ki lett hamarabb részeg a vendégek közül, borkóstolás közben.

Csak a falu volt néma és sötét. Hallgattak a román parasztok s ha kocsink elhaladt közöttük, sokan félrenéztek s nem emelték kezüket köszöntésre. Egy emlékem van ebből az időből, vad és döbbenetes emlék.

A falu hátsó utcáján jöttem fölfele. Menyecske mögöttem baktatott s az egyik ház kapujában meglátott egy macskát. Vakkantott egyet s eliramodott. A macska befutott a parasztudvarra s egy eperfára mászott. Menyecske ugrált a fa alatt s éles tacskóhangon ugatta. Én a kapuig mentem s onnan hívtam a kutyát.

Akkor egyszerre kicsapódott a ház ajtaja s egy gyűröttarcú, alacsony román legény ugrott ki rajta. Káromkodott s fejszét dobott Menyecske felé. A kis sárga tacskó ügyesen félreugrott a fejsze elől s behúzott farokkal trappolt vissza hozzám. Akkor látott meg engem a legény. Elvicsorodott s ordítani kezdett. A fal mellől felkapott egy téglát s felém hajította. A tégla egy félméterrel repült el mellettem. Ocsmány és csúnya szavakat kiabált a legény, tüszőjéből kirántott egy bicskát s megindult felém.

Megfordultam s visszaléptem az útra. Nem néztem hátra, de hallottam a legényt mögöttem kiabálni, szitkai végigcsattogtak az utcán. Aztán egy újabb tégla repült el mellettem.

Lábaim szaladni szerettek volna, akkora volt bennük az ijedség. De csak nyugodt lépésekkel mentem tovább fölfele az utcán, pedig a szívem zakatolt a félelemtől. A szomszédos házak kapuin kíváncsi fejek jelentek meg s néztek engem, meg hallgatták a legényt, ahogy szidott s vigyorogtak. S abban a percben megéreztem a háborút. A háborút, melyik tűzbe borítja kereken a világot s az egyik embert a másik ellenségévé teszi. A háborút, amelyik ott szunnyadt már akkor a falunkban is, csak apámék nem akartak tudni róla.

Mentem fölfele a falu utcáján, a szitkok szavai s gúnyos arcok között s abban a pillanatban éreztem meg, hogy én ebben a háborúban egyedül vagyok. Egyedül vagyok, mert nem tartozom oda sem, ahol nevetnek s oda sem, ahol káromkodnak. Tízéves voltam. S a ketté szakadt világ két ellenséges arca között álltam, megriadva és olyan egyedül, mint gyönge gyertyaláng a hóviharos éjben.

Nem beszéltem erről sem senkinek, a mai napig. Hordoztam magamban s a lelkem nehéz lett tőle sokszor. Víg estéken, mint kísértetes látomás, még ma is vissza-visszatér s én örülök neki, mert emlékeztet arra, hogy háború van s azt elvégezni csak békességgel lehet.

Néha engem is elvittek vadászni apámék, bár nem nagyon szerettem ezeket a lármás vadászatokat, melyek állandó kiabálásból álltak: menj ide, menj oda. Jobban szerettem s jobban szeretek ma is egyedül járni a vad után. De sok volt nálunk akkoriban a nyúl, meg a fácán s ezek a vadászatok vidám mulatságok lehettek azok számára, kik társaságnak élnek.

Nekem egyszer bajom volt régen a fácánok miatt s ezt az ostoba, gőgös madarat őszintén gyűlöltem azóta. Ez még régebben volt, talán hat, vagy hétéves lehettem. Egy nyáron történt. Barátommal, a kocsis fiával lent kószáltunk a faluban. Nagy megrakott szénásszekerek jöttek sorjában egymás mögött, apám szekerei voltak, hozták haza a rétről a szénát. Felkérezkedtünk az egyik szekérre. A béres, egy jókedvű barna ember, odatartotta elénk a kalapját. Tojások voltak benne. Szép hosszúkás, halványzöld tojások.

– Mifélék ezek? – kérdeztem tőle.

– Varjútojások – mondotta nevetve a béres –tessék, vigyék haza!

Nekünk adta a tojásokat. Nekem nem volt kalapom, amibe tehettük volna, de volt a barátomnak, így hát abba öntöttük át őket. Volt vagy egy tucat. Nagy dolog egy tucat szép tojás! Mentünk, mentünk a szénásszekér tetején. Hintázott velünk a magas, puha széna, nagyon jó volt ott ülni fent és nézni a világba. Gondolataim mindegyre visszatértek a tojásokhoz.

– Vajon biztos, hogy varjútojások?

– Persze hogy biztos. Nyikuláj mondta.

Úgy van. Nyikuláj mondta, ez súlyos érv volt.

Felértünk lassan a majorba. Ügyelve szálltunk le, nehogy összetörjenek a tojások. Az istálló sarkánál leültünk, kivettük őket a kalapból és sorbaraktuk a homokba.

– Mit csináljunk velük?

Hosszan tűnődtünk ezen. Eszembe jutott, hogy ki lehetne költeni őket, vagy rántottát enni belőlük. Aztán barátom fejében megszületett az ötlet.

– Nézzük meg bennük a csirkét!

Félelmetesen szép ötlet volt. Azonnal csatlakoztam hozzá. Lehet-e ennél nagyszerűbb felfedezés egy kíváncsi hatéves gyerek számára?

Nekifogtunk a kutató munkának. Hermann Ottó és társai valószínűleg más módszert követtek az ornithologia eme kérdésének tisztázása közben. De mi csak úgy egyszerűen vizsgálgattuk a tojások belsejét. Odavagdostuk őket az istálló falához.

Hát csirke volt mindegyikben, az tény. Megtalálhattuk fejét, lábát. De a tojások dolga sajnos ezzel még nem végződött be. Mert délután éppen a ház előtt játszottam, mikor Apám nagy léptekkel jött lefele az istállóktól. A kocsi befogva állt az ajtó előtt, készült valahova. De látni lehetett mindjárt, hogy nem azért siet úgy. Vörös volt a nyaka meg az arca, nagyokat lépett s állát behúzta, fejét előre dütötte, ami gonosz dolgokat jelentett. S jött egyenesen felém.

– Úrfi! Gyere csak ide! Mért törted össze a fácántojásokat?!

Úgy ordított, hogy a szakácsné ijedten bujt elő a konyhából s az inas is az ablakhoz szaladt, hogy lássa, mi lesz velem. Nem voltam bátor gyerek. Sőt voltak dolgok, amiktől határozottan féltem. Mint például a ricinustól, a doktortól és a bivalytól. De Apámtól, amikor így rámordított, olyan borzasztóan féltem, mint semmi egyébtől a világon. Mert úgy ordított olyankor, mintha ott abban a percben meg akart volna enni. Annyira féltem ettől az ordítástól, hogy ha eszembe jut, még most is félek tőle.

Tehát Apám ordított, aztán kivette a kocsis kezéből az ostort és mivel én nem mente feléje, ő közeledett hozzám.

– Adok én neked fácántojást! – ordította többször egymásután és minden szónál jót húzott rajtam az ostorral. Égett az ütés, akár az eleven tűz. Az arcomon, kezemen, lábamon, az egész testemen, ahol csak ért a cudar szíj-korbács.

– Adok én neked fácántojást! Adok én neked fácántojást!

Menekültem volna, de nem volt hova. Mögöttem sűrű bokor, abba fúrtam bele magamat kétségbeesetten és bőgtem, ahogy a torkomon kifért. Talán még ma is ott lennénk, Apám meg én, ha elő nem rohannak a cselédek és oda nem vetik magukat közénk. Apám leszidta őket, káromkodott, aztán felpattant a kocsira és elment. Így végződött első madártani kutatásom.

Az egész dolog azonban nem is olyan egyszerű, mint ahogyan első percben gondolná az ember. Én széttörtem egy tucat drága fácántojást, amit a kaszások szedtek össze a réten s küldtek föl a béressel, hogy a fácánosban kotló alá tegyék a vadőrök. Apám jogosan haragudott, mivel sok gondja-baja volt a fácánokkal úgy is. Haragudott s mivel ő haragjában mindenkit megvert, aki alája tartozott, tehát elpántolt engem is jól és alaposan. Az ő részéről az ügy ezzel az igazság szabályai szerint elintéződött.

Én azonban nem fácántojásokat vagdostam a falhoz, hanem varjútojásokat és azokat a falhoz verni szabad, ez a világmindenség teremtése óta fönnálló gyermeki jog. Mert hiába tojta fácánmadár azokat a tojásokat, nekem azok varjútojások voltak, mivel azoknak tudtam őket. Számomra nem is lehettek mások, csak varjútojások s azok még ma is, mivel azzal a tudattal mázoltam őket az istálló falához.

Rájöttem azóta, hogy az életben sokszor van így. Ellentétes igazságok korszakát éljük. Ami az egyik embernek fácántojás, az a másiknak varjútojás. Ami az egyik népnek varjútojás, a másiknak fácántojás. Fácántojás és varjútojás: a különböző szempontok és propagandák tojásai, azóta is számtalanszor találkoztam velük.

Apám, hogy megvert akkor, igen jól tette. Ez volt számomra az élet első komoly leckéje. Hogy úgy is lehet – sőt hányszor lehet! – hogy az embernek igazsága van és mégsincsen igazsága. És hogy hallgatni kell. És meg kell gondolni mindig, hogy nem csak nekem lehet igazam, másnak is lehet. Ugyanaz a tojás lehet nekem varjútojás, másnak fácántojás. S ha én tudom, hogy valóban lehet egy és ugyanazon tojás kétféle egyszerre, akkor én okosabb vagyok, mint az, aki ezt nem tudja és ha okosabb vagyok, akkor én kell tűrjek és hallgassak. Mert olyan a világ, hogy aki okos, az tűr és hallgat.

Annak a verésnek a nyoma sokáig meglátszott rajtam. De megmenekültem egyúttal az élet hasonló többi veréstől, mert megtanultam, hogy a fácántojás is igazság s a varjútojás is az. S nem haragszom soha az emberekre, ha másképpen vallják a dolgokat mint én, mert tudom, hogy az ő igazságuk is éppen olyan igazság a maguk szempontjából, akár az enyém. Hogy pedig azok a tojások valóban fácántól, vagy varjútól valók, azt eldönti a jövő, mikor kikelnek a csirkék. S ha falhoz verik őket, akkor úgyis mindegy. Nekem lehet varjú, másnak lehet fácán, nem ártunk egymásnak vele. Az élet sok apró igazsága között talán nem is az igazság a fontos. Hanem a békesség, mellyel megszorítjuk egymás kezét az igazság fölött.

Mindez újra eszembe jutott azon az őszön, mikor a nagy vígság folyt nálunk s a falu olyan sötéten hallgatta. A vígságnak aztán hirtelen vége lett. A fenyegető sötét fellegek összeértek fölöttünk s mi beköltöztünk előle Kolozsvárra.

Nem nagyon értettem akkoriban, hogy mi megy végbe a szemem előtt. Csak az láttam, hogy hazajönnek a katonák, éjszakánként hallottam lövöldözni őket, a városháza előtt néha csődület volt s az erkélyről beszélt valaki. Aztán egy nappal karácsony előtt idegen katonák haladtak végig az utcán. Rongyosak voltak és piszkosak, nem is sorban mentek, csak úgy rendetlenül. Némelyiknek bocskor volt a lábán, némelyiknek csizma. Volt köztük sok szalmakalapos is és rengeteg kucsmás. Azt mondták: román katonák.

Az angyal nem jött abban az évben, szomorú karácsonyunk volt. Az utcákon idegesség és félelem szaladgált. Feltűzött szuronyú őrjáratok lépték egymás nyomát. Néha lövés hasított az éjszakába.

Emlékszem, január tizenkilencedikén történt. Magyarok vonultak fel a Főtéren, néma csapatban s vitték a magyar zászlót. A New York szálloda elé mentek s ott megálltak. Mondták, valami francia generális lakik ottan, aki a békét igazítja s ahhoz mentek küldöttségben a magyar zászlóval.

Kifutottam én is az utcára. Nagy tömeg verődött össze a Főtéren, nekem már csak egészen hátul jutott hely, a Mátyás szobor előtt. Felmásztam a szobor alapzatára s onnan néztem át a tömeg feje fölött. Láttam a szép magyar zászlót egészen elől s kiabálást is hallottam, de a szavakat nem tudtam megérteni. Aztán láttam, hogy a zászló körül kavarogni kezdenek az emberek, mintha dulakodnának. Ordítozás hallatszott, a zászló billegni kezdett, majd megingott s bevágódott a többi közé. Ez volt az utolsó magyar zászló, amit láttunk.

Katonák szaladtak elé a mellékutcákból láttam, ahogy kézbe kapták fegyvereiket, aztán rettentő roppanás hallatszott, eldördültek a puskák. A tömeg megbomlott. Ordítozás és kavargás kezdődött és egyre ropogtak a puskák. Ablakok nyíltak meg, karok és fegyverek nyúltak ki az ablakokon és lőni kezdtek reánk.

Egy pillanatig döbbenve álltam még Mátyás király előtt, aztán leugrottam a kőpárkányról s megiramodtam hazafelé. Láttam már nyulat agár előtt az életéért futni, de az semmi volt ahhoz képest, ahogy én futottam akkor. Ha még egyszer az életben véghez tudnám vinni azt a futást, biztos, hogy világbajnokságot szereznék vele. Emlékszem, egy induló kocsin keresztül ugrottam s még most is látom magam előtt annak a két benne ülőnek ijedt arcát, akiknek a térde fölött átrepültem a kocsin. Futottam, mint a patkány, akinek ég a farka. Körülöttem a járdán kis fényes szikra-tölcsérek pattantak, olyanok éppen, mint erős nyári záporban. Lövések csattantak mindenütt s golyók süvítették a levegőt keresztül-kasul.

S ahogy ott futottam a sok rohanó ember közt egérutat keresve, megláttam egy román katonatisztet. Kocsiban állt, kezében pisztoly és egyenként célozta meg a mellette elszaladó embereket. Gondosan célzott s azután lőtt. Valaki feljajdult. Újra célzott. Újra lőtt. Valaki elesett.

Csak futó pillantást vethettem reá, de arcát örökre megjegyeztem. Sovány, gúnyos arca volt, torz ráncokat rajzolt szája köré a gyűlölet. Csak állt a kocsiban és egyre tüzelt a védtelen, szaladó emberek közé. A magyarok közé. Azért, mert magyarok voltak.

Soha, soha nem felejtettem el ezt a napot. A zászlót, ahogy megingott és eldőlt. A félelmet, ami lábizmaimba gyűlt s öles ugrásokkal vitte életemet a biztos menedék felé. A golyókat, ahogy körülöttem búgtak és süvítettek s szikrázva pattantak vissza a járdán. S a tisztet, ahogy állt nyugodtan és célozgatott, közénk, mint ahogy hajtóvadászaton a vadász célozgatja a mellette elrohanó vadat. Nem felejtettem el ezeket soha. És sokszor, ha ítélnem kell románok dolgában, ott állnak mellettem ezek a képek s nehézzé teszik küzdelmemet az igazsággal.

Abból a házból, amelyikben laktam, két embert lőttek meg azon a napon. Tanáromat s egy öregasszonyt. S az a tiszt lőtte le mind a kettőt aki nem messze házunktól a kocsiban állt s vadászott magyarokra.

Ilyenek történtek azon a télen s a vígság meghalt apámék házában. Mikor hazakerültünk, csönd, panasz és árvaság volt otthon. Elkezdődött az ünneptelen nyomorúság, aminek huszonkét évig nem szakadt vége.

Csúnya idők, nem szívesen emlékszem vissza rájuk. Idegenek jöttek-mentek nálunk, mindig hoztak valami kellemetlenséget és mindig elvittek magukkal valamit. Néha apámat vitték el. Hosszú idő telt el így, a gond, baj és félelem jegyében. Ezekben az időkben még a vakációk is csak gunnyasztó vén varjakhoz hasonlítottak, melyeknek fakó tolla közül aggodalom és szomorúság hulldogál.

Elvitték a puskákat is, nem volt vadászat. Öreg Táti fegyver nélkül döcögött előre-hátra s rosszkedvűen dörmögte a bajusza alatt.

– Megette a fene a világot…

Hát olyan is volt a világ akkoriban, mintha valaki megette volna. Sok idő telt el így s aztán apám egyszer két puskát hozott haza. Engedélye is volt hozzájuk. Nagy dolog volt. Lassan és szerényen, de elkezdődtek a vadászatok megint. Kis idő múlva Táti vállán is megláttam újra a régi fényescsövű puskát.

– Hogyan szerezte vissza? – kérdeztem tőle megörvendve.

– Megvót – mondotta titokzatosan.

– Hát engedély is van? – firtattam tovább.

Táti belemosolyodott a bajuszába, aztán bölcsen legyintett.

– Nem engedély köll ide.

– Hát mi?

– Egy kis pálinka az őrmesternek.

Így kezdtünk ismerkedni Romániával.

Nemsokára nekem is jutott fegyver s vadászidő alatt esténként öreg Tátival patront készítgettünk. Mértük a puskaport, sertét, dömöcköltük a fojtást. Kalandos, izgató vadászatok voltak ezek, ha csendőrt láttunk, behúzódtunk a nádba, vagy a bokrok közé. Ködös hajnalokon keltünk útra, láncon vittük a kopókat, tarisnyában az élelmet s egy-egy nyúlért, rókáért estig szaladgáltunk csavaros erdei utakon a kopók hajtása után.

A bakról, arról a bizonyos őzbakról már egészen megfeledkeztem. Néha láttam őket szaladni lesunyított fejjel a kopók előtt, de azok nagy, ijedt, szürke nyulak voltak, nem is hasonlítottak ahhoz a régi őzbakhoz s így nem is emlékeztettek rá.

Egy nyári vakáción találkoztam vele megint. Elvadult kamasz voltam már akkor, lelkem csupa zavaros érzéssel tele, melyekről magam sem tudtam, hogy micsodák. Elszoktam a rajongástól, eszményeim nem voltak. Sokat voltam azokban az időkben olyan társaságban, amely nem hitt az eszményekben. Kigúnyolt mindent, amit könyvekből szépnek, jónak, nemesnek ismertem meg s ezt az eszményekkel való gúnyolódást kezdtem megtanulni. Ez a társaság szerette a nagy embereket apró hibákkal lekicsinyíteni, véletlennek tüntetni fel a nagy tetteket s eszmények hordozóit haszonlesőknek bélyegezni. Abban az időben sokat voltunk zsidókkal együtt s bár nem vagyok gyűlölője semmiféle fajnak, meg kell állapítsam, hogy a zsidók oltották belénk ezt a minden nagyságot lerántó, minden szépséget kigúnyoló szenvedélyt. Nem jöhetett szóba Petőfi úgy, hogy valaki meg ne állapítsa: sze nem is volt magyar. Ady részeges disznó. Kossuth csirkefogó. S az én kamasz-lelkem nagyon fogékony volt a romboló irányra. Restellem, de bizony nem egyszer gúnyoltuk az istentiszteletet s papunkat szegényt.

Azért mondom el mindezt, hogy látni lehessen: az irány, amerre haladtam cinikus, léha lelki pusztulás felé vitt akkor. S hogy mégis megmentődtem, azt a titokzatos őzbaknak köszönhettem egyedül.

Szép meleg július volt. Galambot lőni mentem az erdőre. A Gyertyános tisztásán üldögéltem, tikkadt szellő borzolgatta a fák álmos leveleit s a legelőről behallatszott a tücsökök zsongása. Elhevertem a füvön, lustán és álmosan és nem gondoltam semmire.

Egyszerre csak éreztem, hogy valami van a tisztáson. Ösztönös érzés volt, valami furcsa erdőhöz tartozó tulajdonság. Füleltem. És akkor éppen szemközt velem megláttam újra, most már másodszor életemben, a bakot. Ott állt, nehány lépésre a sűrűség szélétől, élénk sárga színével pompásan kiválva a sötétzöld háttérből, fejét magasan tartotta s nézett engem. Nem mozdult, csak nézett. Olyan volt, mint egy gyönyörű szobor. Szép gyöngyös agancsán megszámlálhattam az ágakat. Állt és nézett.

Én is néztem őt. Nem mozdultam, nem tudtam mozdulni. A meglepetés varázsa tartott mindkettőnket fogva. Nem tudom mennyi ideig néztük így egymást. Végül a bak lökött egyet magán, mélyen, haragosan böffent s beugrott a sűrűbe.

Ennyi volt az egész. De attól a perctől kezdve nem volt többet nyugtom otthon. Csábíthatott a tó s a fürdőzők lármája, puffanhatott tenisz-pályákon a labda, recseghetett a gramofon: engem már nem érdekelt. Az őzbak vonzott, a nagyszerű őzbak, amelyiket már kétszer láttam megjelenni hirtelen és váratlanul s mely olyan volt mégis, mintha mese lett volna.

Azon a nyáron nem találkoztam vele többet. De nap-nap után jártam az erdőt s a fák újra barátaim lettek, mint évekkel azelőtt. Megtanultam az erdő életét s ez az élet szebb volt, mint minden egyéb élet a földön, szebb, őszintébb, tisztább, igazabb. Komoly és szép gondolatokat adott az erdő s a nagy léha üresség, mely a lelkemet már-már uralni kezdte, megtelt újra szép és jó érzésekkel, amiket az erdőtől kaptam s a csendtől, amely úgy áradt szét az erdő fölött, mint Isten meleg lehelete.

Attól kezdve már beszélhettek otthon cinikus szavakat. Folyhatott az eszmény-rombolás szerte a világban kedve szerint: engem megmentett az erdő. Megtanított igaznak látni a szépet és a jót. Megtanított arra is, hogy a gúny, amely a szépet támadja, nem egyéb, mint irigy takarása emberi mesterkedésekből származó világoknak, melyekből hiányzik a szépség.

Sok-sok estén keresztül jártam az erdő tisztásait, harmatos hajnalokon lopództam vágottak szélén, éjszakáztam koronás fák alatt egy szál kabáttal takarózva s rágicsáltam zsebben hozott kenyeret. Néha láttam legelésző piros őzeket, sutát és gidót, nyársas és villás bakokat. Néha eső után megtaláltam az ő nyomát is és néha hallottam csillagos hajnalokon böffentve megugrani. Maga volt a titokzatosság, ott volt és mégsem volt ott, járt vágottakban és tisztásokon és nem lehetett meglátni soha.

Néha eszembe jutott a régi legendás csodaszarvas, Hunor és Magor meséje. Talán így csalta őket is az a csodaszarvas, mint engem az őzbak. Õket új hazába vitte. Azt hiszem engem is.

Aztán jött egy este. Fülledt augusztus eleje volt. Szellő se moccant, a fák súlyos zöld ágaikat tikkadt fáradtsággal hajtották le a földig. Egy tisztáson cserkésztem át. Lemenőben volt a nap, sárga és vörös sugarakat vetett a pihenő erdőre. Valahol egy gerle burukkolva ringatta a csöndet.

Éppen kiléptem egy bokor mögül, mikor dobogást hallottam. Megfordultam. És abban a pillanatban kővé meredtem az izgalomtól. A cserjésből hosszú szökésekkel egy suta rohant elő s mögötte nyitott szájjal jött a bak maga! Gyönyörű kép volt. Az üzekedő suta kecses ugrásai s a bak, ahogy űzte, elvakulva és szenvedélyesen, vágyakozva és haragosan szívbénító gyönyörű látvány volt.

Nehány lépésre tőlem futott el a suta. És akkor véletlenül felém nézett és meglátott. Nagyot zökkent az ugrása, megállt. Meredten, felém hegyezett fülekkel állt, szemei riadt-kerekre nyíltak.

Megállt a bak is. Nem látott engem, de megállt, mert érezte, hogy valami nincsen rendben. Magasra tartotta a fejét, figyelt.

Megmozdultam, lassan emeltem a puskát. A suta riasztva böffent egyet s nehány ugrással a sűrűben volt.

De a bak nem futott el. Csak nézett a suta után, felemelt fejjel, aztán kereken hordozta fejét a tisztáson. Kicsit idegesen, de büszkén, mint aki tudja, hogy nincs oka félni. Akkor meglátott engem. Egyenesen a szemembe nézett. Felemelt fejjel, merev tartással, egyenesen a szemembe nézett. Nem felejtem el soha ezt a nézést. Dac és büszkeség volt a szemeiben, megvetés és egy kis riadt csodálkozás. Olyan gyönyörű volt és olyan hatalmas, hogy a szívem összeszorult tőle.

A puska már vállamnál volt s ujjam a ravaszon. Még egyszer végignéztem pompás fejedelmi alakján a fegyvercső fölött, aztán eldörrent a lövés. Borzasztó nagyot szólt. Mintha kettéhasította volna a fejemet, az erdőt, mindent. Az őzbak gyorsat és nagyot ugrott, aztán összeesett.

Nehány másodpercig mozdulatlanul álltam és lihegtem az izgalomtól. Éreztem valamit, de az nem öröm volt. Valami szenvedélyes izgalom volt az, amit éreztem, hogy megvan, végre megvan! De nem volt öröm. Nem tudom egészen pontosan elmondani, mert nagy és ritka érzés volt. De olyan volt egészen, mintha félig nagy öröm lett volna, félig nagyon nagy szomorúság.

Aztán lassan odamentem hozzá. De minden lépéssel, amivel közelebb jutottam, nőtt bennem valami bizonytalanság, valami furcsa, tépő kétkedés. Megálltam mellette, lehajoltam. Nem élt már akkor. Megtört szemében zöldes színt játszott a fáradt napfény. Nyaka hátracsuklott, megtörten és szomorúan, finom, vékony lábai valami egészen gyermekes ijedséggel törtek meg a füvön s piros szőrű teste olyan árván simult a földhöz, mint egy letört, különös virág. Vér folyt az oldalából, nagy durva tócsa nőtt körülötte a gyepen s én csak álltam előtte és éreztem, hogy valami borzasztó tévedés történt.

Nem tudom, hogy hol és mit tévesztettem el, de éreztem valami megmagyarázhatatlan szorongást a torkomban s tudtam, hogy ami történt, azt többé soha jóvá tenni nem lehet. Mert az őzbak, amelyik véresen és holtan előttem hevert, nem volt az igazi. Csak egy árva kis halott őzike volt, üldözött, szomorú, finom kis állat. Nem az, akit én kerestem.

És akkor leültem az őz mellé és hang nélkül, csendesen elkezdtem sírni.

Így volt, bizony így. Régi történet ez, kicsit szégyellem is és nem szívesen beszélek róla. Mert közben sok idő eltelt s eldurvult bennem is valami és szégyellem azt, hogy valamikor egy őz miatt sírtam. Pedig ha elgondolom, nincs mit szégyelljek rajta. Szebb lenne és jobb lenne a világ, ha valamennyien leülnénk most is naponta egy-egy ilyen őz mellé és sírnánk, hang nélkül, csöndesen. Mert az életünk csupa ilyen tévedésből áll. Kergetünk titokzatos őzbakokat és mikor leterítjük, kisül, hogy nem az igazi. Jobb lenne és szebb lenne a világ, ha naponta leülnénk és elsírnánk tévedéseinket.

Mert mindannyian járjuk az élet bozótját, büszke és titokzatos őzbakokat keresve és egymásután zsákmányolva a tévedéseket. Életünk szűk tisztásain ilyenkor döbbenve megállunk egy percre, de aztán vállunkra akasztjuk a puskát és megyünk tovább új nyomot keresni, az igazit keresni.

Mint ahogy én is minden nyáron megindulok újra meg újra a Gyertyános vágottai felé, hogy hátha most, hátha most mégis megtalálom az igazit, amit akkor, régen, legelőször láttam. Langyos puha estéken járom a tisztások füvét, harmatos friss hajnalokon végigkeresem a vágottakat s álmodom, hogy keresek valami csodálatosat. Közben eltelik rendre az idő s egy este csak összeesek én is valamelyik tisztás közepén. Isten megáll mellettem puskával a vállán, megcsóválja a fejét és csalódottan mondja:

– Ejnye, ejnye, ez sem az igazi…

S ahogy tovalép, magával viszi a napot és olyan este lesz, amilyennek vége nincsen.  

1941
\end{document}

