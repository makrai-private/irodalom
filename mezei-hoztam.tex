\documentclass{IEEEtran}
\usepackage[utf8]{inputenc} 
\usepackage{times}
\usepackage[T1]{fontenc} 
\usepackage[magyar]{babel}
\author{Mezei Mária}
\title{Hoztam valamit a hegyekből\dots}
\date{2013, december 26 - 13:37}
\begin{document}

Mezei Mária: Hoztam valamit a hegyekből\dots
\vspace{1cm}

Megrendítően szép és felejthetetlen pillanat ez számomra. amikor most idelépek
az Úrasztala elé. Színésznő vagyok, sok-sok szerep van mögöttem, s most mégis
elszorult szívvel, akadó lélegzettel állok itt, mert először történik meg
velem, hogy nem színházban,
a mások élményeit és szavait közvetítem, ami tulajdonképpen a színésznő
feladata, hanem az Isten házában, a saját élményeimet a saját szavaimmal
próbálom átadni Önöknek. Bocsássák meg nekem ezt a merészséget és engedjék
meg, hogy belekezdjek ebbe a csendes beszélgetésbe, amit úgy kívánok, mint
soha semmit egész életemben.

Legelőször is kérni szeretnék Önöktől valamit. Én azt kérném
ha lehet, vessenek le most magukról minden cinizmust és gőgöt s hallgassák meg
az én szerény dadogásomat – szeretettel. Mert én most őszintén szeretnék
beszélni, olyan őszintén, ahogy talán életemben nem beszéltem, olyan őszintén,
mint ahogy a füvek és a. fák, a hegyek és a madarak beszélnek, akik között
közel két évig éltem. Ez a két esztendő volt, azt hiszem mindnyájunk életében
a legnehezebb, a legsötétebb élményekkel telezsúfolt idő.

Mindnyájan hoztunk valamit magunkkal ebből a pokoli két esztendőből, valami
végleges tapasztalást, ami irányítja azóta is az életünket. Én is hoztam. Én
azt, hogy itt állok és élek s ezt olyan nagyon nagy ajándéknak érzem, amit nem
fogadhatok el viszonzás nélkül, s amikor most Önökkel beszélgetni kívánok,
amikor verset mondok vagy énekelek, amikor színházat játszom, vagy egy
gyereket megsimogatok, úgy érzem, nem teszek egyebet, mint törlesztek –
törlesztem az adósságom, amivel tartozom annak a Valakinek, aki az életemmel
újból megajándékozott. S én úgy gondolom, s ezért ne haragudjanak rám, hogy
mindnyájunknak valahogy így kellene felfogni az ez utáni életünket:
ajándéknak, amiért hála jár és viszonzás. Meggyőződésem, hogy ez a végső
értelme ennek a különben olyan nagyon értelmetlennek látszó háborúnak.

Igen, én hoztam valamit a hegyekből. Hazahoztam a hitemet, mint egy nagy, jó
szagú meleg kenyeret, most szeretném, úgy, de úgy szeretném széjjelosztani. A
hitemet, hogy érdemes élni, de érdemes meghalni is, mert csodálatos törvények
igazítják mindnyájunk lépéseit s a halál nem pont az életünk mondatának végén,
legfeljebb pontosvessző, ami után újabb mondat következik.
Elhoztam a meggyőződésemet, hogy van egy nagyon nagy törvény, ami így szól:
semmi, ami történik veled, benned vagy körülötted, nem véletlen, hanem minden
tetted és gondolatod megtermi a következményét s ezért, bármi jó vagy rossz ér
az életben, annak előidézője te és csakis te vagy. S ha értetlenül állsz
szemben valami történéssel, valami méltatlansággal, szenvedéssel vagy magával
a halállal – légy meggyőződve róla, hogy feltétlenül megérdemelted, mert
okozóját, a törvényellenes tettet te követted el még akkor is, ha most már nem
emlékszel rá.

És elhoztam a legnagyobb törvényt, a törvények királyát, amiben minden benne
van, ami mindent betölt és mindent éltet, ami mindennek a megoldása, célja és
értelme – elhoztam a Szeretetet. Itt lakik bennem, benned, mindnyájunkban. A
füvekben és a fákban, a virágok illatában és a kutyád szemében, a kavicsokban
és a hegyekben, a csillagokban és a méhekben, a búzádra hulló esőben és a
gyermekedre hulló könnyben, az igazi dalban, az igazi csókban, mindenben, ami
van, amit látsz és amit csak sejtesz, mindenben a szeretet él – mert a
szeretet az Isten.

Szinte látom most egyesek arcán a gúnyos és főlényes mosolyt. Szegény teremtés
– gondolják -, milyen naiv és megszállott a hitével. Látom a másik megránduló
vállát, amint lerázza magáról a képtelennek és számára szinte már
nevetségesnek hangzó elavult szót: Isten. Látom a harmadik ökölbe szoruló
kezét, haragosan ránduló száját, amint fenyegetően dünnyögi: Istent mernek
emlegetni nekem, nekem, aki annyi szörnyű kínon és megaláztatáson mentem át?
Hát hol van az az Isten, aki eltűrte ezt a pokoli sok szenvedést és halált? De
látom azt a néhány kifényesedő szemet és kisimuló arcot is, akik a saját
lelkük halkan éneklő hangját hallják visszhangozni, amikor elsóhajtom a szót:
Isten. Látom mind a négyfajta embert: a Gőgösöket, a Bosszúállókat, a
Kételkedőket és a Hivőket. Jól látom őket, mert vándorlásom alatt sokszor
találkoztam mindegyikükkel. Akkor hallgattam, kérdeztem, figyeltem őket, most
megpróbálok felelni nekik.

Mondd, te szegény, gőgös ember, miért rángatod a vállad? Mire vagy olyan
gőgös, mondd? Az erődre talán? Hát nem láttad, hogy fosztja le az éhség
csontjaidról a jól ápolt és agyontornáztatott izmokat? Nem láttad, hogy mállik
szét imádott tested a gyűlölet által gyártott fegyverek egyetlen kis
szilánkjától? Nem láttad, hogy dőltek össze palotáid, villáid, gyáraid,
hidaid, nagyszerű erőid minden produktuma?
Mert a gyűlölet szórta a bombát, s a sok büszke emberi alkotás összeomlott,
mint a kártyavár. Nem merült fel benned egy pillanatra sem a kétely, hogy
talán, talán nem volt jó az alap, amire építettünk eddig? Mert nézz csak szét
a romokon, pesti embertársam – nem furcsa az, nézd csak -, hogy a
templomtornyok mind-mind milyen egyenesen állnak? Mint összetett, imádkozó
kezek nyúlnak fel a romok közül az égre…
Mosolyogsz? Nem hiszel a jeleknek? Kár – nagy kár, mert a Jelek és Csodák
vezethetnek már csak ki bennünket abból a sötét és félelmetes rengetegből,
ahová az emberi értelem bevezetett s ottfelejtett minket.

Vagy a vagyonodra vagy gőgös, te szegény vállvonogató? Hát van még vagyonod ?
Vagy ha már nincs, gondolod, hogy ha újra összespekulálod, összefeketézed,
összeházasodod, összeirigykeded? Gondolod, hogy most majd meg fogod tudni
tartani? Nem, és ezerszer is nem! Ez az egyik legnagyobb értelme ennek az
egész elmúlt szörnyűségnek, hogy ráébredjünk végre: csak az marad meg az
anyagi javakból, amiért becsületes és jó szándékú munkával, mi magunk
megdolgoztunk. A többit elviszik, elszórják, felégetik, mind. Nem fontos
kicsoda, milyen jogcímen és milyen jelszó védelme alatt. A jelszavak és emberi
szándékok mögött az isteni szándék rejtezik és könyörtelenül csak annyit hagy
meg mindenkinek, amennyi valóságosan jár neki és amennyire valóságosan
szüksége van ahhoz, hogy az Úton tovább tudjon menni. És vigyázz, ember, minél
kevesebb az úti csomagod, annál közelebb vagy a Célhoz, s aki mindenét és
mindenkijét elvesztette, az nagyon figyeljen, áhítatosan figyeljen, mert ahhoz
már nagyon közel van a Cél: ahhoz már nagyon közel hajolt az Isten.

Te Gőgös, még mindig hozzád szólok. Hozzád, aki nagyhatalmú, tekintélyű, magas
rangú hivatalnok voltál. Te voltál az, aki felvetett fejjel büszkén vitted
feketére festett bajuszodat s csak minden tizedik köszöntést fogadtad kegyes
biccentéssel. Emlékszel-e rám? Mert én nagyon jól emlékszem rád. Ott ültél
szemben velem a teherautón, ami vitt bennünket az ismeretlen felé – akkor úgy
éreztük, a halál felé. Emlékszem alázatos tekintetedre, a szemed sarkán
kibuggyanó könnyre, amivel elhagyott gyermekeidet sirattad. Gyűrött . arcodra,
kopott bajuszodra, amiről úgy kopott le a festék, mint lényedről a hazugság.
Ó, hogy megszerettelek akkor, te szegény-szegény Gőgös – ott, azon az úton
kezdtél ember lenni végre. Én azon az úton, az ismeretlen felé vezető úton
találkoztam az Istennel, s visszajöttem vele, általa és érte.

Téged otthagytalak egy ablakban, riadt szemekkel s alig pislogó reménnyel. De
hiszem és remélem, hogy te is megtalálod az utad értelmét s akkor visszajössz
közénk – őszen és őszintén igazi emberként.

S most hozzátok szólok, öklöt rázó szegény bosszúállók. Ti vagytok mindnyájunk
közül a legsajnálatraméltóbbak, mert a tiétek a legnehezebb szerep. Most rátok
gondolok, akiknek valóban volna miért bosszút állni. Rád, te sárga arcú,
fáradt tekintetű Kapitány, akinek mindenkijét megölték egy jelszó nevében, s
ki most azzal próbálod vigasztalni magad, hogy te kínzol másokat. Megértelek,
de végtelenül sajnállak. Sajnáltalak már akkor is, mikor gúnyos mosollyal
figyelted tántorgásomat a hajnali országúton, ahogy a többi szerencsétlen
között, kínlódva cipeltem a batyumat s még nagyobb kínnal cipeltem magamban az
értetlenséget, hogy miért, miért kell nekem, éppen nekem itt vánszorogni ?
Mert akkor még nem értettem meg az igazi tartalmát annak az unalomig ismert
bibliai mondatnak, hogy “az Isten útjai kifürkészhetetlenek”. Akkor lázadtam,
vergődtem, kínlódtam s homályos szemekkel láttalak csak, Kapitány, de
haragudni akkor sem tudtam rád, amikor – megtaszítottál. Szereped olyan nehéz
lehet, s milyen boldogtalanná tehet téged. Látod, tőlem lassan vett el a jó
Isten mindenkit, akiket szerettem, de amíg meg nem bocsátottam Neki és a
világnak azt, hogy egyedül maradtam – addig mindig szomorú, sápadt és
boldogtalan voltam -, hasonlítottam egy kicsit hozzád, Kapitány.

És te, szerencsétlen, hamis fogú, cseh nyelvű szabó, akivel soha még szót sem
váltottam, mondd, ki öntötte beléd azt a mérhetetlen és fékezhetetlen
gyűlöletet irántunk, magyarok iránt, hogy engem, akit nem is ismertél, meg
akartál öletni csak azért, mert magyar vagyok ? Lehet – mondd, lehet – hogy mi
magunk, magyarok ? Félek, nagyon félek, hogy igen. Talán az apáink vagy az
ükapáink voltak igaztalanok, értetlenek, szeretet nélkül valók őseiddel, s
most termi meg a régi rossz mag nekünk, későbbi utódoknak a gyűlölet
gyümölcsét. – Jaj, mennyi szeretet és mennyi türelem kellene, hogy
kiengesztelhessük azt a sok-sok megsértett lelkű, elcsorbított mosolyú cseh és
más nyelvű szabót! – Mennyit? Pontosan annyit, amennyi gyűlöletet adnak ők
nekünk. Mert ez törvény : a sötétséget csak a világosság, a gyűlöletet csak a
szeretet oszlathatja el.

S most nektek üzenek furcsa, hangoskodó, kicsi, úgynevezett “emberbarát”
családok. Nektek, akik alig vesztettetek valamit. Megvagytok mindnyájan,
megvan a házatok, a bútorotok, megvan a munkátok és a bundátok, de a hiányzó
zöld sál miatt keservesen siránkoztok s a hiányzó zöld sál nevében kértek el,
amit lehet a hozzátok, “emberbarátokhoz” menekülő vándortól. Ti, akiknek a
gyermeke sápadtan és gúnyosan mosolyog – ti vigyázzatok! Mert az Isten
szeretet, igen – de könyörtelen szeretet, aki előbb csak a zöld sálat veszi el
tőled, de ha ez nem elég neked, elveszi a bundát, a bútort, a házat, letörli a
gyermeked gúnyos mosolyát és semmilyent sem fest helyette, téged pedig addig
őröl, amíg ki nem sajtolja belőled egy kis cseppjét a szeretetnek. De addig,
addig nehéz lesz az út.

Ezért üzenem hát, hogy vigyázzatok ! Ne sirassátok a zöld sálat, hanem ha van
még egy piros, adjátok oda annak akinek semmilyen sincs. Talán – talán akkor
előkerül egyszer még a zöld is.

S most te figyelj egy kicsit, te Kételkedő. Te, aki már nem érzed jól magad a
világban. Te, aki nyugtalan vagy, ingerült, rosszkedvű, mert állandóan
hiányzik neked valami. Te, aki kábulttá iszod, dohányzod, csókolod, dolgozod
magad, csak hogy elhallgattasd a Hangot, ami kiabál benned a megnevezhetetlen,
megfoghatatlan, elérhetetlen valami után, amit nem találsz meg se borban, se
dohányban, se csókban, de – nem találsz meg a görcsös, lihegő munkában sem.
Te, aki már sejted, hogy tulajdonképpen Istent keresed, de Tamás-lelked jelt
kíván és csodát követel ahhoz, hogy hinni tudjon. Te – aki én is voltam valaha
– figyelj ide! Tudod-e, mi a csoda?
A Csoda az, hogy élünk! Hiszem és vallom, hogy külön-külön mindannyiunknak,
akik életben maradtunk, személy szerint fogta. az Isten a kezünket, hogy
megmentsen bennünket – s ő tudja, hogy miért! Mert Ő volt az, aki idejében
elvezetett engem Pestről, Ő volt az, aki betegséget adott nekem, hogy még ha
akarnék se tudjak visszatérni. Ő volt az, aki felfogta körülöttem a sivító
golyókat, mikor az emberi közömbösségtől kergetve egyedül és kétségbeesetten
futottam az erdőn. Ő volt az, aki a kellő pillanatban eltörte a lábam, s ezzel
két hónapi menedéket adott a csendőrös hazatoloncolás elől. Ő volt az, aki
elaltatott egy éjszakán, amikor mindenki reszketve várta a halált hozó
robbanást. És Ő volt az, igen Ö volt az, aki messze, künn Lengyelországban egy
kicsi ház lépcsőjén várt rám ahová egy szál ruhában, minden reményt és minden
segítő szeretetet messze magam mögött hagyva, hihetetlen és halált jelentő
váddal terhelve, roskadó lábbal és összeroskadt hittel egy furcsa fényű havas
délután beléptem. Ott állt a lépcsőn – idegen egyenruhában – rám nézett
hosszan s csodálkozó hangon magyarul megszólalt: “Hogy kerül maga ide M. M. ?”
Óh ne, most ne mondjátok azt, hogy “érdekes véletlen”. Ez nem “érdekes” és nem
“véletlen”. Ez a Csoda volt. Mert nem csodálatos-e az, hogy azt a teherautót
máshova küldték, s mégis “véletlenül” éppen ennél a kis háznál állt meg? Nem
csodálatos-e, hogy éppen az a pesti zsidófiú állt ott a lépcsőn; aki ismert, s
tudott minden úgynevezett “jót” rólam, jobban emlékezett minden emberséges kis
cselekedetemre, mint én magam. Nem csodálatos-e elgondolni, hogy ha nem éppen
ez a fiú áll ott, akkor távolban minden igazoló lehetőségtől, az olyan váddal
szemben, mint amivel engem a gyűlölet megvádolt, bizony nagyon röviden szoktak
intézkedni?
Nem, ez nem “érdekes véletlen”, ez igenis a térdre kényszerítéses, „égzengéses
Csoda” volt, amire vártam, amiért kínlódtam egész életemben, ami feloldott
minden kételyt és bizonytalanságot bennem. A Csoda, ami véglegesen és
megmásíthatatlanul rávezetett az Útra, amin mennem kell. A Csoda, ami világít
bennem és világít előttem, hogy soha-soha többé el ne tévedhessek. – Hogy mi
volt az ára, nem fontos, ha százszor ennyi lett volna, mind kevés lett volna.
És köszönök minden gonoszságot, minden ridegséget, minden megalázást, minden
ésszel fel nem fogható rosszakaratot, mert minderre szükségem volt, hogy
elkerülhessek a lengyelországi kis ház lépcsőjére s ott szemtől szembe
állhassak a könyörülő Istennel.

És ez a csoda nemcsak velem történt meg. Megtörtént veled is, te Kételkedő,
megtörtént mindnyájatokkal Gőgösek, Bosszúállók és Hivők, hiszen másként nem
ülhetnétek itt! Nézz magadba Ember, és nézz vissza az elmúlt esztendőkre.
Lehetetlen, hogy olyan érzéketlen, olyan elvakult, eltompult légy, hogy ne
lásd meg a te személyes életmentő csodádat. Meg vagyok róla győződve, hogy nem
egy, de tíz úgynevezett “véletlent” találsz, ami ha nincs – ma már te sem
vagy.

Nézz magadba s azután nézz az égre és halkan, csendesen mondd ki a szót :
Köszönöm. – S a pillanatban, mikor ezt kimondtad, megtelsz hittel, megtelsz
örömmel, megtelsz hálával, megtelsz erővel, megtelsz szeretettel: betelsz az
Istennel.

Ó igen, hiszem és vallom, hogy a végső értelme ennek az elmúlt időnek: megélni
a nagy csodát – életben maradni, a csodán keresztül megtalálni Istent, aki
maga a Szeretet, betelni Istennel, betelni szeretettel, a szeretet
segítségével rálépni az Útra, a szeretet útjára s elvégezni a Feladatot, azaz
leélni az életet a szeretet nevében. Igen, ki merem mondani: Isten országa
soha olyan közel nem volt hozzánk, mint éppen most. Isten országa, ami bennünk
van, s ami egyenlő az új ember, a szeretettel teljes ember megszületésével.
Most, igen most, amikor úgy érzed és úgy látod, hogy eltűnt a szeretet a
földről, mert eltűnt mellőlünk a sok szeretetforrás és szeretetmankó, eltűnt
az anyánk, az apánk, a gyermekünk, eltűnt a szerelmünk, eltűnt a barátunk s
árván és önzően sírunk a szeretet után. Tudod-e, mi az értelme ennek a nagy
árvaságnak?

Éhessé és szomjassá tett bennünket az Isten a szeretetre, azért vett el
körülöttünk minden szeretetforrást, hogy szomjas kínunkban végre önmagunkban
próbáljunk meg kutatni utána. S ekkor ébredünk rá a legnagyobb Titokra, hogy a
Szeretet, mint mag a földben, bennünk alszik, de csak akkor ébred és csak
akkor éltet, ha önzetlenül tovább adjuk másnak. Mert az önző, befelé irányuló
szeretet – az önszeretet – nem csak kifelé pusztít, pusztít befelé is –
elpusztít végül Téged is. De az önzetlen, kifelé irányuló szeretet – a
felebaráti szeretet – épít és éltet másokat s magadat egyaránt. Ezt az
országot elpusztították, az egyéni, osztályi, faji, nemzeti önszeretet
nevében. Újjáépíteni a magunk és mindenki számára egészséges, boldog országgá
újjáteremteni, az önzetlen, mindenkire kiterjedő felebaráti szeretet nevében
lehet! Az Isten nevében lehet – mert az Isten: a Szeretet, s bennünk,
magunkban a szeretet az az isteni mag, ami a mi létünk legbelsőbb lényege, s
az emberi élet értelme nem lehet más, mint felfedezni ezt a magot, táplálni,
öntözni -, hogy felnövekedve betöltse egész valónkat s ilyen módon
beleolvadjunk az isteni szeretet egészébe, belesimuljunk a világ harmóniájába.
S tudod-e, mi ez az állapot kedves embertársam? Ez az amit úgy hívunk:
boldogság, teljesség, – ez az üdvösség.

Boldog akarsz lenni? Először is kezd el Te magadat újjáteremteni. Lépésről
lépésre haladj befelé és tégláról téglára bontsd le magadban az önzés falát.
Legelőszőr is tanuld meg a lemondást, – ha még ez a szörnyű háború-iskola meg
nem tanított volna rá. Mert hogy iskola volt, azt ugye mindnyájan tudjuk.
Félelmetes szigorú iskolában ül az egész emberiség, ahol a fő tantárgy a
szeretet; s ahol könyörtelenü1 le kell vizsgáznia mindenkinek, ha felsőbb
osztályba akar lépni – ha boldogabb, teljesebb, emberhez méltóbb életet akar
ezután élni.

Tehát: lemondani. De nem fogcsikorgatva, átkozódva, hanem jókedvűen
mosolyogva, – önként lemondani. Először csak kevésről, később mindig többről,
s mikor megtanulsz önként, jószántodból mindenről lemondani, akkor, de csakis
akkor megkapsz újra mindent.

Mert a szeretet első törvénye így szól: a szeretetet adni kell, csak akkor
kaphatjuk. S minél többet adsz belőle, annál több lesz neked, magadnak belőle.
Mert az önzetlen szeretet kimeríthetetlen forrásból táplálkozik: a világot
éltető, mozgató, fenntartó isteni szeretetből, s mikor adsz valamit – pénzt,
ruhát, munkát, mosolyt vagy akárcsak egy simogatást -, akkor láthatatlan
sugarak kapcsolnak össze ezzel a legfőbb forrással, s Te csak mint közvetítő,
összekötő vezeték, felfogod és továbbadod az “életnek vizét”, ami soha el nem
apad, amiről Krisztus beszélt, s ami a szeretet.

Hihetetlen, elképzelhetetlen erő az önzetlen szeretet. Csak megmozdul benned s
máris megváltozik körülötted a levegő. Megváltozol elsősorban te magad.
Letörli arcodról a keserű vonást, kisimítja szemed alól a ráncokat, belelop a
szemedbe valami furcsa derűt, amitől fényleni kezd az arcod is – fényleni kezd
az egész lényed is. Újraépíti egész testedet – valóságosan új embert épít
belőled. Jó étvágyat ad és nyugodt álmot. Igazi békét ad és igazi örömet. Úgy
nevetni és úgy nevettetni senki sem tud, mint akiben felébredt a szeretet.
Mert az igazi színház is tulajdonképpen szeretetadás: vigasztalás és örömet
okozás.
S tudod-e, hogy az élő szeretet megváltoztatja körülötted az embereket is.
Mert az is a szeretet törvénye: akinek egyszer szeretetet adtál, az előbb vagy
utóbb kénytelen azt továbbadni – de vigyázz -, nem neked adja vissza, hanem
valaki másnak. Hiszen éppen az a szép, hogy egy kevés élő szeretet végtelen
láncolatát indítja el a szeretet átadásának, és egyszer valahonnan, valakitől
– amikor a legnagyobb szükséged van rá – megnövekedtett formában szépen
visszakerül hozzád.
Ugye, ti fénylő szemű hivők, ti, akikben már megszületett a szeretet, ugye, ti
tudjátok, hogy ez az igazság?

Óh, te kedves-édes Hermann Boriska, kicsi parasztasszony ágyszomszédom a
kassai kórházban, akit hajnalban már felvert az aggodalom s beteg szívét
meghajszolta a félelem, hogy mi van az ő három kis gyermekével otthon, s mi
lesz a negyedikkel, akit beteg szíve alatt hordott, ugye, te tudod, Boriska,
hogy milyen “jó üzletet” csináltam én veled, amikor neked adtam a piros
kabátom? A piros kabátot, ami olyan vonzóan és csábítóan piros volt, hogy
egész nap le nem tudtad a szemed venni róla. Dehogyis merted volna kérni. Csak
vágyódva nézted, csodáltad és sírva fakadtál, mikor a kezedbe nyomtam.
De utánam jöttél, napok múlva megkerestél s édes kis kékkarikás arcocskádban,
puha kis csókodban elhoztad nekem az én soha nem látott édesanyám csókját – s
meleg, hívó szavadban, hogy jöjjek hozzád falura, feküdjek ott nálad –
elhoztad az én régen elsüllyedt igazi otthonom melegét. Óh te kedves, hálás
kicsi lélek, aki férjed helyett szántottál-vetettél, magad helyett főztél
mostál, gyereket dajkáltál s beteg szívvel gyereket vártál, – engem is
vállalni akartál, csakhogy megháláld a piros kabát örömét.

Boriska, szívem, üzenem neked; hogy a tartozásod ki van egyenlítve. Mert
befogadott engem, mikor a legnagyobb szükségem volt rá – egy kicsi testvéred,
a Kababik Rozika, akinek az arca éppen olyan kedves, mint a tied, a szeme
éppen olyan tiszta; s akiben ugyanaz a szeretet él, ami benned, Boriska.

Ugye, te törött tekintetű, reszkető kezű, repedezett: körmű furcsa kicsi
ember, aki a poprádi utcasarkon megfogtad a kezem s nem kérdeztél semmit, csak
felsegítettél az idegen gépkocsira. – aki takargattál, enni adtál, pedig te is
rongyos, te is éhes voltál, hiszen nagyon messziről, egy haláltáborból jöttél
vissza – ugye tudod, hogy a te jóságodat valahol, valaki azóta már visszaadta
a te anyádnak, a te nővérednek, a te gyermekednek.

És te, kedves Tatár, aki őrünk voltál, aki mindig nevettél, és mindig
énekeltél. Raktad a tüzet egész éjjel; hogy mi, szegény foglyok, jaj, csak meg
ne fázzunk – aki enni hoztál, és cserébe semmit se vártál -, talán te nem is
tudod, hogy a szeretet énekelt benned. Te csak élsz és dalolsz, mint a
madarak, s a madarak biztos ösztönével végzed feladatodat a világban.

És te, idegen lány az idegen városban, aki kézen fogtál, lefektettél,
simogattál, az egyetlen megmaradt cipődben az én dolgaim után szaladgáltál –
ugye tudod, hogy a sok megaláztatás és bujkálás alatt lopakodott beléd a
szeretet, amit aztán rám pazaroltál.

Ó igen, adni kell, adni, mindent odaadni. Munkát, erőt, életet. Pénzt, ruhát
és kenyeret. Könnyet, mosolyt, simogatást. Jó szándékot, jó akarást,
imádságot, egészséget. Melegséget, élő hitet. Odaadni akárkinek, a legelső
nincstelennek, a náladnál szegényebbnek. Szeretettel adni és érte semmit sem
kívánni.
Így kell élni.
S ha valóságban így fogsz élni, akkor, de csakis akkor mindent, de mindent
vissza fogsz kapni. Amiről álmodtál, valósággá válik. Amire vágyódtál – elébed
hozzák. Akit elvesztettél – újra megtalálod. Mert megtalálhatod anyád
simogatását egy idegen asszony kezében, elhozhatja neked eltűnt kedvesed
csókját egy idegen csókja, s megvigasztalódhatsz egy idegen gyermek mosolyától
is. De csak akkor, ha adtál valaha valakinek valamit olyan szeretettel, mintha
valóságos anyád, valóságos fiad, valóságos kedvesed lenne.
Mert ne felejtsd el soha, hogy mindnyájan egyek vagyunk, mindnyájan testvérek
vagyunk, és testvéreink a fák, a füvek, a kutyák, a kövek, a hüllők, a
bogarak, a csillagok s a madarak. Mindnyájan testvérek vagyunk – mert
mindnyájan a Szeretet teremtményei vagyunk. S tudjátok-e, hogy aki a
szeretetben él, annak többet soha semmitől sem kell már félnie. Nem kell
töprengenie, hogy jól csinál-e valamit vagy sem, mert nem ő cselekszik már,
hanem a szeretet cselekszik őbenne. Annak a lábát puha kezek fogják s szépen
igazítják, ne ide lépj, – nézd csak – inkább ide, így jobb. S ha
elhomályosodik előtte az Út, kis jelek és csodák gyúlnak ki a sötétben s
világítanak a léptei előtt.

Elmesélek most egy ilyen “kis csodát”, amit akkor kaptam, mikor egyszer nagyon
elfáradtam. Olyan fáradt voltam. hogy úgy éreztem, mégis legjobb lenne
lefeküdni az út mellett s hagyni, hogy az élet menjen szépen tovább –
nélkülem.

Kórházban feküdtem akkor. Egy reggel valami furcsa kis nyugtalanságtól hajtva
bújtam ki az ágyból s minden cél nélkül kisiettem a folyosóra. Mintha . . .
igen, mintha hívott volna valaki. Egy kis szőke, rongyos, angyalszemű fiú sírt
az egyik sarokban. Egyik kezecskéjét elvitte az akna és az anyja szidta,
szidta, hogy milyen rossz fiú, neki ilyen nagy bajt okozott. Behívtam őket
magamhoz. Az asszony szegény – favágó felesége. Férjét elvitték, még négy
lánya van, s ez az egy fia most ilyen költséges bajba sodorta, hiszen nincsen
pénz még vonatra se, hogy kötőzésre tudjanak bejárni.

Szerényen mondta – hát adtam neki. Sírni kezdett. Hihetetlen, mennyire
elszoktak az emberek egy szemernyi jóságtól. Tiltakozott, s a végén erővel a
kezembe nyomott egy kis csomag maradék vajat amit kopott kiskosarából kapart
elő. Elmentek, s ahogy kibontom a piszkos hártyapapírt az egyik oldalán
szépséges arany iniciáléval, égszínkék betűkkel ráírva ezt olvasom : ,,Az
odafelvalókkal törődjetek, ne a földiekkel ” (Kol. 3,2)

Óh, hát elfáradhat-e az, félhet-e az valamitől, akinek – levelet írt az
Úristen?
És most végezetül nagyon halkan szeretnék még valamit mondani. Meg kell
tanulnunk újra imádkozni. Hihetetlen erő az ima. Még egy halk sóhajtás is
felmérhetetlen erőket hív életre. Ne szégyelljük magunkat, sóhajtsunk fel:
“Édes jó Istenem, segíts meg bennünket.”

És az Isten odahajol hozzád és segíteni fog rajtad – mint ahogy segített azon
a kicsi fiún is, akit egy éjjel összeroncsolt hassal hoztak be a műtőbe,
akiről lemondott az apja de lemondott az orvosa is s akinek a műtőasztalon
súgtam a fülébe: „Meglásd megsegít az Isten.” S ő kékülő ajakkal,
kétségbeesetten jajongó hívással kiáltotta: “Istenem, Istenem, Istenem!” És az
Isten odahajolt a műtőasztal fölé, megfogta a sebész elcsüggedt kezét,
vezette, megsimogatta az irtózatos sebet, rálehelte az Ő titokzatos melegét
erre a szegény kihűlő kis testre. S a kicsi Tatárka György tíz nap múlva,
jókedvűen szaladt a kertben, cseresznyét majszolt, s amikor megkérdeztem, hogy
mit gondol, vajon ki segített rajta – lesütött szemekkel, a zavartól pirosan,
halkan azt felelte: „Az Isten”.

Igen. Ő segített rajtad, kicsi Tatárka György, s Ő segít, csak Ő segíthet
rajtunk is, mindnyájunkon. Ezen a fölsebzett országon, ezen az egész irtózatos
világon. Hívjuk hát, hogy jöjjön, jöjjön ide hozzánk. Hajoljon ide hozzánk,
simogassa le rólunk a közönyt, a gőgöt. Nyissa ki a szívünk rozsdás ajtaját,
lépjen be és maradjon bennünk. Hogy építhessünk neki magunkból templomot, s a
sok kicsi ember-templomból építhessünk egy új Országot, aminek az alapja a
szeretet, s tartalma a szeretet, a koronája a szeretet legyen. Mert nincsen
nagyobb hatalom, és nincsen nagyobb erő, mint a szeretet – mert a szeretet
Isten. Jöjj hát Szeretet, hogy legyen a mi lelkünk a te templomod, s ha
sírunk, ha nevetünk – ha imádkozunk vagy bukfencet hányunk, mindig a Te
törvényedet szolgáljuk: melegséget árasszunk, örömet osztogassunk, szeretetet
kínáljunk.

Én Édes Istenem, segíts hozzá minket!

2013. december 26.


\end{document}
